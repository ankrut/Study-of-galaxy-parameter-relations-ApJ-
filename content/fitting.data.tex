\subsection{Data selection}
\label{sec:data}
The SPARC data is distributed in separated files and can be found at \url{http://astroweb.cwru.edu/SPARC/}. Specific information about each galaxy (i.e Hubble type, inclination etc) are provided in the file \href{http://astroweb.cwru.edu/SPARC/Table1.mrt}{Table1.mrt}. The information we are interested in, like galactocentric radius $r$ and rotation curves $\SYMvel$, are provided in the file \href{http://astroweb.cwru.edu/SPARC/Table2.mrt}{Table2.mrt}.

In detail, we extract the observed circular velocity $\SYMvobs$ and the baryonic contribution $\SYMvbar$, composed of a bulge ($\SYMvbulge$), disk ($\SYMvdisk$) and gas component ($\SYMvgas$). The bulge and disk components are inferred from surface brightness observations for a given mass-to-light ratio. In sum, the baryonic component is given by \begin{equation}
	\label{eqn:baryonic-sum}
	\SYMvbar^2 = \Upsilon_\mathrm{b}^{\phantom{2}} \SYMvbulge^2 + \Upsilon_\mathrm{d}^{\phantom{2}} \SYMvdisk^2 + \SYMvgas^2
\end{equation} For convenience, the velocities $\SYMvbulge$ and $\SYMvdisk$ are provided for a mass-to-light ratio of $1\,M_\odot/L_\odot$ what does not represent the real value for a galaxy. Since the mass-to-light ratio is just a constant scaling factor we may correct the velocities simply with the mass-to-light \textit{ratio factors} $\Upsilon_b$ and $\Upsilon_d$ for bulge and disk (in units of $M_\odot/L_\odot$). Then the rotation curve for each component traces immediately its centripetal acceleration $\SYMacc = \SYMvel^2/r$.

For the data selection we proceed similar as was done by \citet{2016arXiv160905917M}. Thus, we want to note that we consider same mass-to-light ratios since the following data selection output depends on the values. For all bulges we choose $\Upsilon_\mathrm{b} = 0.7$ and for all disks $\Upsilon_\mathrm{g} = 0.5$ as convenient average representatives. Further, we exclude all galaxies with a bad quality flag ($Q=3$) and face-on galaxies with an inclination $i<30\degr$. Then we exclude all points with a velocity error greater than $10\%$ and all points where the baryonic velocity is greater than $95\%$ of the observed velocity. The latter affects mainly data points in the inner region which is dominated by baryonic matter and strongly depend on the chosen mass-to-light factors. Afterwards, we exclude all remaining galaxies with less then 6 data points and obtain 124 galaxies (out of 174) with 2396 points (of 3355) in total.