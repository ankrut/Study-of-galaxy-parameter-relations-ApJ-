\begin{abstract}
Galaxies show different scaling relations such as the Radial Acceleration Relation, the Mass Discrepancy Acceleration Relation (MDAR), the dark matter (DM) surface density relation, or the Ferrarese relation connecting the total halo mass and its supermassive central object. \add{At difference with traditional studies using phenomenological $\Lambda$CDM halos, we analyze the above relations assuming that DM halos are formed through a Maximum Entropy Principle (MEP) in which the fermionic (quantum) nature of the DM particles is dully accounted for. Key aspects of this novel approach are: (i) for the first time a competitive model based on first physical principles such as (quantum) statistical-mechanics and thermodynamics is tested against a large data-set of galactic observables; and (ii) such a fermionic model makes theoretical predictions in the central and outermost regions of the DM profiles (not feasible in $\Lambda$CDM halos) while leading to a good agreement with observations. In particular, we compare the fermionic model with other DM profiles}: the NFW, a generalized NFW accounting for baryonic feedback, the Einasto model and the Burkert model. For this task, we use a large sample of $120$ galaxies taken from the Spitzer Photometry and Accurate Rotation Curves (SPARC) data-set, from which we infer the DM content to compare with the models. We find that the Radial Acceleration Relation and MDAR are well explained by all the models with comparable accuracy, while the fits to the individual rotation curves, in contrast, show that cored DM halos are statistically preferred with respect to the cuspy NFW profile. %However, very different physical principles justify the flat inner halo slope in the most favored DM profiles: generalized NFW and Einasto models rely on complex baryonic feedback processes, while the MEP scenario involves a quasi-thermodynamic equilibrium of the DM particles.
\end{abstract}
% Galaxies show different scaling relations such as the Radial Acceleration Relation, the Mass Discrepancy Acceleration Relation, the dark matter (DM) surface density relation, or the Ferrarese relation connecting the total halo mass and its supermassive central object. These relations are usually explained in terms of $\Lambda$CDM halos including for baryonic feedback.
%
%Here we analyze all of them assuming that DM halos are formed through a Maximum Entropy Principle (MEP), and we show that they are compatible with a fermionic nature of the DM particles.
%
%
%The understanding of the Radial Acceleration Relation and MDAR is backed here by comparing the goodness of fit