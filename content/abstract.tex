\begin{abstract}
The study of dark matter (DM) and the way it affects the rotation curve of disk galaxies has unveiled different intricate relations between the dark and baryonic components on galactic scales. Recently, the radial acceleration correlation, found in the SPARC data set by \citet{2016arXiv160905917M}, has regained great attention. We show that this correlation and its characteristic scatter can be explained with dark matter composed of self-gravitating fermions in spherical symmetry including the effect of escaping particles.
%
%Analyzing individual galaxies we conclude that the correlation is more intricate than the proposed empirical based on the average of many disk-galaxies. Thus, galaxies show characteristic deviations in the very low and high acceleration regimes what we interpret as the source of the scatter in the correlation.
%
The understanding of the correlation is backed by a goodness of model analysis for 124 filtered galaxies of the SPARC data set, covering different Hubble types. Besides the fermionic DM model, the same kind of analysis is done also for NFW and DC14 models in order to properly compare among them. A better understanding of the nature of the scatter is then given by the analysis of individual galaxies.
%
Following the fermionic DM model (RAR model), we predict several relations for different pairs of structural galaxy parameters including a massive compact core in every galactic center. Of special interest are the $M_{\rm BH}$-$M_{\rm tot}$ relation and the constant surface density law. Recently, \citet{arguelles_novel_2018} showed that the RAR model with a dark matter particle mass of $\sim$50\,keV is able to explain them. Here, we are going to enhance these relations with predictions inferred from disk-galaxies of the SPARC data, being below the observable window of the $M_{\rm BH}$-$M_{\rm tot}$ relation.
\end{abstract}