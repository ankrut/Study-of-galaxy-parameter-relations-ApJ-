\begin{abstract}
Galaxies show different halo scaling relations such as the Radial Acceleration Relation, the Mass Discrepancy Acceleration Relation (MDAR) or the dark matter Surface Density Relation \add{(SDR)}. At difference with traditional studies using phenomenological $\Lambda$CDM halos, we analyze the above relations assuming that dark matter (DM) halos are formed through a Maximum Entropy Principle (MEP) in which the fermionic (quantum) nature of the DM particles is dully accounted for. For the first time a competitive DM model based on first physical principles, such as (quantum) statistical-mechanics and thermodynamics, is tested against a large data-set of galactic observables.
In particular, we compare the fermionic DM model with empirical DM profiles: the NFW model, a generalized NFW model accounting for baryonic feedback, the Einasto model and the Burkert model. For this task, we use a large sample of $120$ galaxies taken from the Spitzer Photometry and Accurate Rotation Curves (SPARC) data-set, from which we infer the DM content to compare with the models. We find that the Radial Acceleration Relation and MDAR are well explained by all the models with comparable accuracy, while the fits to the individual rotation curves, in contrast, show that cored DM halos are statistically preferred with respect to the cuspy NFW profile.
\add{However, very different physical principles justify the flat inner halo slope in the most favored DM profiles: while generalized NFW or Einasto models rely on complex baryonic feedback processes, the MEP scenario involves a quasi-thermodynamic equilibrium of the DM particles.}
% (ii) such a fermionic model makes theoretical predictions in the central and outermost regions of the DM profiles (not feasible in $\Lambda$CDM halos) while leading to a good agreement with observations.
%\add{Further, we show that fermionic DM is consistent with SDR and that particular solutions of the rich morphology of fermionic DM halos can be associated with different DM models depending on the particle evaporation.}
\end{abstract}