\begin{abstract}
%Disk galaxies show a tight non-linear correlation between the radial acceleration caused by the total matter and the one generated by its baryonic component only. This relation, which holds at any resolved galactocentric radius and for different Hubble types, is usually explained in terms of $\Lambda$CDM halos including for baryonic feedback, or within the MOND theory of gravity. Here, we analyze such a correlation assuming that dark matter (DM) halos are formed through a Maximum Entropy Principle (MEP), and we show it to be compatible with a fermionic nature of the DM particles. \add{In addition to the Radial Acceleration Relation, we study other scaling relations to further test the later model, such as the Mass Discrepancy Acceleration Relation, the DM surface density Relation and the Ferrarese relation connecting the total halo mass and its supermassive central object.} The understanding of the \add{acceleration relations} is backed here by comparing the goodness of fit of different DM halo models: the traditional NFW, a generalized NFW accounting for baryonic feedback, the Einasto model, the Burkert model, and the fermionic halos obtained from entropy maximization. For this task, we use a large sample of \add{$120$ galaxies} taken from the Spitzer Photometry and Accurate Rotation Curves (SPARC) data-set, from which we infer the DM content to compare with the models. We show that the \add{above two} empirical acceleration \add{relations are} well explained by all the models with comparable accuracy, \add{but instead} the fits to the individual rotation curves show that cored DM halos (\add{like Einasto or the fermionic model}) are statistically preferred with respect to the cuspy NFW profile. \add{Though} very different physical principles justify the flat inner halo slope in the most favored DM profiles: while the generalized NFW \add{or Einasto models rely} on complex baryonic feedback processes, the MEP scenario involves a quasi-thermodynamic equilibrium of the DM particles. %Further, we extract characteristic galaxy properties from the center to the periphery as obtained for the fermionic DM model and show that they are consistent with known relations as obtained from observations.
\add{Galaxies show different scaling relations such as the Radial Acceleration Relation, the Mass Discrepancy Acceleration Relation, the dark matter (DM) surface density relation, or the Ferrarese relation connecting the total halo mass and its supermassive central object. These relations are usually explained in terms of $\Lambda$CDM halos including for baryonic feedback. Here we analyze all of them assuming that DM halos are formed through a Maximum Entropy Principle (MEP), and we show they are compatible with a fermionic nature of the DM particles. In particular, the understanding of the above acceleration relations is backed here by comparing the goodness of fit of different DM halo models: the traditional NFW, a generalized NFW accounting for baryonic feedback, the Einasto model, the Burkert model, and the fermionic halos obtained from entropy maximization. For this task, we use a large sample of $120$ galaxies taken from the Spitzer Photometry and Accurate Rotation Curves (SPARC) data-set, from which we infer the DM content to compare with the models. We find that such relations are well explained by all the models with comparable accuracy, but instead the fits to the individual rotation curves show that cored DM halos (like Einasto or the fermionic model) are statistically preferred with respect to the cuspy NFW profile. However, very different physical principles justify the flat inner halo slope in the most favored DM profiles: while generalized NFW or Einasto models rely on complex baryonic feedback processes, the MEP scenario involves a quasi-thermodynamic equilibrium of the DM particles.}
\end{abstract}