%%%%%%%%%%%%%%%%%%%%%%%%%%%%%%%%%%%%%%%%%%%%%%%%%%%%
%%%%%%%%%%%%%%%%%%%%%%%%%%%%%%%%%%%%%%%%%%%%%%%%%%%%
\section{Dark matter models}
\label{sec:dark-matter-models}
%%%%%%%%%%%%%%%%%%%%%%%%%%%%%%%%%%%%%%%%%%%%%%%%%%%%
%%%%%%%%%%%%%%%%%%%%%%%%%%%%%%%%%%%%%%%%%%%%%%%%%%%%

\add{Most of the DM halo models in the literature are phenomenological, i.e. motivated by the phenomenology of rotation curves either from observations or from numerical N-body simulations. In contrast, we consider a fermionic DM model based on first physical principles including statistical-mechanics and thermodynamics.}

%%%%%%%%%%%%%%%%%%%%%%%%%%%%%%%%%%%%%%%%%%%%%%%%%%%%
\subsection{Fermionic DM halos from MEP}
\label{sec:model:rar}
%%%%%%%%%%%%%%%%%%%%%%%%%%%%%%%%%%%%%%%%%%%%%%%%%%%%

It has been proposed by several authors (see, e.g. \citealp{2020EPJP..135..290C} for an exhaustive list of references) that DM halos could be made of fermions (e.g. sterile neutrinos) in gravitational interaction. It is usually assumed that the fermions are in a statistical equilibrium state described by the Fermi-Dirac distribution function. However, the notion of statistical equilibrium for systems with long-range interactions is subtle. If the fermions are non-interacting, apart from gravitational forces, the relaxation time towards statistical equilibrium due to gravitational encounters scales as $(N/\ln N)t_D$ \citep[see e.g.][]{Binney2008} and exceeds the age of the Universe by many orders of magnitude.

For example, assuming a fermion mass $m c^2 \sim \SI{50}{\kilo\eV}$, a DM halo of mass $M\sim \SI{E11}{\Msun}$ and radius $R \sim \SI{30}{\kilo\parsec}$ contains $N\sim \num{E72}$ fermions for a dynamical time $t_D\sim 1/\sqrt{R^3/G M}\sim 100\, {\rm Myrs}$.

Therefore, on the Hubble time, the gas of fermions is essentially collisionless, being described by the Vlasov-Poisson equations. Yet, it can achieve a form of statistical equilibrium on a coarse-grained scale through a process of violent relaxation. This concept was introduced  by \cite{1967MNRAS.136..101L} in the case of collisionless stellar systems and has been exported to DM by \citet{1996ApJ...466L...1K} and \citet{2015PhRvD..92l3527C}. 

Assuming ergodicity (efficient mixing), \citet{1967MNRAS.136..101L} used a MEP and looked for the {\it most probable} equilibrium state consistent with the constraints of the collisionless dynamics. The maximization of the Lynden-Bell entropy $S$ under suitable constraints leads to a coarse-grained distribution function $\bar{f}({\vec r},{\vec v})$ similar to the Fermi-Dirac distribution function. Therefore, the process of violent relaxation may provide a justification of the Fermi-Dirac distribution function for DM halos without the need of efficient gravitational encounters.

However, when coupled to gravity, this distribution function has an infinite mass (i.e., there is no maximum entropy state), implying that either violent relaxation is incomplete or that tidal effects have to be taken into account (if the system is not isolated). The problem therefore becomes an out-of-equilibrium problem and it is necessary to develop a kinetic theory of collisionless relaxation (see e.g. \citealp{2021arXiv211213664C} for a review).

One approach is to use a Maximum Entropy Production Principle (MEPP) and argue that the most probable evolution of the system on the coarse-grained scale is the one that maximizes the rate of Lynden-Bell entropy $\dot S$ under the constraints of the collisionless dynamics \citep{1996ApJ...471..385C}. This leads to a generalized Fokker-Planck equation having the form of a fermionic Kramers equation 
\begin{equation}
    \label{kramers}
     \frac{\partial \bar f}{\partial t} + \vec v \cdot \frac{\partial \bar f}{\partial \vec r} - \nabla\Phi \cdot \frac{\partial \bar f}{\partial \vec v} = \frac{\partial \vec J}{\partial \vec v},
\end{equation}
where $\vec J=D[\partial \bar f/\partial \vec v + (mc^2/kT) \bar f (1-f/\eta_0) \vec v]$ is a diffusion current pushing the system towards statistical equilibrium, $D$ the diffusion coefficient, $T \equiv T(t)$ is the temperature evolving in time so as to conserve the total energy \citep{1998MNRAS.300..981C}, $k$ is the Boltzmann constant, $c$ is the speed of light, and $m$ is the DM fermion mass. However, this approach is heuristic and does not determine the expression of the diffusion coefficient.

An alternative, more systematic, approach is to develop a quasilinear theory of ``gentle'' collisionless relaxation \citep{1970PhRvL..25.1155K,1980Ap&SS..72..293S,1998MNRAS.300..981C,2004PhyA..332...89C} leading to a fermionic Landau equation of the form 
%
\begin{multline}
    \label{landau}
    \frac{\partial \bar f}{\partial t} + \vec v \cdot \frac{\partial \bar f}{\partial \vec r} - \nabla\Phi \cdot \frac{\partial \bar f}{\partial \vec v}
    = \frac{8\pi G^2m^8\epsilon_r^3\epsilon_v^3\ln\Lambda}{h^6}\frac{\partial}{\partial v_i} \int {\rm d}\vec v'\\
    \times \frac{u^2\delta_{ij} - u_i u_j}{u^3}\left\lbrace \bar f' \left (1 - \bar f'\right ) \frac{\partial \bar f }{\partial v_j} - \bar f \left(1-\bar f\right) \frac{\partial \bar f'}{\partial
    v'_j}\right\rbrace,
\end{multline} 
%
where $\bar{f}' \equiv \bar{f}({\vec r},{\vec v}', t)$, $\ln\Lambda=\ln(R/\epsilon_r)$ is the Coulomb logarithm, $R$ is the typical size of the system, $\vec u = \vec v' - \vec v$ is the relative velocity between the ``macro-particles'' of mass $m_{\rm eff}\sim 2m^4 \epsilon_r^3\epsilon_v^3/h^3 \gg m$ (see also below), and $\epsilon_r$, $\epsilon_v$ are the correlation lengths in position and velocity respectively. One can make a connection between the above two kinetic equations by using a form of thermal bath approximation, i.e., by replacing $\bar f'$ in \cref{landau} by its equilibrium (Fermi-Dirac) expression. This substitution transforms an integro-differential (Landau) equation into a differential (Kramers) equation. In this manner one can compute the diffusion coefficient explicitly \citep{1998MNRAS.300..981C}.

The timescale of violent relaxation is a few $10-100$ dynamical times ($t_D$), which is shorter than the Hubble time $t_H$. This is confirmed by the kinetic theory of violent relaxation that predicts a collisionless relaxation time $t_R^{\rm non-coll.}\sim (M/m_{\rm eff})t_D$ which is much shorter than the collisional relaxation time $t_R^{\rm coll.}\sim (M/m)t_D$ because $m_{\rm eff}\gg m$ (see formula in the above paragraph).

Indeed, the relaxation of the coarse-grained DF $\bar{f}({\vec r},{\vec v},t)$ towards the Lynden-Bell distribution (of Fermi-Dirac type, see Eq. \ref{fcDF} below) on a few dynamical times can be interpreted in terms of ``collisions'' between  ``macro-particles'' or ``clumps'' (i.e. correlated regions) with a large effective mass $m_{\rm
eff}$ \citep{1970PhRvL..25.1155K}. These macro-particles considerably accelerate the relaxation of the system (as compared to ordinary gravitational encounters between particles of mass $m$) by increasing the diffusion coefficient $D$ in \cref{kramers}. 

Processes of incomplete relaxation could be taken into account by generalizing the kinetic approach so that the diffusion coefficient rapidly falls off to zero in space and time, thereby leading to a sort of kinetic blocking. Alternatively, if the system is submitted to tidal interactions from neighboring systems one can look for a stationary solution of \cref{kramers} which accounts for the depletion of the distribution function above an escape energy.

For classical systems evolving through two-body gravitational encounters like globular clusters, this procedure leads to the King model \citep{1962AJ.....67..471K}. For fermionic DM halos, one obtains the fermionic King model \citep{1983A&A...119...35R,1998MNRAS.300..981C} 
%
\begin{equation}
    \bar{f}(r,\epsilon\leq\epsilon_c) = \frac{1-e^{[\epsilon-\epsilon_c(r)]/kT(r)}}{e^{[\epsilon-\mu(r)]/kT(r)}+1}, \qquad \bar{f}(r,\epsilon>\epsilon_c)=0\, ,
    \label{fcDF}
\end{equation} 
%
which has been written in the case of general relativistic fermionic systems for the sake of generality \citep{2018PDU....21...82A,2022IJMPD..3130002A}. Here, $\epsilon=\sqrt{p^2c^2 + m^2 c^4} - mc^2$ is the particle kinetic energy, $\mu(r)$ is the chemical potential (with the particle rest-energy subtracted off), $\epsilon_c(r)$ is the escape energy (with the particle rest-energy subtracted off), and $T(r)$ is the effective temperature. The corresponding set of three dimensionless parameters (for fixed $m$) are defined by the temperature, degeneracy and cutoff parameters, $\beta(r)=k T(r)/(m c^2)$, $\theta(r)=\mu(r)/[k T(r)]$ and $W(r)=\epsilon_c(r)/[k T(r)]$, respectively (a subscript $0$ is used when the parameters are evaluated at the center of the configuration).

This distribution function takes into account the Pauli exclusion principle as well as tidal effects, and can lead to a relevant model of fermionic DM halos usually referred as the RAR model, which has been successfully contrasted against galaxy observables \citep{2018PDU....21...82A,2019PDU....24..278A,2020A&A...641A..34B,2021MNRAS.505L..64B,2022MNRAS.511L..35A}.

The full family of density $\rho(r)$ and pressure $P(r)$ profiles within this model can be directly obtained as the corresponding integrals of $\bar{f}(p)$ over momentum space (bounded from above by $\epsilon \leq \epsilon_c(r)$) as detailed in \citet{2018PDU....21...82A}. This leads to a four-parametric fermionic equation of state depending on ($\beta_0,\theta_0,W_0,m$) according to the parameters in \cref{fcDF}. Once with the fermionic distribution function at equilibrium as obtained from the MEP explained above, we make use of the fact that a relaxed system of fermions under self-gravity does admit a perfect fluid approximation \citep{1969PhRv..187.1767R}. Thus, we use the stress-energy tensor of a perfect fluid in a spherically symmetric metric, $g_{\mu\nu}=\diag(\e^{2\nu(r)},-\e^{2\lambda(r)},-r^2,-r^2\sin^2\vartheta)$ with $\nu(r)$, $\lambda(r)$ being the temporal and spatial metric functions and $\vartheta$ the azimutal angle. Such configuration leads to hydrostatic equilibrium equations of self-gravitating fermions. The local $T(r)$, $\mu(r)$ and $W(r)$ fulfill the Tolman, Klein and particle's energy conservation relations, respectively (see \citealp{2018PDU....21...82A} for details). Further, $\nu_0$ is here constrained by the Schwarzschild condition $g_{00}g_{11} = -1$ at the surface where the halo pressure (and density) falls to zero.

An illustration of a core-halo ($\theta_0 > 10$) and a corresponding halo-only ($\theta_0 \ll -1$) solution are shown in \cref{fig:profile-illustration-mep} (we refer to section \ref{sec:morph} and to the cited works above to get a better understanding of the rich morphology of the fermionic DM model).

\loadfigure{figure/ProfileIllustrationsAll}

Mass distributions of that fermionic model are well characterized by the cutoff difference $W(r_p) - W(r_s)$. Since $W(r)$ is defined to be zero at the surface, i.e. $W(r_s) = 0$ with $r_s$ being the surface radius where the density drops to zero, we need to focus only on the plateau cutoff $W_p = W(r_p)$ with $r_p$ being the plateau radius defined at the first minimum in the rotation curve. 

Other important quantities of the \textit{core-halo} family of fermionic DM mass solutions are the core mass $M_c = M(r_c)$ with $r_c$ being the core radius defined at the first maximum in the rotation curve, the halo mass $M_h = M(r_h)$ with $r_h$ being the halo radius defined at the second maximum in the rotation curve, and the total mass $M_s = M(r_s)$ given at the surface radius $r_s$.

The density profiles in the fermionic model can develop a rich morphological behaviour: while the halo region is King-like (i.e. from polytropic-like with $W_p \ll 1$ to power law-like with $W_p \gtrsim 10$, see \ref{sec:morph}), the inner region can either develop a \textit{dense core} at the center of such a halo (i.e. for large central degeneracy $\theta_0 > 10$), or not (i.e. $\theta_0 \ll -1$ in the dilute regime). Both kind of family profiles are thermodynamically and dynamically stable as well as long lived in a cosmological framework, as recently demonstrated in \citet{2021MNRAS.502.4227A} for typical galaxies with total masses of the order $\sim \SI{5E10}{\Msun}$.

Remarkably, for \textit{core} - \textit{halo} RAR solutions with fermion masses of $m c^2\approx \SI{50}{\kilo\eV}$, the degenerate and compact DM cores may work as an alternative to the BH paradigm at the center of non-active galaxies \citep{2018PDU....21...82A,2019PDU....24..278A,2020A&A...641A..34B,2021MNRAS.505L..64B,2022MNRAS.511L..35A}. Furthermore, their eventual gravitational core-collapse in larger galaxies may offer a novel supermassive BH formation mechanism from DM \citep{2021MNRAS.502.4227A}.

%%%%%%%%%%%%%%%%%%%%%%%%%%%%%%%%%%%%%%%%%%%%%%%%%%%%
\subsection{Other DM halo models}
\label{sec:model:dc14-nfw}
%%%%%%%%%%%%%%%%%%%%%%%%%%%%%%%%%%%%%%%%%%%%%%%%%%%%

From a phenomenological viewpoint, a DM halo density profile is described by three characteristics: the inner halo, the outer halo and the transition in between. Such density profiles are usually described by the ($\alpha, \beta, \gamma$)-model \citep{1990ApJ...356..359H} 
%
\begin{equation}
    \label{eqn:hernquist}
	\frac{\rho(r)}{\rho_\mathrm{N}} = \qbracket{\frac{r}{R_\mathrm{N}}}^{-\gamma}\qbracket{1 + \qbracket{\frac{r}{R_\mathrm{N}}}^\alpha}^{-\frac{\beta - \gamma}{\alpha}},
\end{equation} 
%
where $\alpha$ describes the transition, $\beta$ the outer slope and $\gamma$ the inner slope. Following Newtonian dynamics --- that is fully sufficient on halo scales --- then the velocity is given by \begin{equation}
    \label{eqn:circular-velocity}
    \frac{v^2(r)}{\sigma_\mathrm{N}^2} = \frac{R_{\rm N}}{r}\frac{M(r)}{M_{\rm N}}
\end{equation} and the enclosed mass by \begin{equation}
    \label{eqn:enclosed-mass}
     \frac{M(r)}{M_{\rm N}} = \int \limits_0^r \qbracket{\frac{r}{R_{\rm N}}}^2 \frac{\rho(r)}{\rho_\mathrm{N}} \frac{\d r}{R_{\rm N}},
\end{equation} For the following DM halo models we will use $R_\mathrm{N}$, $\rho_\mathrm{N}$, $\sigma_\mathrm{N}^2 = G M_\mathrm{N}/R_\mathrm{N}$ and $M_\mathrm{N} = 4\pi \rho_\mathrm{N} R_\mathrm{N}^3$ as scaling factors for length, density, velocity and mass, respectively.

In \cref{fig:profile-illustration-mep} we illustrate the typical morphology of common DM halo models used here for the SPARC galaxies. For a better comparison the plots are normalized with respect to the halo located at the velocity maximum on halo scales.

%% -- DC14 --
Based on the general ($\alpha, \beta, \gamma$)-model, \citet{2014MNRAS.441.2986D} modelled CDM halos including baryonic feedback mechanisms in galaxy formation. They found that the three parameters ($\alpha, \beta, \gamma$) are related through 
\begin{align}
    \label{eqn:dc14:alpha}
	\alpha &= 2.94 - \log_{10}\qbracket{(10^{X + 2.33})^{-1.08} + (10^{X+2.33})^{2.29}},\\
	\label{eqn:dc14:beta}
    \beta &= 4.23 + 1.34 X + 0.26 X^2,\\
    \label{eqn:dc14:gamma}
    \gamma &= -0.06 + \log_{10}\qbracket{(10^{X + 2.56})^{-0.68} + 10^{X+2.56}},
\end{align} 
%
where $X = \log_{10}(M_*/M_\mathrm{halo})$ describes the stellar-to-dark matter ratio. For the circular velocity \cref{eqn:circular-velocity} the enclosed mass \cref{eqn:enclosed-mass} is given by a hypergeometric function
%
\begin{equation}
	\frac{M(r)}{M_\mathrm{N}} = \frac{1}{3-\gamma} \qbracket{\frac{r}{R_\mathrm{N}}}^{3-\gamma} \pFq{2}{1}(p_1,p_2;\,q_1;\,-[r/R_\mathrm{N}]^\alpha),
\end{equation} 
%
with $p_1 = (3-\gamma)/\alpha$, $p_2 = (\beta-\gamma)/\alpha$ and $q_1 = 1 + (3 - \gamma)/\alpha$. In the following, we refer this model as DC14.

%% -- NFW --
Alternatively, for $\alpha = 1$, $\beta = 3$ and $\gamma = 1$ the ($\alpha, \beta, \gamma$)-model reduces to the NFW model \citep{1996ApJ...462..563N,1997ApJ...490..493N} as obtained from early DM-only N-body simulations. This DM model develops cuspy halos of the following type
%
\begin{equation}
	\frac{\rho(r)}{\rho_\mathrm{N}} = \qbracket{\frac{r}{R_\mathrm{N}}}^{-1}\qbracket{1 + \frac{r}{R_\mathrm{N}}}^{-2},
\end{equation} 
%
with the circular velocity \begin{equation}
	\frac{v^2(r)}{\sigma_\mathrm{N}^2} = \frac{\ln(1 + r/R_\mathrm{N})}{r/R_\mathrm{N}} - \frac{1}{1 + r/R_\mathrm{N}}.
\end{equation} 

%% -- Burkert --
In contrast to NFW, \citet{1995ApJ...447L..25B} proposed a DM density profile with a cored halo of the following type
%
\begin{equation}
    \frac{\rho(r)}{\rho_\mathrm{N}} = \qbracket{1 + \frac{r}{R_\mathrm{N}}}^{-1}\qbracket{1 + \bracket{\frac{r}{R_\mathrm{N}}}^{2}}^{-1}.
\end{equation} 
%
For the circular velocity \cref{eqn:circular-velocity} the enclosed mass \cref{eqn:enclosed-mass} is given by
%
\begin{multline}
     \frac{M(r)}{M_{\rm N}} = \frac14 \ln(1 + [r/R_{\rm N}]^2) + \frac12 \ln(1 + r/R_{\rm N})\\
     - \frac12 \arctan(r/R_{\rm N}).
\end{multline} 
%
With $M_0 \approx M(R_N)$ being the mass scale originally interpreted as the core mass of the halo \citep[e.g.][]{2000ApJ...537L...9S} we obtain the relation $M_N = 8 M_0$. Further, the density scale $\rho_N$ describes the central density $\rho_0$ and the length scale $R_{\rm N}$ can be identified with the Burkert radius $r_{\rm B}$ fulfilling the condition $\rho(r_{\rm B}) = \rho_0/4$.

%% -- Einasto --
Another interesting and successful candidate is the Einasto model \citep{1989A&A...223...89E}, a purely empirical fitting function with no commonly recognized physical basis \citep{2006AJ....132.2685M}. The DM halo density profiles of that model are of the following type, given in a normalized form,
%
\begin{equation}
    \frac{\rho(r)}{\rho_\mathrm{N}} = \e^{-[r/R_{\rm N}]^{\kappa}}.
\end{equation} 
%
The exponent $\kappa$ describes the shape of the density profile. The circular velocity and the enclosed mass are given by \cref{eqn:circular-velocity,eqn:enclosed-mass}. This model develops mass distributions with a finite mass $M_{\rm tot}/M_{\rm N} = \Gamma(3/\kappa)/\kappa$ for $r\to\infty$ (see also \citealp{2012A&A...540A..70R}). The typical $\kappa$ values obtained in this work for the SPARC data-set, as well as the comparison with the same values coming from N-body simulations (either with or without baryonic effects), are given in subsections \ref{boundaryC}, \ref{sec:baryonic-effect} and \ref{sec:result:gof}, and Fig. \ref{fig:parameter-distribution:einasto}.

%%%%%%%%%%%%%%%%%%%%%%%%%%%%%%%%%%%%%%%%%%%%%%%%%%%%
\subsection{Fitting priors and Monte-Carlo approach}
\label{boundaryC}
%%%%%%%%%%%%%%%%%%%%%%%%%%%%%%%%%%%%%%%%%%%%%%%%%%%%

For the fermionic DM model we fix the particle mass $m$ and therefore reduce the number of free parameters by one, e.g. $\vec p = (\beta_0, \theta_0, W_0)$. A particle mass of $mc^2 = \SI{50}{\kilo\eV}$ is well motivated by the promising results obtained in \citet{2018PDU....21...82A,2020A&A...641A..34B,2021MNRAS.505L..64B,2022MNRAS.511L..35A}, where the fermionic core-halo DM profile was able to explain both the S-stars orbits around SgrA*, and the Milky Way rotation curve. In \citet{2019PDU....24..278A,2021MNRAS.502.4227A}, in particular, and for the same particle mass, the fermionic core-halo profiles were successfully applied to other galaxy types from dwarf to larger galaxy types, providing a possible explanation for the nature of the intermediate mass BHs, as well as a possible mechanism for SMBH formation in active galaxies. Moreover, as demonstrated in \citet{2018PDU....21...82A} there exists a particle mass range between $\sim$ 50 and $\sim$ 350 keV where the compacity of the central core can increase (all the way to its critical value of collapse) while maintaining the same DM halo-shape. Therefore, regarding the SPARC RC fitting, as well as all the scaling relations on halos-scales, our conclusions are not biased by the choice of the particle mass in the above range.

For the fermionic model we consider solutions which are either in the dilute regime ($\theta_0 \ll -1$, i.e. are King-like), or which have developed a degenerate core (i.e. $\theta_0 > 10$) at the center of such a halo. Fermionic solutions within only these two families have been shown to be thermodynamically and dynamically stable when applied to galaxies \citep{2021MNRAS.502.4227A}.

The NFW and the Burkert models are described by two free scaling parameters, e.g. $\vec p = (R_{\rm N},\rho_{\rm N})$. The DC14 model with e.g. $\vec p = (X, R_{\rm N}, \rho_{\rm N})$ and the Einasto model with e.g. $\vec p = (\alpha, R_{\rm N}, \rho_{\rm N})$ are described by three free parameters. Compared to NFW and Burkert, both (Einasto and DC14) have an additional parameter which affects the sharpness of the transition from the inner to the outer halo.

In order to find the best-fits we use the LM algorithm (see section \ref{LM-fitting}) with well chosen initial values (i.e. priors) reflecting astrophysical (realistic) scenarios. Because the LM algorithm finds only local minima, we choose the parameter sets randomly within a range and follow a Monte-Carlo approach. For the fermionic \textit{core}-\textit{halo} solutions, we choose $\beta_0 \in [10^{-8},10^{-5}]$, $\theta_0\in [25,45]$ and $W_0\in[40,200]$ which correspond to a conservatively wide range of parameters according to \citet{2019PDU....24..278A}. For the fermionic \textit{diluted} solutions, we cover the same range for $\beta_0$ and $W_0$, but with $\theta_0\equiv\theta_p \in [-40,-20]$. For the other DM models, the initial scaling factors are chosen from $R_{\rm N}\in[10^{1}, 10^{4}]\si{\parsec}$ and $\rho_{\rm N} \in [10^{-4}, 10^{-1}]\si{\Msun\per\parsec^3}$. According to \citet{2017MNRAS.466.1648K} the additional parameter of the DC14 model is chosen from $X\in[-3.75,-1.3]$. For the Einasto model, the exponent is chosen from $\alpha\in[0.1, 10]$. This relatively large window has been chosen in order (i) to account for the broad diversity of rotation curves covered by the SPARC data, and (ii) not to be limited by any fixed value (e.g. as typically obtained by CDM-only simulations) since this is an independent analysis to that of N-body simulations, and may also account for other effects such as baryonic feedback (see also the discussion in next subsection).

%%%%%%%%%%%%%%%%%%%%%%%%%%%%%%%%%%%%%%%%%%%%%%%%%%%%
\subsection{\add{Baryonic feedback}}
\label{sec:baryonic-effect}
%%%%%%%%%%%%%%%%%%%%%%%%%%%%%%%%%%%%%%%%%%%%%%%%%%%%
%[This section has to be updated with the more appropriate ref. to $\alpha$ Einasto parameters from [Bozek+ 2019] either in WDM-only simula-Hydro ones. ANDREAS: I would say we can move this whole subsection to section 4, because it is mainly about results and contains some discussion.]

\add{While some of the models here considered (effectively) account for baryonic feedback effects such as the DC14 model \citep{2014MNRAS.439..300L} and the Einasto profile \citep{2019MNRAS.483.4086B}, the others do not. However it is important to recall that the baryonic feedback is DM-model dependent, and, for example, in WDM cosmologies such effects are typically diminished with respect to CDM \citep{2019MNRAS.483.4086B}. The main reasons behind this attenuation are that in WDM cosmologies DM halos form later, are less centrally dense on inner-halo scales, and therefore contain galaxies that are less massive than the CDM counterparts. DM halos within the MEP-formation scenario share this properties because they belong to WDM cosmologies (the fermion masses are in the keV range), and are certainly less dense than in the CDM case since they develope a plateau on the inner-halo \citep{2021MNRAS.502.4227A}. Thus baryonic feedback effects in the WDM models here studied are expected to be milder than in CDM ones.}
    
\add{Even when some baryonic effects are present as to cause resulting DM profiles with somewhat lower inner-halo densities (as for DC14 and Einasto), these effects can be thought to be indirectly (or effectively) accounted in our statistical analysis since we perform parametric fits. That is, for DM profiles whose universal shape is expected to account already for some baryonic feedback (such as Einasto or DC14 mentioned above), if such effects are considerable for SPARC galaxies, they should be reflected in the best-fit parameters. In section \ref{sec:result:gof} (ii) we compare our statistical results regarding baryonic effects in the DM model free-parameters, with the ones reported in the literature.} 

%Interestingly \cite{2019MNRAS.483.4086B} reports, in DM-only simulations, a reduction in WDM inner-halo density relative to CDM ($\rho^{Ein}_{\rm WDM}/\rho^{Ein}_{\rm CDM}(r_{in})$) of about $0.7$, which is consistent with an Einasto concentration parameter in the WDM case of about $10$ \citep{2019MNRAS.483.4086B. Even when no explicit reference to the $\alpha$ parameter is there given for such a concentration, it must be somewhat larger than $\alpha=0.2$ in order to cause the above density-reduction (recall $\alpha=0.2$ is a best fit parameter to standard CDM halos: see e.g. fig. 2 in \citealp{2014MNRAS.441.3359D}). Our results are then consistent with these values for the Einasto case (see \cref{fig:parameter-distribution:einasto})