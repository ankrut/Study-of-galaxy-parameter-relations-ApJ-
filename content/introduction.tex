%%%%%%%%%%%%%%%%%%%%%%%%%%%%%%%%%%%%%%%%%%%%%%%%%%%%
%%%%%%%%%%%%%%%%%%%%%%%%%%%%%%%%%%%%%%%%%%%%%%%%%%%%
\section{Introduction}
%%%%%%%%%%%%%%%%%%%%%%%%%%%%%%%%%%%%%%%%%%%%%%%%%%%%
%%%%%%%%%%%%%%%%%%%%%%%%%%%%%%%%%%%%%%%%%%%%%%%%%%%%

How the total gravitating mass 
%is distributed 
distributes with respect to the luminous mass on galaxy scales is an 
%intriguing 
open question which has regained much attention in the last decade thanks to the vast data-sets covering 
%always 
broader radial extents across different Hubble types \citep{2008AJ....136.2648D,2011MNRAS.413..813C,2016AJ....152..157L}.
Several universal relations exist between different pairs of structural galaxy parameters, which refer either \begin{inparaenum}[(i)]
    \item to the outer regions of galaxies such as the baryonic Tully-Fisher relation (\add{BTFR}) \citep{2000ApJ...533L..99M}, the DM surface density relation \citep{2009MNRAS.397.1169D}, the Radial Acceleration Relation \citep{2016PhRvL.117t1101M}, \add{and the Mass Discrepancy Acceleration Relation (MDAR) \citep{2004ApJ...609..652M}}, \add{which are indeed all closely related \citep{2004ApJ...609..652M,2016arXiv161208857S,2018FoPh...48.1517S}}; 
    \item to their central regions such as the $M$-$\sigma$ relation between the bulge's dispersion velocity and the central object mass \citep{2000ApJ...539L...9F}; or
    \item to a combination of both regimes such as the Ferrarese relation \citep{2002ApJ...578...90F,2011Natur.469..377K,2015ApJ...800..124B} between the total halo mass and its supermassive central object mass.
\end{inparaenum}

In this paper, we focus 
%our attention 
on the \add{above two} Acceleration Relations, 
%but we will
and use other universal relations discussed above to constrain the underlying dark matter (DM) models. \add{In particular} the Radial Acceleration Relation connects the centripetal accelerations of the baryonic and total matter components: it shows a non-linear correlation between the radial acceleration caused by the total matter and the one generated by its baryonic component only. 
%Importantly, 
This relation is not limited to disk galaxies 
%and 
but also holds for other galaxy types (e.g. ellipticals, lenticulars, dwarfs spheroidals and even low-surface-brightness galaxies), what makes it a true universal law among morphology classification \citep{2016PhRvL.117t1101M,2017ApJ...836..152L,2019ApJ...873..106D}. Despite a relatively large scatter, this relation on halo scales seems to imply a fundamental acceleration scale $\mathfrak{a}_0$, which is present in the non-linear fitting function as obtained in \cite{2016PhRvL.117t1101M}, and given by
%
\begin{equation}
	\label{eq:mcgaugh-fit}
	\SYMatot = \frac{\SYMabar}{1 - \e^{-\sqrt{\SYMabar/{\SYMafrak}}}},
\end{equation} 
%
where $\SYMafrak$ is the only adjustable parameter. In the low acceleration regime ($\SYMabar \ll \SYMafrak$), where DM dominates, it clearly shows a deviation from a linear correlation (see top panels of \cref{fig:acceleration:grid}). While in the high acceleration regime ($\SYMabar \gg \SYMafrak$), dominated by baryonic matter, the linear relation is recovered.

% different explanations
An acceleration scale of this kind naturally arises in Modified Newtonian Dynamics (MOND) \citep{2015CaJPh..93..169K,2016arXiv160906642M,2016PhRvL.117t1101M,2018A&A...615A...3L},
%. Thus, allowing them 
which has been used to interpret such a fundamental scale as evidence against the $\Lambda$CDM paradigm and in favor to the MOND theory. However, more recent studies dedicated to analyze this universal relation within the (Bayesian) posterior distributions on the acceleration scales of individual galaxies (across a large sample), have provided evidence against the existence of such a fundamental constant and in favour of $\mathfrak{a}_0$ to be an emergent magnitude (see e.g. \citealp{2020MNRAS.494.2875M} and references therein). On the other hand, it has been extensively shown that the Radial Acceleration Relation is consistent with the $\Lambda$CDM paradigm, as found in hydrodynamical N-body simulations  \citep{2016MNRAS.456L.127D,2017MNRAS.471.1841N,2017PhRvL.118p1103L,2019MNRAS.485.1886D}. This conclusion is in line with a more phenomenological (independent) study based on Universal Rotation Curves including for baryonic mass models \citep{2018FoPh...48.1517S}.

%Also hydrodynamical simulations are able to reproduce the radial acceleration correlation \citep{2016MNRAS.455..476S,2016arXiv161006183K,2016arXiv161007663L,2017arXiv170305287T}. [INCLUDE Salucci's excplanation about this].
%Accordingly, universal relations involving different pairs of structural galaxy parameters have been unveiled for galactic structures.
% from center to periphery,
% dark-to-dark relations
%Focusing on dark components only, \citet{2002ApJ...578...90F} found a link between the total DM mass of a galaxy and the mass of its (massive) dark compact object in the galactic center. This relation covers many order of magnitudes in total DM mass, from $\sim \SI{E11}{\Msun}$ (spirals) to $\sim \SI{E14}{\Msun}$ (big ellipticals). Additionally, DM halos show a nearly constant surface density $\rho_0 r_h \approx 140^{+80}_{-30}\,\si{\Msun\parsec^{-2}}$, where $\rho_0$ is the central density of the halo core and $r_h$ the one-halo-length-scale of the Burkert profile. This universal halo surface density law is valid over a range of 14 mag in luminosity and for all Hubble types \citep{2009MNRAS.397.1169D}.

% dark-to-baryonic relations (core)
%Interestingly, those dark component relations seem to be independent of the obviously existing baryonic matter. But dark and baryonic matter affect their dynamics gravitationally, clearly, what implies possible further relations. Indeed, since the discovery of the $M-\sigma$ relation \citep{2000ApJ...539L...9F}, supermassive compact objects (e.g. BHs) were considered as main components in galaxy formation and evolution.

% dark-to-baryonic relations (halo)
%Focusing on the outer part of a galaxy, \citet{1977A&A....54..661T} demonstrated an empirical relationship between the stellar mass (or luminosity) and the maximal halo rotation velocity. Later, it was shown that this Tully-Fisher relation holds even tighter when the gas component is added. Known as baryonic Tully-Fisher relation, it connects the total baryonic mass (stellar and gas) with the maximal halo rotation velocity \citep{2000ApJ...533L..99M}. Important to emphasize is that the halo rotation velocity appears typically as a nearly flat tail in the DM dominated regime. Historically, that flat rotation curve is the classical argument for dark matter because baryonic matter inferred from light implies a Keplerian behavior what is in conflict with observations \citep{1980ApJ...238..471R}. Thus, it is possible to connect the (baryonic) Tully-Fisher relation with dark matter.

In this work, we present an independent approach of the two discussed above, where DM halos are formed through a Maximum Entropy Principle (MEP) for collisionless self-gravitating systems \citep{1998MNRAS.300..981C,2015PhRvD..92l3527C,2020EPJP..135..290C,2021MNRAS.502.4227A}. \add{It is based on the Fermi-Dirac (or Lynden-Bell) entropy functional}, in which the fermionic (quantum) nature of the DM particles is accounted for in the phenomenology of real galaxies. It was recently demonstrated that fermionic halos obtained via this mechanism can arise in a cosmological framework, and remain thermodynamically and dynamically stable during the life of the Universe \citep{2021MNRAS.502.4227A}. Indeed, it was there shown the self-consistency of the approach, in the sense that the nature and mass of the DM particles involved in the linear matter power spectrum --- as calculated in \citet{2021MNRAS.502.4227A} within a CLASS code for $\mathcal{O}(10)$ keV fermions --- are the very same building blocks at the basis of the virialized DM configurations with its inherent effects in the DM profiles. 

The main emergent features of such DM fermionic profiles, which differentiate them with respect to the $\Lambda$CDM N-body simulation ones, are: 
%
\begin{asparaenum}[(i)]
    \item DM fermions with a finite-temperature can be in a diluted (Boltzmannian-like) regime or become semi-degenerate. The corresponding DM halos can be, respectively, King-like or may develop a dense and degenerate compact core at the center of such a halo \citep{2015PhRvD..92l3527C,2021MNRAS.502.4227A}. A fully relativistic model in which this more general \textit{core}--\textit{halo} profiles arise, is usually referred in the literature as the Ruffini-Argüelles-Rueda (RAR) model \citep{2015MNRAS.451..622R,2018PDU....21...82A,2019PDU....24..278A,2020A&A...641A..34B,2021MNRAS.505L..64B,2021MNRAS.502.4227A,2022IJMPD..3130002A,2022MNRAS.511L..35A}. In either case, these fermionic DM profiles are cored (i.e. develop an extended plateau on halo scales similar to Burkert), thereby not suffering from the core-cusp problem associated with the standard $\Lambda$CDM cosmology \citep{2017ARA&A..55..343B}. This cored feature seems to be a general conclusion reached for any DM profile which has reached a (quasi) thermodynamic equilibrium in cosmology \citep{2021MNRAS.504.2832S}.

    \item \add{The relevance of the fermionic} solutions with a degenerate DM core surrounded by a diluted halo (named from now on as \textit{core}--\textit{halo} profiles) \add{imply different consequences: (a)} the core might become so densely packed that above a threshold --- the critical mass --- the quantum pressure can not support it any longer against its own weight, leading to the gravitational core-collapse into a supermassive black hole (SMBH;  \citealp{2020EPJB...93..208A,2021MNRAS.502.4227A}). For DM particle masses of $\mathcal{O}(10)$ keV, this result provides, for large enough galaxies, a novel SMBH formation mechanism in the early Universe as proposed in \citet{2021MNRAS.502.4227A}; \add{and (b) for core masses below its critical value, it exist a set of free parameters in the fermionic model, such that the quantum cores correlate with their outer halos explaining the $M_{\rm BH} - M_{\rm tot}$ relation in large enough galaxies \citep{2019PDU....24..278A}. However this result may not hold for smaller halos as shown for the SPARC data-set in \ref{sec:parameter-corelation:ferrarese}}.

    \item \add{For small enough galaxies as in the case of typical dwarfs}, the degenerate core cannot collapse towards a BH \add{(i.e. the total mass of the galaxy is below the critical mass of collapse)} and thus it remains in a \textit{core}--\textit{halo} state where the central nucleus still mimics the effects of a singularity \add{(with core masses in the range of Intermediate mass BHs)} while the outer halo explains the rotation curves (RCs) \citep{2019PDU....24..278A,2021MNRAS.502.4227A,2022IJMPD..3130002A}.
    
    \item In the more extreme case where the fermions are fully-degenerate (i.e. when they are treated under the $T \to 0$ approximation), the corresponding halos are polytropic and may be only applicable to dwarfs \citep{2015JCAP...01..002D}.
\end{asparaenum}

In this work, we will consider a filtered SPARC sample of $2369$ data points (corresponding to a total of $120$ galaxies) and apply a non-linear least square statistical analysis in order to check the goodness of fit of: 
\begin{inparaenum}[(a)]
    \item the above fermionic DM halos --- either in the \textit{core}--\textit{halo} regime \citep{2019PDU....24..278A,2021MNRAS.502.4227A} or in the purely King-like one \citep{2021MNRAS.502.4227A};
    \item a baryonic feedback motivated halo model within $\Lambda$CDM according to \citet{2014MNRAS.441.2986D}, named here as DC14;
    \item the classical NFW model based on early numerical simulations \citep{1997ApJ...490..493N};
    \item the Einasto profile \citep{1989A&A...223...89E,2006AJ....132.2685M}, and
    \item the Burkert model \citep{1995ApJ...447L..25B}.
\end{inparaenum} 
%
The task is to compare and contrast the different models against the universal relations and individual RCs, as well as to see the physical insight that can be gained by inspecting the inner DM morphology of the statistically preferred DM profiles.



% mass discrepancy acceleration relation
%With this idea some authors focus on the mass discrepancy encoded in the magnitude $D = \SYMvobs^2/\SYMvbar^2$, where $\SYMvobs$ is the total observed velocity and $\SYMvbar$ is the inferred velocity of the baryonic component only \citep{2004ApJ...609..652M,2014Galax...2..601M}. They found that for disk galaxies the mass discrepancy shows clearly a systematic increasing ($D > 1$) with decreasing centripetal acceleration of the baryonic component, $\SYMabar = \SYMvbar^2/r$, below a particular scale $\SYMafrak \approx \SI{1.2E-10}{\metre/\second^2}$. Above $\SYMafrak$ dark matter becomes negligible with $D\approx 1$ where the Keplerian law is recovered. This correlation is known as the mass discrepancy acceleration relation (MDAR).

% radial acceleration relation
%An alternative representation is the so-called radial acceleration correlation (RAC)\footnote{
%    Also known as the radial acceleration relation (RAR) not to be confused with the dark matter model named after Ruffini-Argüelles-Rueda (RAR).
%} what connects the centripetal accelerations of the baryonic and total component. It turned out that this relation, equivalent to MDAR, is independent of the Hubble type. Thus, the relation is not limited to disk galaxies and holds also for other galaxy types (e.g. ellipticals, lenticulars, dwarfs spheroidals) what makes it a true universal law among morphology classification \citep{2017ApJ...836..152L}. Despite a relatively large scatter, this relation --- independent of the representation --- implies a fundamental link between dark and baryonic matter on halo scales that corresponds to the low acceleration regime.

% new approach
%We face the above problem from a different and complementary perspective relying on semi-analytic approaches which are based on self-gravitating systems of elementary fermions \citep{2019PDU....24..278A,2018PDU....21...82A,2015MNRAS.451..622R,2015ARep...59..656S,2014JKPS...65..809A,2014IJMPD..2342020A,2014JKPS...65..801A}. That recently proposed new model (hereafter RAR model) produces novel dark matter distributions with a compact and degenerate core in the galactic center embedded in a finite halo. In other words: a \textit{dense quantum core - classical halo} distribution (or simply \textit{core-halo} distribution hereafter). For a given particle mass range of few $10$ -- $\SI{100}{\kilo\eV}$ these continuous core-halo profiles describe halo observables and explain simultaneously the compact core, existing in dwarfs to ellipticals, as an alternative to supermassive black holes \citep{2015MNRAS.451..622R,2018PDU....21...82A,2019PDU....24..278A}. Thus, the RAR model can naturally explain the universal relations involving dark components only, emphasizing the importance of this kind of first principles approaches including quantum statistics, self-gravity and thermodynamics.

% purpose of paper
%It is the purpose of this paper to extend the applicability of the RAR model to universal relations which include both dark and baryonic structural galaxy parameters. We center our attention here in the radial acceleration correlation \citep{2016arXiv160905917M}, broadening the extent of applicability of the RAR model from those including only DM as analyzed before. We thus consider a filtered SPARC sample of 2369 data points and apply a non-linear least square statistical analysis in order to check the goodness of fit for the RAR model, making also the same kind of analysis for the original RAR model (i.e. without the effect of escaping particles), a baryonic feedback motivated model called DC14 and the classical NFW model in order to properly compare among them. The task is then not only to compare between the different approaches but to see what more information can be gained by inspecting the behavior in the (free) physical parameter space of the RAR model depending on some galaxy characteristics.

% outline
Outlining the structure of this paper, in section \ref{sec:fitting} we give a brief overview about the SPARC data set, \add{data selection} and fitting procedure. In section \ref{sec:dark-matter-models}, we describe the competing DM halo models considered above. In section \ref{sec:results}, we present our results on the Radial Acceleration Relation by performing a goodness of the fit for each DM model based on a filtered SPARC sample. Finally, in section \ref{sec:conclusion}, we give a brief summary and draw the conclusions.

We refer to the appendix for further details. In Appendix \ref{sec:appendix:parameter-distribution}, we provide model parameter distributions of the competing DM models. In Appendix \ref{sec:parameter-correlations}, we focus on the above mentioned fermionic model and analyze its consistency with different Universal relations such as the Ferrarese relation, comparing the total halo mass with its massive central object, and the DM surface density relation.
