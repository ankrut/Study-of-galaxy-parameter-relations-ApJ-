\section{Introduction}
The intricate relation between the mass distribution of dark and baryonic (stars, gas) matter on galactic scales, from center to periphery, is an intriguing open question which has gained much attention in the last decade thanks to the vast datasets covering broader radial extents and including for different galaxy types \citep{2008AJ....136.2648D,2016AJ....152..157L,2011MNRAS.413..813C}. Big efforts have been aiming to understand those dark-to-baryonic relations within completely independent approaches. Accordingly, universal relations involving different pairs of structural galaxy parameters have been unveiled for galactic structures.

% dark-to-dark relations
Focusing on dark components only, \citet{2002ApJ...578...90F} found a link between the total DM mass of a galaxy and the mass of its (massive) dark compact object in the galactic center. This relation covers many order of magnitudes in total DM mass, from $\sim 10^{11}$ (spirals) to $\sim 10^{14}$ (big ellipticals). Additionally, DM halos show a nearly constant surface density $\rho_0 r_h \approx 140^{+80}_{-30}\,\si{\Msun\parsec^{-2}}$, where $\rho_0$ is the central density of the halo core and $r_h$ the one-halo-length-scale of the Burkert profile. This universal halo surface density law is valid over a range of 14 mag in luminosity and for all Hubble types \citep{2009MNRAS.397.1169D}.

% dark-to-baryonic relations (core)
Interestingly, those dark component relations seem to be independent of the obviously existing baryonic matter. But dark and baryonic matter affect their dynamics gravitationally, clearly, what implies possible further relations. Indeed, since the discovery of the $M-\sigma$ relation \citep{2000ApJ...539L...9F}, supermassive compact objects (e.g. BHs) were considering as main components in galaxy formation and evolution.

% dark-to-baryonic relations (halo)
Focusing on the outer part of a galaxy, \citet{1977A&A....54..661T} demonstrated an empirical relationship between the stellar mass (or luminosity) and the maximal rotation velocity. Later, it was shown that this Tully-Fisher relation holds even tighter when the gas component is added. Known as baryonic Tully-Fisher relation, it connects the total baryonic mass (stellar and gas) with the maximal rotation velocity \citep{2000ApJ...533L..99M}. Important to emphasize is that the maximal rotation velocity appears typically as a flat tail in the DM dominated regime. Historically, that flat rotation curve is the classical argument for dark matter because baryonic matter inferred from light implies a Keplerian behavior what is in conflict with observations \citep{1980ApJ...238..471R}. Thus, it is possible to connect the (baryonic) Tully-Fisher relation with dark matter.

% mass disrepancy acceleration relation
With this idea some authors focus on the mass discrepancy encoded in the magnitude $D = \SYMvobs^2/\SYMvbar^2$, where $\SYMvobs$ is the total observed velocity and $\SYMvbar$ is the inferred velocity of the baryonic component only \citep{2004ApJ...609..652M,2014Galax...2..601M}. They found that for disk galaxies the mass discrepancy shows clearly a systematic increasing ($D > 1$) with decreasing centripetal acceleration of the baryonic component, $\SYMabar = \SYMvbar^2/r$, below a particular scale $\SYMafrak \approx \SI{1.2E-10}{\metre/\second^2}$. Above $\SYMafrak$ dark matter becomes negligible with $D\approx 1$ where the Keplerian law is recovered. This correlation is known as the mass discrepancy acceleration relation (MDAR).

% radial acceleration relation
An alternative representation is the so-called radial acceleration correlation (RAC) what connects the centripetal accelerations of the baryonic and total component. It turned out that this relation, equivalent to MDAR, is independent of the Hubble type. Thus, the relation is not limited to disk galaxies and holds also for other galaxy types (e.g. ellipticals, lenticulars, dwarfs spheroidals) what makes it a true universal law among morphology classification \citep{2017ApJ...836..152L}. Despite a relatively large scatter, this relation - independent of the representation - implies a fundamental link between dark and baryonic matter on halo scales what corresponds to the low acceleration regime.

% various explanations
The link arises naturally in Modified Newtonian Dynamics (MOND) and other modified gravity theories \citep{2015CaJPh..93..169K,2016arXiv160905917M,2017arXiv170204355L} but may be explained within the $\Lambda$CDM context as well \citep{2016MNRAS.456L.127D,2016arXiv161206329N,2016arXiv161208857S}. Also hydrodynamical simulations are able to reproduce the radial acceleration correlation \citep{2016MNRAS.455..476S,2016arXiv161006183K,2016arXiv161007663L,2017arXiv170305287T}.

% new approach
We face the above problem from a different and complementary perspective relying on semi-analytic approaches which are based on self-gravitating systems of elementary fermions \citep{arguelles_novel_2018,2015MNRAS.451..622R,2015ARep...59..656S,2014JKPS...65..809A,2014IJMPD..2342020A,2014JKPS...65..801A}. That recently proposed new model (hereafter RAR model) produces novel dark matter distributions with a compact and degenerate core in the galactic center embedded in a finite halo. In other words: a \textit{dense quantum core - classical halo} distribution (or simply \textit{core-halo} distribution hereafter). For a given particle mass range of few $10$ -- $\SI{100}{\kilo\eV}$ these continuous core-halo profiles describe halo observables and explain simultaneously the compact core, existing in dwarfs to ellipticals, as an alternative to supermassive black holes \citep{2015MNRAS.451..622R,arguelles_novel_2018}. Thus, the RAR model can naturally explain the universal relations involving dark components only, emphasizing the importance of this kind of first principles approaches including quantum statistics, self-gravity and thermodynamics.

% purpose of paper
It is the purpose of this paper to extend the applicability of the RAR model to universal relations which include both dark and baryonic structural galaxy parameters. We center our attention here in the radial acceleration correlation \citep{2016arXiv160905917M}, broadening the extent of applicability of the RAR from those including only DM as analyzed before. We thus consider a filtered SPARC sample of 2369 data points and apply a non-linear least square statistical analysis in order to check the goodness of fit for the RAR model, making also the same kind of analysis for NFW and DC14 models in order to properly compare among them. The task is then not only to compare between the different approaches but to see what more information can be gained by inspecting the behavior in the (free) physical parameter space of the RAR model depending on some galaxy characteristics.

% deviations importance (RESULTS!)
It is important to emphasize that we focus here on the relation between the acceleration of the dark and baryonic components (instead of the total and baryonic as usually). In this picture characteristic deviations from the empirical radial acceleration correlation are often present in the high acceleration regime (e.g. $\SYMvobs \approx \SYMvbar$), when inspecting single galaxies. Note that those deviations are qualitatively in contrast to the best-fit of \citet{2016arXiv160905917M} because their analysis is based on data limited by observational methods what remains the acceleration of the dark matter component poorly resolved in the baryonic matter dominated regime. An additional analysis of the much better resolved Milky Way contributes in a better understanding of the acceleration relation and its characteristic deviations from the best-fit.

% deviations source (RESULTS!)
Moreover, according to our RAR model we predict a different behavior, compared to the best-fit, in the very low acceleration end and even an increase of the dark matter acceleration in the high acceleration end after the baryonic matter dominated region. The increase is due to the massive quantum core in the galactic center, a fundamental feature of the RAR model. Exactly these deviations are a satisfying explanation of the scatter in the empirical radial acceleration correlation based on the average of many spiral galaxies.

% outline
In section \ref{sec:fitting} (\nameref{sec:fitting}) we describe the SPARC database, data selection, the halo models we use to fit the inferred DM rotation curve of each galaxy given in the SPARC sample and the fitting methods. The four parametric RAR model is applied for a fixed particle mass ($mc^2 = \SI{50}{\kilo\eV}$) what reduces the number of free parameters by one. For comparison we consider the NFW model, a two parametric empirical model motivated from N-body simulations \citep{1996ApJ...462..563N}, and the DC14 model, a baryonic feedback motivated halo model with three free parameters \citep{doi:10.1093/mnras/stu729}. 

In section \ref{sec:results} (\nameref{sec:results}) we present the results of the SPARC analysis, perform a goodness of model analysis to compare the competing dark matter models and give a satisfying explanation for the characteristic deviations (a scatter shower) in the acceleration correlation based on the prototype NGC0055 and the better resolved Milky Way. We also predict relations for different pairs of structural galaxy parameters for the 50keV-RAR model. Of special interest are the $M_{\rm BH}$-$M_{\rm tot}$ relation \citep{2002ApJ...578...90F,2011Natur.469..377K,2015ApJ...800..124B} and the constant surface density relation \citep{2009MNRAS.397.1169D}. Recently, \citet{arguelles_novel_2018} showed that RAR model is able to explain them for typical galaxies, ranging from dwarfs to elliptical. Here, we are going to enhance these relations with predictions inferred from disk galaxies of the SPARC data set.

Finally, in section \ref{sec:conclusion} (\nameref{sec:conclusion}) we give a brief summary and conclusion of our results.