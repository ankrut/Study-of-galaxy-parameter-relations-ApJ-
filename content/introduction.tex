%%%%%%%%%%%%%%%%%%%%%%%%%%%%%%%%%%%%%%%%%%%%%%%%%%%%
%%%%%%%%%%%%%%%%%%%%%%%%%%%%%%%%%%%%%%%%%%%%%%%%%%%%
\section{Introduction}
%%%%%%%%%%%%%%%%%%%%%%%%%%%%%%%%%%%%%%%%%%%%%%%%%%%%
%%%%%%%%%%%%%%%%%%%%%%%%%%%%%%%%%%%%%%%%%%%%%%%%%%%%

How the total gravitating mass distributes with respect to the luminous mass on galaxy scales is an open question which has regained much attention in the last decade thanks to the vast data-sets covering broader radial extents across different Hubble types \citep{2008AJ....136.2648D,2011MNRAS.413..813C,2016AJ....152..157L}.
Several universal relations exist between different pairs of structural galaxy parameters, which refer either \begin{inparaenum}[(i)]
    \item to the outer regions of galaxies such as the baryonic Tully-Fisher relation (BTFR) \citep{2000ApJ...533L..99M}, the DM surface density relation \citep{2009MNRAS.397.1169D}, the Radial Acceleration Relation \citep{2016PhRvL.117t1101M}, and the Mass Discrepancy Acceleration Relation (MDAR) \citep{2004ApJ...609..652M}, which are indeed all closely related \citep{2004ApJ...609..652M,2016arXiv161208857S,2018FoPh...48.1517S}; 
    \item to their central regions such as the $M$-$\sigma$ relation between the bulge's dispersion velocity and the central object mass \citep{2000ApJ...539L...9F}; or
    \item to a combination of both regimes such as the Ferrarese relation \citep{2002ApJ...578...90F,2011Natur.469..377K,2015ApJ...800..124B} between the total halo mass and its supermassive central object mass.
\end{inparaenum}

\add{Actual attempts for a unified understanding of many of the above scaling relations are typically given in terms of phenomenological halos obtained from N-body simulations within $\Lambda$CDM (see e.g. \citealp{2017MNRAS.471.1841N,2017PhRvL.118p1103L,2017MNRAS.464.2419S,2018FoPh...48.1517S}).    
However, when DM halos are formed through a MEP for collisionless systems of self-gravitating fermions \citep{1998MNRAS.300..981C,2015PhRvD..92l3527C,2020EPJP..135..290C,2021MNRAS.502.4227A}, it leaves place to novel theoretical predictions in the phenomenology of real galaxies \citep{2018PDU....21...82A,2019PDU....24..278A,2020A&A...641A..34B,2021MNRAS.505L..64B,2021MNRAS.502.4227A,2022MNRAS.511L..35A} such as:}

\begin{asparaenum}[(1)]
    \item DM fermions with a finite-temperature can be in a diluted (Boltzmannian-like) regime or become semi-degenerate. The corresponding DM halos can be, respectively, King-like or may develop a dense and degenerate compact core at the center of such a halo \citep{2015PhRvD..92l3527C,2021MNRAS.502.4227A}. A fully relativistic model in which this more general \textit{core}--\textit{halo} profiles arise, is usually referred in the literature as the Ruffini-Argüelles-Rueda (RAR) model \citep{2015MNRAS.451..622R,2018PDU....21...82A,2019PDU....24..278A,2020A&A...641A..34B,2021MNRAS.505L..64B,2021MNRAS.502.4227A,2022IJMPD..3130002A,2022MNRAS.511L..35A}. In either case, these fermionic DM profiles are cored (i.e. develop an extended plateau on halo scales similar to the Burkert \add{DM profile as shown in \cref{fig:profile-illustration-mep} and section \ref{sec:dark-matter-models}}), thereby not suffering from the core-cusp problem associated with the standard $\Lambda$CDM cosmology \citep{2017ARA&A..55..343B}. This cored feature seems to be a general conclusion reached for any DM profile which has reached a (quasi) thermodynamic equilibrium in cosmology \citep{2021MNRAS.504.2832S}.

    \item The relevance of the fermionic solutions with a degenerate DM core surrounded by a diluted halo (named from now on as \textit{core}--\textit{halo} profiles) imply different consequences: (a) the core might become so densely packed that above a threshold --- the critical mass --- the quantum pressure can not support it any longer against its own weight, leading to the gravitational core-collapse into a supermassive black hole (SMBH;  \citealp{2020EPJB...93..208A,2021MNRAS.502.4227A}). For DM particle masses of $\mathcal{O}(10)$ keV, this result provides, for large enough galaxies, a novel SMBH formation mechanism in the early Universe as proposed in \citet{2021MNRAS.502.4227A}; and (b) for core masses below its critical value, it exist a set of free parameters in the fermionic model, such that the quantum cores correlate with their outer halos \citep{2019PDU....24..278A} \add{explaining the relation between the total mass $M_{\rm tot}$ of large enough galaxies with the (presumably) embedded BH mass $M_{\rm BH}$, i.e. the Ferrarese relation (see above). However, such a $M_{\rm BH} - M_{\rm tot}$ relation} may not hold for smaller halos as shown for the SPARC data-set in \ref{sec:parameter-corelation:ferrarese}.

    \item For small enough galaxies as in the case of typical dwarfs, the degenerate core cannot collapse towards a BH (i.e. the total mass of the galaxy is below the critical mass of collapse) and thus it remains in a \textit{core}--\textit{halo} state where the central nucleus still mimics the effects of a singularity (with core masses in the range of Intermediate mass BHs) while the outer halo explains the rotation curves (RCs) \citep{2019PDU....24..278A,2021MNRAS.502.4227A,2022IJMPD..3130002A}.
    
    \item In the more extreme case where the fermions are fully-degenerate (i.e. when they are treated under the $T \to 0$ approximation), the corresponding halos are polytropic and may be only applicable to dwarfs \citep{2015JCAP...01..002D}.
\end{asparaenum}

Moreover, it was recently demonstrated that fermionic halos obtained via this \add{MEP} mechanism can arise in a cosmological framework, and remain thermodynamically and dynamically stable during the life of the Universe \citep{2021MNRAS.502.4227A}. Indeed, it was there shown the self-consistency of the approach, in the sense that the nature and mass of the DM particles involved in the linear matter power spectrum --- as calculated in \citet{2021MNRAS.502.4227A} within a CLASS code for $\mathcal{O}(10)$ keV fermions --- are the very same building blocks at the basis of the virialized DM configurations with its inherent effects in the DM profiles. 

\add{Thus, the main purpose of this work is to analyze most of the galaxy relations discussed above in (i)-(iii) together with a large set of observed RCs, assuming that DM halos are formed through a MEP in which the fermionic (quantum) nature of the DM particles is dully accounted for. This takes special interest since it is the first time a predictive model of this kind, i.e. based on first physical principles such as (quantum) statistical-mechanics and thermodynamics, is tested against a large set of galaxy observables while leading to a good agreement with observations. To see how well this fermionic model can reproduce the given observables, it will be compared with most of the commonly used DM models used in the literature. Further, we check which set of observed data can (or cannot) discriminate the goodness of the competing models.} 

\add{For this task, we will first focus on the Radial Acceleration Relation --- a non-linear correlation between the radial acceleration caused by the total matter and the one generated by its baryonic component only (see section \ref{sec:result:ac}) --- and on the directly related MDAR.}

\add{The main motivation to start studying these acceleration relations is due to the intense debate they have generated in the past few years about their underlying physical origin. One potential explanation for the Radial Acceleration Relation comes from Modified Newtonian Dynamics (MOND) \citep{2015CaJPh..93..169K,2016arXiv160906642M,2016PhRvL.117t1101M,2018A&A...615A...3L}, which has been used to interpret it as evidence against the $\Lambda$CDM paradigm and in favor to the MOND theory}. However, more recent studies dedicated to analyze this universal relation within the (Bayesian) posterior distributions on the acceleration scales of individual galaxies (across a large sample), have provided evidence against the existence of such a fundamental constant and in favour of $\mathfrak{a}_0$ to be an emergent magnitude (see e.g. \citealp{2020MNRAS.494.2875M} and references therein). On the other hand, it has been extensively shown that the Radial Acceleration Relation is consistent with the $\Lambda$CDM paradigm, as found either in hydrodynamical N-body simulations  \citep{2016MNRAS.456L.127D,2017MNRAS.471.1841N,2017PhRvL.118p1103L,2019MNRAS.485.1886D}, or from other more phenomenological (independent) study based on Universal Rotation Curves \citep{2018FoPh...48.1517S}.
% since it has been suggested that an acceleration scaling of this kind naturally arises within this paradigm
% and use other universal relations discussed above 

\add{Thus, in the first part of this work we will use the MEP approach for fermions to evaluate how competitive it is to reproduce the above relations with respect to other phenomenological DM halos. The galactic observables are taken from the} filtered SPARC sample of $2369$ data points (corresponding to a total of $120$ galaxies) and then apply a non-linear least square statistical analysis in order to check the goodness of fit of: 
%
\begin{inparaenum}[(a)]
    \item the above fermionic DM halos --- either in the \textit{core}--\textit{halo} regime \citep{2019PDU....24..278A,2021MNRAS.502.4227A} or in the purely King-like one \citep{2021MNRAS.502.4227A};
    \item a baryonic feedback motivated halo model within $\Lambda$CDM according to \citet{2014MNRAS.441.2986D}, named here as DC14;
    \item the classical NFW model based on early numerical simulations \citep{1997ApJ...490..493N};
    \item the Einasto profile \citep{1989A&A...223...89E,2006AJ....132.2685M}, and
    \item the Burkert model \citep{1995ApJ...447L..25B}.
\end{inparaenum} 
%}

%where DM halos are formed through a Maximum Entropy Principle (MEP) for collisionless self-gravitating systems as. It is based on the Fermi-Dirac (or Lynden-Bell) entropy functional, in which the fermionic (quantum) nature of the DM particles is accounted for in the phenomenology of real galaxies. 

%The main emergent features of such DM fermionic profiles, which differentiate them with respect to the $\Lambda$CDM N-body simulation ones, are: 

%The task is to compare and contrast the different models against the universal relations and individual RCs, as well as to see the physical insight that can be gained by inspecting the inner DM morphology of the statistically preferred DM profiles.

% outline
Outlining the structure of this paper, in section \ref{sec:fitting} we give a brief overview about the SPARC data set, data selection and fitting procedure. In section \ref{sec:dark-matter-models}, we describe the competing DM halo models considered above. In section \ref{sec:results}, we present our results on the Radial Acceleration Relation by performing a goodness of the fit for each DM model based on a filtered SPARC sample. Finally, in section \ref{sec:conclusion}, we give a brief summary and draw the conclusions.

We refer to the appendix for further details. In Appendix \ref{sec:appendix:parameter-distribution}, we provide model parameter distributions of the competing DM models. In Appendix \ref{sec:parameter-correlations}, we focus on the above mentioned fermionic model and analyze its consistency with different Universal relations such as the Ferrarese relation, comparing the total halo mass with its massive central object, and the DM surface density relation.
