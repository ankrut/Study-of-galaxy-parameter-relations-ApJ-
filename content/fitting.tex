%%%%%%%%%%%%%%%%%%%%%%%%%%%%%%%%%%%%%%%%%%%%%%%%%%%%
%%%%%%%%%%%%%%%%%%%%%%%%%%%%%%%%%%%%%%%%%%%%%%%%%%%%
\section{Methodology}
\label{sec:fitting}
%%%%%%%%%%%%%%%%%%%%%%%%%%%%%%%%%%%%%%%%%%%%%%%%%%%%
%%%%%%%%%%%%%%%%%%%%%%%%%%%%%%%%%%%%%%%%%%%%%%%%%%%%

The data used in this work is obtained from the Spitzer Photometry and Accurate Rotation Curves (\href{http://astroweb.cwru.edu/SPARC/}{SPARC}) data-set. It contains independent observations of the total velocity ($V_{\rm tot}$) and luminous mass distributions which allows to infer the bulge ($V_{\rm b}$), disk ($V_{\rm d}$), and gas ($V_{\rm g}$) velocity contributions. We extract the tangential velocities of the DM component from the data (for each galaxy of the sample) by subtracting the inferred baryonic components (as provided in the SPARC data-set) from the total velocity $V_{\rm tot}$, and statistically compare with the corresponding velocities predicted by each DM model as detailed in next. %Of great importance is the impact of baryonic matter onto the DM distribution.

%Early numerical simulations within the CDM framework predict a cuspy DM halo well known as the NFW model \citep{1996ApJ...462..563N}. Although its cuspy prediction has been in conflict with the observed cored halos from the beginning, it is often used for comparative reasons. For further information about the \textit{core-cusp} problem see \citet{2010AdAst2010E...5D}.

%It is important to recall here that early CDM simulations were limited to gravitational interaction and dark matter only. The strongest suspicion, therefore, was the neglect of the obvious baryonic components. The inclusion of baryonic matter and its effects onto the DM distribution is referred as baryonic feedback (e.g. supernovae, stellar wind, dynamical friction). However, all cosmological models currently adopt phenomenological implementations of many of these core processes, which must be tuned to observations. Many details of how these diverse processes interact within a hierarchical structure formation setting remain poorly understood \citep[see e.g.][]{2015ARA&A..53...51S}.

%Following the idea of baryonic feedback induced halos, \citet{doi:10.1093/mnras/stu729} has shown that including the influence of galaxy formation may flatten the inner density profile to transform halos from cuspy into cored. In that work they considered the Hernquist model \citep{1990ApJ...356..359H}, a generalized phenomenological halo model, and linked its parameters with baryonic matter.

%With a similar approach, although based on the Burkert model \citep{1995ApJ...447L..25B} producing cored halos, it was possible to reveal a link between characteristics of the DM and baryonic distributions in galaxies \citep{2018FoPh...48.1517S,2019ApJ...873..106D}.

%A completely different idea is followed by all MONDian approaches which claim that the observed total velocities and inferred velocities from light are linked through new underlying physics, so called Modified Newtonian Dynamics, without the assumption of any additional dark matter \citep{1983ApJ...270..365M,1983ApJ...270..371M,1983ApJ...270..384M}. Further information about  various modified gravity theories, within a wider context not limited only to the necessity of dark matter, are given in the extensive review by \citet{2012PhR...513....1C}.

%On the other hand, as mentioned in the introduction, our approach is based on first principles. That is, we assume dark matter to be composed of massive and self-gravitating fermions governed by quantum statistics and the relativistic gravitational equation. Such a DM distribution then shares many characteristics of phenomenological DM halo models. However, note that this approach does not include the interaction with baryonic matter. It is therefore of great interest to compare our approach with the approaches including baryonic feedback.

\subsection{Data selection}
\label{sec:data}
The SPARC data is distributed in separated files and can be found at \url{http://astroweb.cwru.edu/SPARC/}. Specific information about each galaxy (i.e Hubble type, inclination etc) are provided in the file \href{http://astroweb.cwru.edu/SPARC/Table1.mrt}{Table1.mrt}. The information we are interested in, like galactocentric radius $r$ and rotation curves $\SYMvel$, are provided in the file \href{http://astroweb.cwru.edu/SPARC/Table2.mrt}{Table2.mrt}.

In detail, we extract the observed circular velocity $\SYMvobs$ and the baryonic contribution $\SYMvbar$, composed of a bulge ($\SYMvbulge$), disk ($\SYMvdisk$) and gas component ($\SYMvgas$). The bulge and disk components are inferred from surface brightness observations for a given mass-to-light ratio. In sum, the baryonic component is given by \begin{equation}
	\label{eqn:baryonic-sum}
	\SYMvbar^2 = \Upsilon_\mathrm{b}^{\phantom{2}} \SYMvbulge^2 + \Upsilon_\mathrm{d}^{\phantom{2}} \SYMvdisk^2 + \SYMvgas^2
\end{equation} For convenience, the velocities $\SYMvbulge$ and $\SYMvdisk$ are provided for a mass-to-light ratio of $1\,M_\odot/L_\odot$ what does not represent the real value for a galaxy. Since the mass-to-light ratio is just a constant scaling factor we may correct the velocities simply with the mass-to-light \textit{ratio factors} $\Upsilon_b$ and $\Upsilon_d$ for bulge and disk (in units of $M_\odot/L_\odot$). Then the rotation curve for each component traces immediately its centripetal acceleration $\SYMacc = \SYMvel^2/r$.

For the data selection we proceed similar as was done by \citet{2016arXiv160905917M}. Thus, we want to note that we consider same mass-to-light ratios since the following data selection output depends on the values. For all bulges we choose $\Upsilon_\mathrm{b} = 0.7$ and for all disks $\Upsilon_\mathrm{g} = 0.5$ as convenient average representatives. Further, we exclude all galaxies with a bad quality flag ($Q=3$) and face-on galaxies with an inclination $i<30\degr$. Then we exclude all points with a velocity error greater than $10\%$ and all points where the baryonic velocity is greater than $95\%$ of the observed velocity. The latter affects mainly data points in the inner region which is dominated by baryonic matter and strongly depend on the chosen mass-to-light factors. Afterwards, we exclude all remaining galaxies with less then 6 data points and obtain 124 galaxies (out of 174) with 2396 points (of 3355) in total.
\subsection{Data fitting}
We fit the inferred DM rotation curve, $\SYMvdark^2 = \SYMvobs^2 - \SYMvobs^2$, with the Levenberg–Marquardt (LM) algorithm to find a $\chi^2$ minima. The quantity $\chi^2$ is calculated by \begin{equation}
	\chi^2(\vec p) = \sum \limits_{i=1}^N \qbracket{\frac{V_i - v(r_i,\vec p)}{\sigma_i}}^2
\end{equation} with $N$ the number of data points, $V_i$ is the set of circular velocity data, $r_i$ is the corresponding set of radius data, $v(r_i,\vec p)$ is the predicted circular velocity at radius $r_i$ for the model parameter vector $\vec p$ and $\sigma_i$ is the uncertainty for $V_i$.

For the RAR model, $\vec p = (\theta_0, W_0, \beta_0, m)$, we vary the three free parameter ($\theta_0, W_0, \beta_0$) for a fixed particle mass $m$. Due to numerical stability improvements we consider the cutoff parameter $W_0 =  1.73 \theta_0 + 1.07 + 10^\omega$ to make sure our fitting algorithm obtains only solutions with a halo (for approx. $W_0 < 1.73 \theta_0 + 1.07$ the halo gets disrupted and only a degenerate core remains). Phenomenally, we can vary the cutoff through the cutoff parameter $W_0$ and the steepness through the degeneracy parameter $\theta_0$. The latter is only possible within the transition regime ($\theta_0 \in [-5,15]$). For high degeneracy, $\theta_0 > 15$, we obtain a cored halo with a degenerate core. For these degenerate solutions we propose a particle mass of $m = 50 \mathrm{keV/c^2}$ as a promising candidate \citep{RAR-II}.

The NFW model describes a fixed cuspy halo with two free scaling parameter, e.g. $\vec p = (R_N,\rho_N)$. Therefore, that model can not explain the variation of the inner RC steepness or the variation in the cutoff. Instead, it is expected that NFW covers the rotation curves well on average due to its wide maxima bump.

The DC14 model, e.g. $\vec p = (X,R_N,\rho_N)$, has the additional parameter $X$ compared to NFW which affects the inner steepness and the maxima bump width simultaneously.

For the LM fitting algorithm we need well chosen initial values to ensure convergence. Because that algorithm finds only local minima we choose 100 parameter sets randomly within a range. For the RAR model we have $\theta_0\in [25,45]$, $\beta_0 = [10^{-8},10^{-5}]$ and $\omega\in[0,2]$. For NFW we have $R_N\in[10^{1},10^{4}],$ and $\rho_N = [10^{-4},10^{-1}]$. For the DC14 model we choose the same ranges as for the NFW model. Also, according to \citet{2016arXiv160505971K} we may bound the initial values of the additional parameter to $X\in[-3.75,-1.3]$. These ranges are no restrictions such that the fitting algorithm may find solutions beyond the boundaries.

