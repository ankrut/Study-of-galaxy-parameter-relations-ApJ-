%%%%%%%%%%%%%%%%%%%%%%%%%%%%%%%%%%%%%%%%%%%%%%%%%%%%
\subsection[Best-fit analysis]{Diversity of SPARC rotation curves}
\label{sec:fermionic-halos:diversity}
%%%%%%%%%%%%%%%%%%%%%%%%%%%%%%%%%%%%%%%%%%%%%%%%%%%%

\loadfigure{figure/BenchmarkTotalRotationCurves}

We show in this section a detailed $\chi^2$ analysis of the RC fits for three selected galaxies, each representing some characteristics of given observational data. We divide the SPARC galaxies in three groups by the inferred DM component as explained next. This analysis is based on the fermionic model where such a grouping seems to be appropriate to select galaxies with valuable predictions about the inner halo.

The first group, represented by UGC05986, shows only a single maximum in its DM RC, i.e. a rising trend in the inner halo followed by a clear turning point, as can be seen by the data points in the left plot of \cref{fig:benchmark:total-rotation-curves}. In the same plot for UGC05986 we show the best-fits of the competing DM models as thick curves, i.e. the fermionic (blue), DC14 (yellow) and NFW (red). Regarding the fermionic model, this kind of RC is better fitted by the solutions with a significant escape of particles ($W_p \lesssim 10$), as can be explicitly seen through the $\chi^2$ valleys in top panels of \cref{fig:chi-analysis}. However, due to the lack of information in the inner halo structures there is some uncertainty in the strength of particle escape. The uncertainty is physically better reflected in the core mass $M_c$ which covers about two orders of magnitude (see middle panel of first row in \cref{fig:chi-analysis}).

\loadfigure{figure/Chi2Analysis}

This result goes totally in line with an analogous phenomenological analysis \citep{2019PDU....24..278A}, developed for typical dwarf, spiral and elliptical galaxies within the RAR model. According to that analysis (done for $mc^2\approx \SI{50}{\kilo\eV}$), the maximal core mass of larger galaxies is limited by the critical configuration where the quantum core becomes unstable and collapses to a BH of mass $M_c^{cr} \approx \SI{2E8}{\Msun}$.

Among the cases, which are disfavored, are the ones with very large total DM masses $M_s$ corresponding to isothermal-like halos and implying negligible escape of particles ($W_p \gtrsim 10$). These solutions provide a minimal core mass $M_c$ with a huge uncertainty in the total mass.

The second group, represented by DDO161, shows a rising part in the RC towards a maximum without a clear turning point compared to the first group, see central plots in \cref{fig:benchmark:total-rotation-curves}. Fitting those galaxies for different $W_p$ values does not favor solutions with or without escaping particles effects. The variation in the $\chi^2$ value remains rather small, see middle panels of \cref{fig:chi-analysis}.

Finally, the third group, represented by NGC6015, shows some oscillations in the RC, mainly in the outer halo, see right plots in \cref{fig:benchmark:total-rotation-curves}. There are various and speculative reasons for the oscillation, e.g., ongoing merging process, deviation from equilibrium, etc. In any case, those galaxies are clearly better fitted by extended isothermal-like halos ($W_p \gtrsim 10$) --- although being far from good --- see bottom panels of \cref{fig:chi-analysis}. Such solutions provide a wide halo maximum followed by a flat RC. In contrast, the non-isothermal solutions with a cutoff provide only a narrow maximum in the halo, followed by a Keplerian decreasing tail.

It is worth to recall that different DM models such as the fermionic model, NFW, DC14 and others are not appropriate to fit the oscillations, characterized through multiple maxima in the RC. All solutions with a wide halo are suitable to fit the oscillations well on average, although the best-fits remain rather poor, leaving almost no insight into the physical properties of DM on halo scales for those galaxies.
