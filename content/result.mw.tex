\subsection{Milky Way}
\label{sec:mw}
We change now the focus to the Milky Way galaxy which is much better resolved than the SPARC galaxies. In particular, the widely covered radial extent, from the center to the outer halo, gives a better insight of the mass discrepancy.

The rotation curve for the Milky Way is given by \citet{2013PASJ...65..118S}. In comparison to the SPARC data the baryonic components (bulge and disk) are given only analytically.

The bulge structure is composed of a main and an inner bulge. Each component is given by an exponential sphere model \begin{equation}
	\frac{\rho(r)}{\rho_b} = \e^{-r/R_b}
\end{equation} with a central density $\rho_b$ and a length scale $R_b$. The mass is then given by $M(r) = \int_0^r 4\pi r^2 \rho(r) \d r$ and the circular velocity becomes \begin{equation}
	\frac{v^2(r)}{\sigma_b^2} = \frac{M(r)}{M_b} \frac{R_b}{R}
\end{equation} The scale factors are defined by $M_b = 4\pi \rho_b R_b^3$ and $\sigma_b^2 = G M_b/R_b$.

The disk is given by an exponential disk model where the surface density follows an exponential law, \begin{equation}
	\frac{\Sigma(r)}{\Sigma_d} = \e^{-r/R_d}
\end{equation} with the parameters $\Sigma_d$ and $R_d$. The mass is then given by $M(r) = \int_0^r 2\pi r \Sigma(r) \d r$. Since the dynamics for an axial symmetric mass distribution differ from a spherical symmetric mass distribution we have to use the circular velocity (on the equatorial plane) given by \begin{equation}
	\frac{v^2(r)}{\sigma_d^2} = 2 y^2 \qbracket{I_0(y) K_0(y) - I_1(y) K_1(y)}
\end{equation} with the substitution $y = r/(2R_d)$ and the modified Bessel functions $I_n(x)$ and $K_n(x)$ \citep{Binney2008}. The scale factors are defined by $M_d = 2\pi \Sigma_d R_d^2$ and $\sigma_d^2 = G M_d/R_d$.


\loadfigure{figure/MWrc}
\begin{table}
  \centering
  \begin{tabular}{@{}lcc@{}}
  \toprule
  component & total mass ($M_\odot$) & length scale (kpc) \\ \midrule
  inner bulge    & $5.5 \times 10^7$      & $4.07 \times 10^{-3}$\\
  main bulge     & $9.7 \times 10^9$      & $0.137$             \\
  disk           & $6.0 \times 10^{10}$   & $2.92$              \\ \bottomrule
  \end{tabular}
	\caption{Parameter for baryonic components of Milky Way. Note: baryonic and especially disk parameter imply RAR-DM component described by $mc^2=50 \mathrm{keV}, \theta_0 \approx 37.8, W_0 \approx 66.4$ and $\beta_0 \approx 7.66 \times 10^{-6}$}
  \label{tbl:mw:parameter}
\end{table}

In sum, the RAR model is able to describe the halo and explains simultaneously the compact object in the Galactic center, see \cref{fig:mw-rc}. Further details, especially concerning the supermassive black hole alternative in the Galactic center, are provided in \cite{arguelles_novel_2018}.

Here, we are mainly interested in the centripetal acceleration given by $a(r) = v^2(r)/r$. The length scales $R_b$ and $R_d$ as well as the masses $M_b$ and $M_d$, describing the total bulge mass and disk mass, are provided in \citet{2013PASJ...65..118S} and \cref{tbl:mw:parameter}. Important attention has to be given to the disk parameters since they depend on the chosen dark matter model and have to be adjusted. Then, the baryonic contribution is the sum of inner bulge, main bulge and disk components. For the total contribution we have to add also the dark matter component.

\loadfigure{figure/MWac}

In \cref{fig:mw-ac} we find fundamental differences in the acceleration correlation compared to McGaugh's empirical fit. In the low acceleration regime (dark matter dominated) we obtain a linear proportionality what is due to the necessary strong cutoff in the halo. After that it follows the transition into the baryonic matter dominated regime with decreasing dark matter acceleration similar to McGaugh. However, in the high acceleration regime we have again an increase of dark matter acceleration because of the degenerate dark matter core, resembling a supermassive dark object, what is a fundamental feature of the RAR model.

For comparison, we have added the NFW model. The results show a similar trend in the low acceleration regime, but diverge towards a constant acceleration of the dark matter component in the high acceleration regime. Thus, all shown predictions (RAR, NFW and McGaugh) have a very different behavior in the baryonic dominated region.