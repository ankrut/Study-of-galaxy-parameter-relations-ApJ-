%%%%%%%%%%%%%%%%%%%%%%%%%%%%%%%%%%%%%%%%%%%%%%%%%%%%
\subsection{Morphology of fermionic halos}
\label{sec:morph}
%%%%%%%%%%%%%%%%%%%%%%%%%%%%%%%%%%%%%%%%%%%%%%%%%%%%

We are interested in providing approximate analytic (and semi-analytic) expressions for the halo morphology associated with the corresponding fermionic solutions.

On halo scales, i.e. for $r \sim r_h$ with $r_h$ being the halo radius, the fermionic DM model resembles the King model \citep{1966AJ.....71...64K}. These DM profiles are characterized by a cored inner halo (i.e. a flat inner slope) followed by a transition towards a finite mass. When applied to galactic halos, such a transition can show different behaviours leading to a rich morphology of density profiles, from polytropic-like to power law-like, mainly depending on the value of $W_p$. Consequently, this cutoff (or particle escape) parameter controls the sharpness of the inner-outer halo transition as well.

Very similar characteristics are also produced by the Einasto model, although with a wider spectrum for the sharpness of the transition described by the parameter $\kappa$.

In the fermionic model, the sharpness of the transition can be quantified as follows: for $W_p\ll 1$, the fermionic density profiles are polytropes of index $n=5/2$ as clearly shown in \cref{fig:halo-profiles} for bluish solutions (see \citealp{2015PhRvD..92l3527C} for a derivation of the polytropic $n=5/2$ equation and its link with the fermionic solutions, further justifying our results). In contrast, for negligible escape of particles (e.g. $W_p \gtrsim 10$), the fermionic halos become isothermal-like, with $\rho(r) \propto r^{-2}$ down to the virial radius as evidenced through the reddish solutions in the same figure. Finally, for $W_p$ values in between the above limiting cases, the fermionic DM halo profiles start to develop a power-law trend almost matching the Burkert profile for $W_p\approx 7$ (see \cref{fig:halo-profiles}).

These results agree with those obtained in Sec. VII of \citet{2015PhRvD..91f3531C} for the classical King model, which correspond to our fermionic model in the dilute regime (see also paragraph below). There it is shown that the Burkert profile gives a good fit of such King distribution for a value of $W_p$ close to the point of marginal thermodynamical stability, the latter given at $W_p^{(c)} = 7.45$. In \citet{2015PhRvD..91f3531C} this marginal solution is labelled through the equivalent parameter $k$.

Statistically, we identified $W_p \approx 10$ as a discriminator between two groups and further detailed in Appendix \ref{sec:appendix:parameter-distribution}. Even if the precise value might be biased by the data, there is an interesting underlying physical explanation for it: fermionic DM profiles as obtained from a MEP are thermodynamically and dynamically stable for $W_p \leq 7.45$ if they are in the dilute regime ($\theta_0 \ll -1$ where $W_0 \equiv W_p$), while in the core-halo regime ($\theta_0 > 10$) the same stability conditions hold for a bounded $W_p \in [W_p^{\rm min}, W_p^{\rm max}]$. For typical SPARC galaxies with a halo mass of $\sim \SI{5E10}{\Msun}$, we find $W_p^{\rm max}\sim \num{E-2}$.

\loadfigure{figure/HaloProfilesComparison}

The point of stability-change for the classical King model was first obtained in \cite{2015PhRvD..91f3531C}, and further re-derived here for fermionic DM profiles but this time for realistic average halo-mass galaxies, as obtained from the SPARC data-set, following the thermodynamic analysis of \cite{2021MNRAS.502.4227A}. That is, typical galaxies with halo masses of about $\sim \SI{5E10}{\Msun}$ and with appreciable escape of particles (i.e. $W_p \ll 1$) are thermodynamically and dynamically stable, suggesting a deep link between thermodynamics of self-gravitating systems and galaxy formation. A case-by-case stability analysis in relation to observed galaxies (e.g. SPARC data-set) is out of the scope of the present paper and will be the subject of a future work. 
