%%%%%%%%%%%%%%%%%%%%%%%%%%%%%%%%%%%%%%%%%%%%%%%%%%%%
%%%%%%%%%%%%%%%%%%%%%%%%%%%%%%%%%%%%%%%%%%%%%%%%%%%%
\section{Parameter distributions}
\label{sec:appendix:parameter-distribution}
%%%%%%%%%%%%%%%%%%%%%%%%%%%%%%%%%%%%%%%%%%%%%%%%%%%%
%%%%%%%%%%%%%%%%%%%%%%%%%%%%%%%%%%%%%%%%%%%%%%%%%%%%


\loadfigure{figure/ParameterDistribution_MEPP}
\loadfigure{figure/ParameterDistribution_DC14}
\loadfigure{figure/ParameterDistribution_Einasto}

We analysed the distribution of configuration parameters which affect the shape of the DM RC for each DM model. On this basis, we are interested only in fermionic, DC14 and Einasto models, which are the only ones where the halo slopes (usually characterized by $\gamma$) depend on one configuration parameter.
% The scaling parameters (e.g. $R_{\rm N}$ or $\rho_{\rm N}$) do not show any insights.

Starting with the core-halo solutions, see upper plots in \cref{fig:parameter-distribution:mepp}, we find that the majority of central temperature values falls in the range $\beta_0 \in [\num{E-8}, \num{E-6}]$ corresponding to solutions in which the central DM cores are far from reaching the core-collapse towards a SMBH \citep{2019PDU....24..278A,2021MNRAS.502.4227A}. The distribution of the central degeneracy values looks like a Gaussian with the mean value at about 36, far in the degenerate regime $\theta_0 \gtrsim 10$. The majority is in the range $\theta_0 \in [30, 40]$. A similar distribution pattern is given for $W_0$. However, for core-halo solutions (i.e. $\theta_0 \gtrsim 10$) it is better to look at the plateau cutoff $W_p$ which can be identified with $W_0$ of a corresponding halo-only solution.


The plateau of a core-halo solution acts as a connection between the halo and the embedded core. Therefore, every core-halo solution of the fermionic model has a corresponding halo-only solution describing the diluted halo without the embedded, degenerate core. The corresponding values for such halo-only solutions are given at the plateau which resemble the inner halo, see lower plots in \cref{fig:parameter-distribution:mepp}. The plateau temperature shows a very similar distribution due to the tiny temperature changes when in the low temperature regime ($\beta_0 \ll \num{E-4}$) where pressure effects are negligible leading to $\beta_p \approx \beta_0$. The plateau degeneracy distribution looks also similar to the central degeneracy but being mirrored and shifted to the negative (diluted) regime. We find the relation $\theta_p \approx -0.7 \theta_0 - 1.2$.

Of great interest is the plateau cutoff $W_p$ which is a proxy for the central cutoff $W_0$ parameter and provides better insights about the halo shape. The plateau cutoff describes the particle escape intensity on halo scales and is defined as $W_p = W(r_p)$ with the plateau radius $r_p$ located at the first minimum in the DM RC. The lower $W_p$ the more truncated is the halo due to evaporation. See also section \ref{sec:morph} for a discussion about the halo morphology and comparison with other DM models.

Within the fermionic model we identify two groups in the $W_p$ distribution of core-halo solutions and $W_0$ distribution of halo-only solution, respectively: (1) an isothermal (non-truncated) group with 76 galaxies and (2) a non-isothermal (truncated) group with 44 galaxies. Both groups are divided at about $W_p \approx 10$ (core-halo) and $W_0 \approx 10$ (halo-only), respectively.

For the first group the outer halo seems to be isothermal (i.e. characterized by a flat tail and $\rho(r) \propto r^{-2}$ for large enough $r$). However, for those galaxies the exact value of $W_0$ cannot be determined due to insufficient and/or too limited information in the outer halo data, and thus they are not shown in \cref{fig:parameter-distribution:mepp}. In contrast, the second group seems to have a broader data coverage and/or admitting for a cleaner description of the outer RC, allowing for a better constrain of the outer halo of the fermionic DM profiles (i.e. with finite $W_p \lesssim 10$ and consequent non-isothermal halo tails). See \cref{fig:halo-profiles} for a comparison between typical solution of both groups.

For halo-only solutions (i.e. $\theta_0 \lesssim -5)$ there is an interesting accumulation of galaxies around the particular value of $W_0 \approx 7.45$, see dashed line in \cref{fig:parameter-distribution:mepp}. This particular value reflects the point of stability change where the diluted solutions become unstable, i.e. for $W_0 \gtrsim 7.45$ \citep{2015PhRvD..91f3531C}, and may indicate physical insights into galaxy formation. Interestingly, for this value of $W_0$ the density profile of the classical King model resembles the Burkert profile (see section \ref{sec:morph}). %Nevertheless, keep in mind that already for $W_0 \gtrsim 10$ data does not allow to safely determine the central cutoff $W_0$.

In the case of the DC14 model the majority of best-fitted $X$ values is in the range $[-5, 0]$ with a peak at the median of $X \approx -2.4$ as indicated by the boxplot in \cref{fig:parameter-distribution:dc14}. The crosses may be outliers. When we calculate the parameters $\alpha$, $\beta$ and $\gamma$ with \cref{eqn:dc14:alpha,eqn:dc14:beta,eqn:dc14:gamma}, there seems to be a grouping in the distribution of $\alpha$ with a separation at $\alpha = 0$, see \cref{fig:parameter-distribution:dc14}. However, \cref{eqn:hernquist} is not defined for $\alpha = 0$. Phenomenologically, $\alpha$ describes the transition from the inner to the outer halo. The larger $\abs{\alpha}$ the more extended is the transition, characterized by a long wide maximum. $\beta$ describes the slope in the outer halo while $\gamma$ describes the slope in the inner halo. For $\gamma = 0$ the DM profiles become cored.

The Einasto model has only a single configuration parameter $\kappa$ describing the transition from the cored inner halo to the outer halo. The larger $\kappa$ the less extended is the RC maximum. As shown in \cref{fig:parameter-distribution:einasto} the majority has $\kappa < 1$ that corresponds to a rather extended RC maximum. This distribution represents well the majority of SPARC galaxies showing an extended outer halo trend without a clear maximum in the outer RC.
% with $v \sim r^{-2} due the the finite total mass.$ 
