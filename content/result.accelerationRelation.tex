%%%%%%%%%%%%%%%%%%%%%%%%%%%%%%%%%%%%%%%%%%%%%%%%%%%%
\subsection{Acceleration relations}
\label{sec:result:ac}
%%%%%%%%%%%%%%%%%%%%%%%%%%%%%%%%%%%%%%%%%%%%%%%%%%%%

\loadfigure{figure/AccelerationGrid}

% radial acceleration relation
The Radial Acceleration Relation connects the centripetal accelerations of the baryonic and total matter components. This relation is not limited to disk galaxies but also holds for other galaxy types (e.g. ellipticals, lenticulars, dwarfs spheroidals and even low-surface-brightness galaxies), what makes it a true universal law among morphology classification \citep{2016PhRvL.117t1101M,2017ApJ...836..152L,2019ApJ...873..106D}. Despite a relatively large scatter, this relation on halo scales seems to imply a fundamental acceleration scale $\mathfrak{a}_0$, which is present in the non-linear fitting function as obtained in \cite{2016PhRvL.117t1101M}, and given by
% : it shows a non-linear correlation between the radial acceleration caused by the total matter and the one generated by its baryonic component only.
%
\begin{equation}
	\label{eq:mcgaugh-fit}
	\SYMatot = \frac{\SYMabar}{1 - \e^{-\sqrt{\SYMabar/{\SYMafrak}}}},
\end{equation}
%
where $\SYMafrak$ is the only adjustable parameter. In the low acceleration regime ($\SYMabar \ll \SYMafrak$), where DM dominates, it clearly shows a deviation from a linear correlation (see top panels of \cref{fig:acceleration:grid}). While in the high acceleration regime ($\SYMabar \gg \SYMafrak$), dominated by baryonic matter, the linear relation is recovered.

% different explanations
%An acceleration scale of this kind naturally arises in Modified Newtonian Dynamics (MOND) \citep{2015CaJPh..93..169K,2016arXiv160906642M,2016PhRvL.117t1101M,2018A&A...615A...3L}, which has been used to interpret such a fundamental scale as evidence against the $\Lambda$CDM paradigm and in favor to the MOND theory. However, more recent studies dedicated to analyze this universal relation within the (Bayesian) posterior distributions on the acceleration scales of individual galaxies (across a large sample), have provided evidence against the existence of such a fundamental constant and in favour of $\mathfrak{a}_0$ to be an emergent magnitude (see e.g. \citealp{2020MNRAS.494.2875M} and references therein). On the other hand, it has been extensively shown that the Radial Acceleration Relation is consistent with the $\Lambda$CDM paradigm, as found in hydrodynamical N-body simulations  \citep{2016MNRAS.456L.127D,2017MNRAS.471.1841N,2017PhRvL.118p1103L,2019MNRAS.485.1886D}. This conclusion is in line with a more phenomenological (independent) study based on Universal Rotation Curves including for baryonic mass models \citep{2018FoPh...48.1517S}.

The rotation curve for each component (e.g. bulge, disk, gas) traces its centripetal acceleration $\SYMacc = v^2/r$, giving access to independent acceleration measurements. The Radial Acceleration Relation compares the radial acceleration due to the total mass ($\SYMatot$) with that due to the baryonic mass ($\SYMabar$). \citet{2016PhRvL.117t1101M} describes that correlation empirically by \cref{eq:mcgaugh-fit}.

% source of data
\add{For the SPARC galaxies $\SYMabar = V_{\rm bar}^2/r$ and $\SYMatot = V_{\rm tot}^2/r$ are inferred from the SPARC data. See section \ref{sec:data} for details how the circular velocities and radii are obtained. For the competing DM models $\SYMabar = V_{\rm bar}^2/r$ is equally inferred from the SPARC data while $\SYMatot = [V_{\rm bar}^2 + v_{\rm DM}(r)^2]/r$ is inferred from the total mass distribution, composed of the observed baryonic component (taken from SPARC data) and the best-fitted circular velocities of the dark matter component $v_{\rm DM}(r)$ for each DM model. See sections \ref{sec:fitting} and \ref{boundaryC} for details how best-fits are obtained.}

% RAR/MDAR reproduction qualitatively
For the SPARC data as well as for each DM model, we applied then a least-square fitting to obtain $\SYMafrak$. The result obtained here from the SPARC data only (i.e. without assuming any specific DM model) is fully consistent (within errors) with $\SYMafrak \approx \SI{1.2E-10}{\metre\per\second^2}$ as obtained originally in \citet{2016PhRvL.117t1101M}, validating our procedure. Each DM model is then characterized by a specific best-fit $\SYMafrak$. The corresponding curves are plotted as solid lines in \cref{fig:acceleration:grid}. In all cases, we obtain values close to the one obtained in \citet{2016PhRvL.117t1101M}. This allows to conclude that all competing DM models are able to reproduce the Radial Acceleration Relation. Moreover, they reproduce it equally good without a clear statistically preferred model.

% mass discrepancy acceleration relation
A closely related relation is the MDAR relation between the baryonic and total mass components, defined by $D = M_{\rm tot}/M_{\rm bar}$ \add{with $M_{\rm bar}$ being the total baryonic mass and $M_{\rm tot}$ the total galaxy mass accounting for baryonic and dark matter}. For $a = \diff{\Phi}{r}$ and $\Phi(r)$ being the gravitational potential of a spherically symmetry mass distribution the mass discrepancy can be equivalently written as $D = \SYMatot/\SYMabar$. The results are illustrated in the bottom panels of \cref{fig:acceleration:grid}.

% beyond RAR/MDAR
According to some authors, the above two acceleration relations do not imply the need of any new physics and may be explained within the $\Lambda$CDM framework \citep{2016MNRAS.456L.127D,2017MNRAS.466.1648K,2016arXiv161208857S,2018FoPh...48.1517S}. For smaller disk and LSB galaxies (extending the original SPARC sample), \citet{2019ApJ...873..106D} found that the McGaugh et al. relation is a limiting case of a more complex relation with the need of adding one extra galaxy parameter. In addition, based on modern cosmological simulations, \citet{2017ApJ...835L..17K} predict even a redshift dependency of the acceleration parameter $\SYMafrak$, emphasizing that the correlation is universal only regarding the morphological classification.

% next strategy
Additionally to the acceleration relations mentioned above (which account for the entire accelerations distributions among all galaxies and at different radii), the next strategy is to focus on the DM components of each galaxy and gather best-fits of the inferred rotation curves allowing for another (related) quantitative comparison of the DM models.
