\loadfigure{figure/AccelerationGrid}

%%%%%%%%%%%%%%%%%%%%%%%%%%%%%%%%%%%%%%%%%%%%%%%%%%%%
\subsection{Acceleration relations}
\label{sec:result:ac}
%%%%%%%%%%%%%%%%%%%%%%%%%%%%%%%%%%%%%%%%%%%%%%%%%%%%

% radial acceleration relation
The RC for each component (e.g. bulge, disk, gas) traces its centripetal acceleration $\SYMacc = v^2/r$, giving access to independent acceleration measurements. The Radial Acceleration Relation compares the radial acceleration due to the total mass ($\SYMatot$) with that due to the baryonic mass ($\SYMabar$). \citet{2016PhRvL.117t1101M} describes that correlation empirically by \cref{eq:mcgaugh-fit}.

% source of data
\add{For the SPARC galaxies $\SYMabar$ and $\SYMatot$ are inferred from the SPARC data. For the competing DM models $\SYMabar$ is equally inferred from the SPARC data while $\SYMatot$ is inferred from the total mass distribution: composed of the measured baryonic component (i.e SPARC data) and the given DM component (i.e. best-fit).}

% RAR/MDAR reproduction qualitatively
For the SPARC data as well as for each DM model, we applied then a least-square fitting to obtain $\SYMafrak$. The result obtained here from the SPARC data only (i.e. without assuming any specific DM model) is fully consistent (within errors) with $\SYMafrak \approx \SI{1.2E-10}{\metre\per\second^2}$ as obtained originally in \citet{2016PhRvL.117t1101M}, validating our procedure. Each DM model is then characterized by a specific best-fit $\SYMafrak$. The corresponding curves are plotted as solid lines in \cref{fig:acceleration:grid}. In all cases, we obtain values close to the one obtained in \citet{2016PhRvL.117t1101M}. This allows to conclude that all competing DM models are able to reproduce the Radial Acceleration Relation. Moreover, they reproduce it equally good without a clear statistically preferred model.

% mass discrepancy acceleration relation
A closely related relation is the MDAR relation between the baryonic and total mass components, defined by $D = M_{\rm tot}/M_{\rm bar}$.  For $a = \diff{\Phi}{r}$ and $\Phi(r)$ being the gravitational potential of a spherically symmetry mass distribution the mass discrepancy can be equivalently written as $D = \SYMatot/\SYMabar$. The results are illustrated in the bottom panels of \cref{fig:acceleration:grid}.

% beyond RAR/MDAR
According to some authors, the above two acceleration relations do not imply the need of any new physics and may be explained within the $\Lambda$CDM framework \citep{2016MNRAS.456L.127D,2017MNRAS.466.1648K,2016arXiv161208857S,2018FoPh...48.1517S}. For smaller disk and LSB galaxies (extending the original SPARC sample), \citet{2019ApJ...873..106D} found that the McGaugh et al. relation is a limiting case of a more complex relation with the need of adding one extra galaxy parameter. In addition, based on modern cosmological simulations, \citet{2017ApJ...835L..17K} predict even a redshift dependency of the acceleration parameter $\SYMafrak$, emphasizing that the correlation is universal only regarding the morphological classification.

% halo focus critics
%Moreover, both relations have been criticized that their representation focus only on the low acceleration regime where DM dominates. Therefore, \citet{2019ApJ...877...18C} suggested to compare the baryonic component $\SYMabar$, with the ratio $\SYMaDM/\SYMabar$ which gives also information about DM in the high acceleration regime, dominated by baryonic matter. They realized that the original representation of such acceleration relations shows a link on halo scales but simultaneously obscures the relation between dark and baryonic matter on inner halo scales. Therefore, any DM model (cored or cuspy) would reproduce the linear relation in the baryons dominated region. However, due to rather large uncertainties in that region, any inferred relation between dark and baryonic components would be highly speculative.

% next strategy
Additionally to the acceleration relations mentioned above (which account for the entire accelerations distributions among all galaxies and at different radii), the next strategy is to focus on the DM components of each galaxy and gather best-fits of the inferred rotation curves allowing for another (related) quantitative comparison of the DM models.