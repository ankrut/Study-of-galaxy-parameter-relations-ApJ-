%%%%%%%%%%%%%%%%%%%%%%%%%%%%%%%%%%%%%%%%%%%%%%%%%%%%
\subsection{Data fitting}
\label{LM-fitting}
%%%%%%%%%%%%%%%%%%%%%%%%%%%%%%%%%%%%%%%%%%%%%%%%%%%%

We infer the circular velocity of the DM component directly from observations via 
%
\begin{equation}
	\label{eqn:baryonic-diff}
	\SYMvdark^2 = \SYMvtot^2 - \SYMvbar^2,
\end{equation} 
%
where the uncertainty in $V_{\rm DM}$ is calculated from the uncertainties in the other velocity components within linear error propagation theory. Note that only the uncertainties in the total RCs ($\Delta V_{\rm tot}$) are provided within the SPARC data-set. The uncertainty is therefore given by
%
\begin{equation}\label{eqn:VDM-error}
    \Delta V_{\rm DM} = \abs{\diffpart{V_{\rm DM}}{V_{\rm tot}}} \Delta V_{\rm tot}.
\end{equation} 
%
The inferred DM contribution of each galaxy then will be fitted by the competing DM models which are described in section \ref{sec:dark-matter-models}.

With this information, we use the Levenberg–Marquardt (LM) algorithm of least-square error minimization for each dark matter component calculated by 
%
\begin{equation}
    \label{eqn:method:chi-square}
	\chi^2(\vec p) = \sum \limits_{i=1}^N \qbracket{\frac{V_i - v(r_i,\vec p)}{\Delta V_{i}}}^2.
\end{equation} 
%
Here, $N$ is the number of data points for a given galaxy, $V_i$ is the set of inferred DM circular velocities from the data at each corresponding measured radius $r_i$, while $v(r_i,\vec p)$ accounts for the circular velocity at $r_i$ for each model parameter vector $\vec p$ (described below), and $\Delta V_{i}$ is the uncertainty in $V_i$ as given in \cref{eqn:VDM-error}.
