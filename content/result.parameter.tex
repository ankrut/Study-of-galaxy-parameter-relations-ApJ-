\subsection{Correlations}
In this section we analyze different pairs of structural galaxy parameters obtained from the $50\,{\rm keV}$-RAR model, such as $\theta_0$ and $\beta_0$, the halo radius $r_h$ and mass $M_h$ as well as the core radius $r_c$ and core mass $M_c$. Further, we are also interested in the total dark matter mass what needs a careful definition here.

% boundary mass definition
Usually, mass distributions infinite in mass and space are truncated for example at a critical radius or density to obtain reasonable values. In that fashion, we define the \textit{boundary mass} $M_b = M(r_b)$, being $r_b$ the boundary radius where the density falls to the critical density of the Local Group ($10^{-5} M_\odot/{\rm pc}^3$).

% surface mass definition
On contrary, the RAR model provides naturally finite mass distributions via the cutoff parameter $W_0$. Thus, we define the \textit{surface mass} $M_s = M(r_s)$, being $r_s$ the natural surface radius where the density falls to zero. Important, for strong cutoff values we have total masses $M_s \approx M_b$. 

% galaxy classification
It is important to emphasize that appropriate information about the halo, especially the outer halo, are needed to determine the cutoff parameter $W_0$. Thus, we divide the galaxy sample in two groups: galaxies with \begin{itemize}
\item \textit{appropriate} halo information\\($M_{\rm tot} = M_b \approx M_s$) 
\item \textit{inappropriate} halo information\\($M_{\rm tot} = M_b \ll M_s$)
\end{itemize} Here, \textit{appropriate} refers to galaxies with sufficient information in the dark matter rotation curve (e.g. a clear maximum) while \textit{inappropriate} refers either to a lack (e.g. no clear maximum) or to an abundance of information (e.g. oscillations in the outer halo).

% flat tail fits
Additionally, we introduce another group, the so called \textit{flat-tail fits}. All candidates of this group have in common that their whole dark matter distribution is fitted well by the flat tail of the RAR model which implies no (or weak) evaporation. Due to the poor data of those galaxies we consider their results as an artifact of the fitting method. Thus, flat-tail fits produce dark matter distributions with too little halo radii and therefore too high halo and core masses. We speculate therefore that a proper choice of parameters, following the relation \begin{equation}
	\label{eqn:rel:Mh-rh}
	\ln \frac{M_h}{M_\odot} \approx 2.02\ln \frac{r_h}{\mathrm{pc}} + 5.10 \pm 1.54
\end{equation} as found in the $r_h$-$M_h$ plot (see left diagram of \cref{fig:param_50keV-RAR}), would produce adequate dark matter profiles, sufficient to describe the rotation curves with slightly different $\chi^2$ values.


% obtained galaxy parameter range
All considered solutions of the RAR model have always a degenerate core and a diluted halo. The core is defined as the first maxima in the rotation curve and the halo is defined as the second maxima. We obtain halo radii mainly in the interval $[10^3,10^5]\mathrm{pc}$ and halo masses in the interval $[10^8,10^{12}] M_\odot$. For the corresponding cores we obtain radii and masses in the intervals $[10^{-4},10^{-2}]\mathrm{pc}$ and $[10^3,10^{7}] M_\odot$.

% quantum core relation
It is important to note that the quantum core is well described by a fully degenerate core. Analytically, we find the simple relation $M_c r_c^3 \sim m^{-8}$. Because we have set the particle mass, the mass-radius relation for the core has no predictive character here.

% core-halo relation
Of more interest is the core-halo relation for the mass. In that projection we clearly identify a distinction between the appropriate and the inappropriate halo groups. Both follow approx. the relation $M_h \sim M_c^2$ while the appropriate halo group shows a higher diversity. Following the best-fit analysis of NGC0055 (see. \cref{fig:NGC0055:deep-chi2}), the diversity is most probably due to some uncertainty in the cutoff parameter $W_0$.

% degeneracy and temperature regime classification
The group distinction with same characteristics is also visible in the $\theta_0$-$\beta_0$ parameter space, see \cref{fig:param_50keV-RAR}. All best fits have a central degeneracy parameter in the degenerate regime, $\theta_0 \in [20,50]$. For the central temperature parameter we obtain the range $[10^{-9},10^{-5}]$ which corresponds to the low temperature regime with negligible pressure effects.

% outlier
Only for one galaxy (NGC4088) we find $\beta_0 \sim 10^{4}$ where pressure effects are not negligible any more. But that galaxy has a fundamental lack of information about the halo. Data shows only the inner part of the halo without any trend towards a maxima. This allows only to connect the galaxy with the $r_h$-$M_h$ correlation group described by \cref{eqn:rel:Mh-rh}. Without any information about the outer halo (e.g. maxima) it is impossible to predict a narrow window of halo radii and masses. In the following we consider this candidate as an outlier.


\loadfigure{figure/ParamCorrelations}
\loadfigure{figure/Ferrarese}


\subsubsection{Central core vs. total halo mass relation}
We move now to the $M_{\rm BH} - M_{\rm tot}$ relation \citep{2002ApJ...578...90F,2011Natur.469..377K,2015ApJ...800..124B} where $M_{\rm tot}$ is the total dark matter halo mass and $M_{\rm BH}$ is the mass of the compact dark object at the center of galaxies. Traditionally, the central dark objects are assumed as SMBHs but here interpreted as dark matter quantum cores in the case of inactive galaxies. In the following we consider $M_{\rm BH} = M_c$, being $M_c = M(r_c)$ the quantum core mass. \citet{RAR-II} showed that the RAR model is able to explain this relation for typical dwarf galaxies to normal ellipticals. Here, we extend the results with predictions inferred from disk galaxies of the valid SPARC sample. The results are summarized in \cref{fig:SPARC:Ferrarese}.

% two groups: surface and boundary masses
Following the two definitions of enclosed mass ($M_b$ and $M_c$, see above) we focus on the \textit{appropriate} and \textit{inappropriate} groups, excluding flat tail fits and outliers, to demonstrate the natural benefits of the RAR model. Especially the importance of the cutoff parameter $W_0$. The first group (green circles) corresponds to galaxies where the total mass is given by the surface mass, $M_{\rm tot} = M_s < 10^{14} M_\odot$. The second group (light blue crosses) corresponds to galaxies where the total masses given at the boundary radius, $M_{\rm tot} = M_b \ll M_s$.

% group division criteria
The arbitrary upper mass limit $10^{14} M_\odot$ is applied to distinguish between realistic surface masses ($M_s < 10^{14} M_\odot$) and non-realistic ($M_s \geq 10^{14} M_\odot$). Clearly, spirals with masses above $10^{13} M_\odot$ are already highly improbable. And indeed, the prediction of the RAR model tells that SPARC spirals with appropriate halo information have natural total masses mainly below $10^{12} M_\odot$. Only one candidate has few $10^{12} M_\odot$. 

% missleading correlation
Note that due to inappropriate information in the outer halo of the second group it is not possible to constrain the cutoff parameter $W_0$ what results in extended mass distributions with total masses $M_s \gg 10^{14} M_\odot$, see also \cref{fig:NGC6015:deep-chi2}. Thus, it is necessary to truncate those extended mass distribution what results in a misleading narrow correlation in the $M_c - M_{\rm tot}$ relation. Instead, the RAR model predicts a much higher diversity rather than a narrow correlation according to the first group (green circles).

% mass ranges
The majority of simulated galaxies has a total dark matter mass between $10^{9} M_\odot$ and $10^{12} M_\odot$ while only a few are slightly more massive. The core mass spans a majority range between $10^{4} M_\odot$ and few $10^{6} M_\odot$. More important, the RAR model predicts here a break in the Ferrarese relation ($M_{\rm tot} \sim M_c^{0.6}$) at about $10^{11} M_\odot$, following $M_{\rm tot} \sim M_c^2$.

% recommendation
We want to emphasize that the second galaxy population (light blue crosses) has to be taken with caution since they don't provide appropriate information about the outer halo. We therefore recommend to rely mainly on the first galaxy population (green circles) with appropriate halo information.

\loadfigure{figure/Donato}

\subsubsection{DM surface density relation}
Finally, we turn to the constant surface density \citep{2009MNRAS.397.1169D} \begin{equation}
	\Sigma_{0D} = \rho_{0D} r_0 \approx 140_{-50}^{+80} M_\odot/{\rm pc}^2
\end{equation} This value is valid for about 14 orders of magnitude in absolute magnitude ($M_B$) where $\rho_{0D}$ and $r_0$ are the \textit{central} dark matter halo density at the one-halo-scale-length of the Burkert profile.

% parameter correspondense
Note that the \textit{center} in the Burkert model corresponds to the plateau in the RAR model, $\rho_{0D} \approx \rho_{\rm pl}$ where $\rho_{\rm pl}$ is defined at the first minima in the rotation curve. The relation between both one-halo scale lengths is $r_0\approx 2/3\, r_h$. We thus calculate the product $2/3\,\rho_{\rm pl}\,r_h$ for each galaxy.

% absolute magnitude (from CGS)
The absolute magnitude was taken from the Carnegie-Irvine Galaxy Survey, providing eight overlapping galaxies with the SPARC sample. These candidates are very well in agreement with the DM surface density observations, see \cref{fig:SPARC:Donato}.

% full sample
The central density results of the full galaxy sample (appropriate and inappropriate) is given as a histogram. Thus, the RAR model predicts for all valid candidates \textit{central} surface densities within the $3\sigma$ uncertainty. It is worth to note that the spread is relatively high, although not in conflict with observables. This phenomena may be explained by the diversity, similar to the spread in the $M_c$-$M_{\rm tot}$ relation.

% majority
Considering only galaxies with appropriate halo information (green bars), we find that the majority is closer to the mean value of about $140 M_\odot$~pc$^{-2}$, compared to the full sample (dark gray bars).

