%%%%%%%%%%%%%%%%%%%%%%%%%%%%%%%%%%%%%%%%%%%%%%%%%%%%
%%%%%%%%%%%%%%%%%%%%%%%%%%%%%%%%%%%%%%%%%%%%%%%%%%%%
\subsection{DM surface density relation}
\label{sec:dark-matter-surface-density}
%%%%%%%%%%%%%%%%%%%%%%%%%%%%%%%%%%%%%%%%%%%%%%%%%%%%
%%%%%%%%%%%%%%%%%%%%%%%%%%%%%%%%%%%%%%%%%%%%%%%%%%%%

The constant surface density \citep{2009MNRAS.397.1169D} 
%
\begin{equation}
\label{eqn:Donato}
	\Sigma_{0D} = \rho_{\rm 0D} r_0 \approx 140_{-50}^{+80} \si{\Msun\per\parsec^2}.
\end{equation} 
%
is valid for about 14 orders of magnitude in absolute magnitude ($M_B$) where $\rho_0$ is the \textit{central} DM halo density and $r_0$ the one-halo-scale-length --- both of the Burkert model. At $r_0$ the density falls to one-forth of the central density, i.e. $\rho(r_0) = \rho_{\rm 0D}/4$.

% parameter correspondence
Note that the \textit{center} in the Burkert model corresponds to the plateau in the fermionic DM model, i.e. $\rho_{\rm 0D} \approx \rho_{\rm p}$ where the plateau density $\rho_{\rm p}$ is defined at the first minimum in the RC. Following the definition of the Burkert radius $r_0$, we identify the one-halo-scale-length $r_B$ of the fermionic model such that $\rho(r_B) = \rho_p/4$. We thus calculate the product $\rho_p r_B$ for each galaxy.

% absolute magnitude (from CGS)
The absolute magnitude $M_B$ was taken from the Carnegie-Irvine Galaxy Survey \citep{2011ApJS..197...21H}, providing nine overlapping galaxies with the SPARC sample. These candidates are well in agreement with the DM surface density observations, see \cref{fig:SPARC:Donato}. The shown candidates include isothermal-like (blue outlined points) and non-isothermal (green circles) solutions. For comparison, the results are amended by the MW solution following the fermionic RAR model \citep{2018PDU....21...82A}.

\loadfigure{figure/Donato}

% histogram
Although absolute magnitude information is incomplete, all of the predicted DM surface densities are within the range of the $3\sigma$ area as well. This is visualised by a histogram for the full sample (dark grey bars, 120 galaxies) with comparison to the sub-sample (green bars, 44 galaxies) including non-isothermal solutions (i.e. $W_p < 10$). Considering the sub-sample only, we obtain a mean surface density of about $\SI{148.5}{\Msun/\parsec^2}$, fully inside the 1-$\sigma$ uncertainty in \cref{eqn:Donato}.

% equivalent relation
The SDR given by \cref{eqn:Donato} is qualitatively consistent with the scaling relation $M_h \sim r_h^2$ as given by \cref{eqn:rel:Mh-rh}. Due to the different halo profiles of the fermionic model, ranging from polytropes of $n=5/2$ to isothermal (see section \ref{sec:morph}), there is a non-linear relation between the halo radius $r_h$ and the one-halo-scale-length $r_B$. Nevertheless, considering that the halo is nearly homogeneous up to approximately the halo radius $r_h$ (i.e. $M_h \sim \rho_p r_h^3$), and that $r_h \sim r_B$ we obtain $\rho_p r_B \approx \const$. (see also \citealp{2019PDU....24..278A}).
