%%%%%%%%%%%%%%%%%%%%%%%%%%%%%%%%%%%%%%%%%%%%%%%%%%%%
\subsection{Fitting priors and Monte-Carlo approach}
\label{boundaryC}
%%%%%%%%%%%%%%%%%%%%%%%%%%%%%%%%%%%%%%%%%%%%%%%%%%%%

For the fermionic DM model we fix the particle mass $m$ and therefore reduce the number of free parameters by one, e.g. $\vec p = (\beta_0, \theta_0, W_0)$. A particle mass of $mc^2 = \SI{50}{\kilo\eV}$ is well motivated by the promising results obtained in \citet{2018PDU....21...82A,2020A&A...641A..34B,2021MNRAS.505L..64B,2022MNRAS.511L..35A}, where the fermionic core-halo DM profile was able to explain both the S-stars orbits around SgrA*, and the Milky Way rotation curve. In \citet{2019PDU....24..278A,2021MNRAS.502.4227A}, in particular, and for the same particle mass, the fermionic core-halo profiles were successfully applied to other galaxy types from dwarf to larger galaxy types, providing a possible explanation for the nature of the intermediate mass BHs, as well as a possible mechanism for SMBH formation in active galaxies. Moreover, as demonstrated in \citet{2018PDU....21...82A} there exists a particle mass range between $\sim$ 50 and $\sim$ 350 keV where the compacity of the central core can increase (all the way to its critical value of collapse) while maintaining the same DM halo-shape. Therefore, regarding the SPARC RC fitting, as well as all the scaling relations on halos-scales, our conclusions are not biased by the choice of the particle mass in the above range.

For the fermionic model we consider solutions which are either in the dilute regime ($\theta_0 \ll -1$, i.e. are King-like), or which have developed a degenerate core (i.e. $\theta_0 > 10$) at the center of such a halo. Fermionic solutions within only these two families have been shown to be thermodynamically and dynamically stable when applied to galaxies \citep{2021MNRAS.502.4227A}.

The NFW and the Burkert models are described by two free scaling parameters, e.g. $\vec p = (R_{\rm N},\rho_{\rm N})$. The DC14 model with e.g. $\vec p = (X, R_{\rm N}, \rho_{\rm N})$ and the Einasto model with e.g. $\vec p = (\alpha, R_{\rm N}, \rho_{\rm N})$ are described by three free parameters. Compared to NFW and Burkert, both (Einasto and DC14) have an additional parameter which affects the sharpness of the transition from the inner to the outer halo.

In order to find the best-fits we use the LM algorithm (see section \ref{LM-fitting}) with well chosen initial values (i.e. priors) reflecting astrophysical (realistic) scenarios. Because the LM algorithm finds only local minima, we choose the parameter sets randomly within a range and follow a Monte-Carlo approach. For the fermionic \textit{core}-\textit{halo} solutions, we choose $\beta_0 \in [10^{-8},10^{-5}]$, $\theta_0\in [25,45]$ and $W_0\in[40,200]$ which correspond to a conservatively wide range of parameters according to \citet{2019PDU....24..278A}. For the fermionic \textit{diluted} solutions, we cover the same range for $\beta_0$ and $W_0$, but with $\theta_0\equiv\theta_p \in [-40,-20]$. For the other DM models, the initial scaling factors are chosen from $R_{\rm N}\in[10^{1}, 10^{4}]\si{\parsec}$ and $\rho_{\rm N} \in [10^{-4}, 10^{-1}]\si{\Msun\per\parsec^3}$. According to \citet{2017MNRAS.466.1648K} the additional parameter of the DC14 model is chosen from $X\in[-3.75,-1.3]$. For the Einasto model, the exponent is chosen from $\alpha\in[0.1, 10]$. This relatively large window has been chosen in order (i) to account for the broad diversity of rotation curves covered by the SPARC data, and (ii) not to be limited by any fixed value (e.g. as typically obtained by CDM-only simulations) since this is an independent analysis to that of N-body simulations, and may also account for other effects such as baryonic feedback (see also the discussion in next subsection).
