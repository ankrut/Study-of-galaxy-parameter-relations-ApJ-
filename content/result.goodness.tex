%%%%%%%%%%%%%%%%%%%%%%%%%%%%%%%%%%%%%%%%%%%%%%%%%%%%
\subsection{Goodness of model}
\label{sec:result:gof}
%%%%%%%%%%%%%%%%%%%%%%%%%%%%%%%%%%%%%%%%%%%%%%%%%%%%

The SPARC galaxies \add{show different characteristics in their rotation curve} such as a nearly flat curve through the entire galaxy data; a rising trend in the inner halo followed by a single maximum; or multiple extrema in the form of oscillations. \add{See \cref{fig:benchmark:total-rotation-curves} for three typical examples within the SPARC data-set.} Some galaxies show just a rising trend implying that the rotation curves are incomplete, likely due to the faintness and/or lack of data for outermost halo stars. Of interest is therefore a quantitative description about the goodness of a DM halo model fitting the entire galaxy sample (120 galaxies).

The goodness of a fit for a single galaxy is well described by the $\chi^2$ value, see \cref{eqn:method:chi-square}. When competing models with different number of parameters are compared it is appropriate to consider the reduced $\chi^2$ defined as $\chi_r^2 = \chi^2/d$ with the degree of freedom $d = N-p$, $N$ being the number of observables (for a single galaxy) and $p$ the number of parameters (of the considered model).

The question now arises how to compare the competing models for a population of galaxies. In order to find the goodness of a model which is robust against outliers, we ask \textit{how many} fitted galaxies have a (reduced) $\chi^2$ \textit{lower} than a given one. It turns out that the population curve \add{resembling a cumulative distribution function (CDF)} follows nearly a log-normal distribution. We use the mean value, labelled as $\hat \chi^2_r$, as the criteria to described the goodness of a model for fitting a population of galaxies within the SPARC data-set. \add{A parameter analysis of the best-fit solutions for each DM halo model is detailed in appendix \ref{sec:appendix:parameter-distribution}. The goodness analysis is} shown in \cref{fig:goodness:all} and can be summarized as:

\loadfigure{figure/GoodnessAll}
\loadfigure{figure/GoodnessWithCutoff}

\begin{asparaenum}[(i)]
    \item The NFW model ($p=2$) is statistically disfavoured with respect to the other DM halo models.
    %
    \item The Einasto model ($p=3$) and the DC14 model ($p=3$), on the other hand, are statistically favoured with similarly good results.
    
    We remind that the DC14 model is based on the analysis of hydrodynamically simulated galaxies including complex baryonic feedback processes \citep{2014MNRAS.441.2986D}. The good performance of DC14 may indicate the importance of baryonic feedback in galaxy formation. \add{Indeed, our results regarding DC14 are in line as well with the literature since, as can be seen from \cref{fig:parameter-distribution:dc14} (bottom panel), the bulk of our ($\alpha,\beta,\gamma$) parameters for the SPARC data-set lies within the windows (0,2.6); (2.3,4); (0,2) in rough agreement with \citet{2014MNRAS.441.2986D}. Interestingly, we obtain a mean for the stellar to DM mass ratio $X = \log_{10}(M_*/M_\mathrm{halo})$ of $-2.4$, which is within the $X$ values ($-2.5,-2.3$) where DM cusps are most effectively flattened due to baryonic effects as reported in \citet{2014MNRAS.441.2986D}. 
    
    However, the interpretation regarding the relevance (or not) of baryonic effects for Einasto profiles is more subtle. Results from hydrodynamical (zoom-in) simulations obtained for $\SI{E10}{\Msun}$ halos, show a reduction on inner-halo densities (i.e. at the convergence radius $r=\SI{0.26}{\kilo\parsec}$) of up to $\SI{45}{\percent}$ in WDM cosmologies with respect to the analogous DM-only simulations \citep{2019MNRAS.483.4086B}. A reduction which, when calibrated through Einasto profiles, is already a $15-20 \%$ more pronounced than the one typically obtained for the same WDM halos respect to the CDM ones in DM-only cosmological simulations \citep{2019MNRAS.483.4086B}. Now, for SPARC halos with total mass of few $\SI{E10}{\Msun}$ and with roughly the same Einasto scale-radius our results show that the Einasto $\alpha$ parameter required to produce the above $\sim\SI{45}{\percent}$ density reduction at $\SI{0.26}{\kilo\parsec}$ is $\alpha\approx 0.3$ (which is close to the mean value $\alpha \approx 0.4$ as shown in \cref{fig:parameter-distribution:einasto}, and only one dex above the standard $0.2$ of CDM). However it is important to mention that for the mean $\alpha$ value, the corresponding inner-halo density drop is so pronounced (i.e the inner-slope is almost flat, see \cref{fig:profile-illustration-mep}), that the halos look more like fermionic profiles, indicating instead a possible relevance of an underlying MEP mechanism for halo formation.}
    
    %However, note also that the Einasto model can be motivated by CDM-only simulations without any baryonic feedback \citep{2004MNRAS.349.1039N}, still producing comparable (or even better) results. Thus, keeping the underlying physics of each DM model in mind (with special attention to the fermionic model arising from a MEP), it is therefore questionable how much \textit{weight} one should give to baryonic feedback processes regarding the halo RC fits.
%
    \item The Burkert model ($p=2$) and the fermionic model ($p=3$), both produce comparable results but are somewhat statistically less favoured when the entire SPARC sample (120 galaxies) is considered. However, in the fermionic scenario the picture changes considerably for the sub-sample (44 galaxies) where data supports for a significant escape of DM particles, that is for energy-cutoff values at plateau of $W_p < 10$. %$(see e.g. first column in \cref{fig:benchmark:dark-matter-profiles}), implying a steeper trend in the halo tail with respect to an isothermal-like one (as given in third column of the same figure). 
For that sub-sample, the NFW model is statistically even more disfavoured while all other models are comparable, though with a little tendency for Einasto and against Burkert, see \cref{fig:goodness:with-cutoff}.
\end{asparaenum}

We would like also to point out that many galaxies in the SPARC data set are missing significant information in the outer halo (e.g. due to faint stars) or show a complex behavior \add{(oscillatory-pattern) in their rotation curves (see e.g. right box in \cref{fig:benchmark:total-rotation-curves})}. In any case, it does not allow to univocally determine the cutoff parameter (i.e. $W_p$) for the fermionic DM model since any sufficiently large $W_p$ would not change the $\chi^2$ value (see e.g. bottom left panel of \cref{fig:chi-analysis}). Nevertheless, there are some other individual galaxies where the escaping particles effects are clearly preferred. Interestingly, all of those galaxies are of magellanic type: NGC0055 (Sm), UGC05986 (Sm), UGC05750 (Sdm), UGC05005 (Im), F565-V2, (Im), UGC06399 (Sm), UGC10310 (Sm), UGC07559 (Im), UGC07690 (Im), UGC05918 (Im) and UGC05414 (Im).

Moreover, many galaxies, which are poorly fitted by any of the considered models, show \textit{short range} oscillations \add{in their rotation curves} with more than one maximum. None of the models can provide a clear explanation of that phenomena, found usually in non-magellanic galaxy types: e.g. NGC2403 (Scd), UGC02953 (Sab), NGC6015 (Scd), UGC09133 (Sab), UGC06787 (Sab), UGC11914 (Sab), NGC1003 (Scd), NGC0247 (Sd), UGC08699 (Sab) and UGC03205 (Sab).

On phenomenological grounds, in the fermionic DM model it is possible to vary the width of the maximum bump in the RC through the cutoff parameter in the strong or moderate cutoff regime ($W_p \lesssim 10$). Whether with weak ($W_p \gtrsim 10$) or even without cutoff-effects, \add{the RC} solutions of the model show long range oscillations, similar to the isothermal model. In any case, these \add{RC} oscillations have a too long wavelength and therefore do not offer a convenient explanation. On the other hand, in the case of strong cutoff, we obtain a narrow maximum bump necessary for many RCs, especially for galaxies of magellanic type (see above for examples), which usually do not show those oscillations, but also for some non-magellanic galaxy types, e.g. NGC5585 (Sd), NGC7793 (Sd), UGC06614 (Sa), ESO079-G014 (Sbc), F571-8 (Sc), NGC0891 (Sb), UGC06614 (Sa), UGC09037 (Scd), NGC4217 (Sb), UGC04278 (Sd).

NFW and Burkert models cannot explain variations of the inner and outer rotation curve because the parameters ($\beta$ and $\gamma$) responsible for such a behaviour (see \cref{eqn:hernquist}) are fixed. Additionally a transition from the inner to the outer halo is generally characterized by $\alpha$. In contrast to NFW and Burkert, the DC14 and Einasto models have a free parameter which affects the inner/outer rotation curve steepness and the sharpness of the halo transition, simultaneously. Such a flexibility is reflected in generally better $\chi^2$ values. Nevertheless, the goodness for oscillating RCs remains rather poor.