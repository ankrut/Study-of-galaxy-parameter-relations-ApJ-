\section{Summary and conclusion}
\label{sec:conclusion}
We faced the intricate relation between the mass distribution of dark and baryonic matter for disk galaxies from a different perspective compared to the cosmological $\Lambda$CDM principles and MOND. Thus, we considered dark matter as a self-gravitating system composed of elementary fermions while the baryonic component was provided from the SPARC data base.

\subsubsection*{Acceleration correlation}
The radial acceleration correlation \citep{2016arXiv160905917M} as well as the equivalent mass discrepancy acceleration relation \citep{2004ApJ...609..652M,2014Galax...2..601M} show clearly a link between the acceleration due to the baryonic matter and the acceleration due to dark matter. In the $\SYMabar$-$\SYMaDM$ projection we found a linear relation between those two acceleration components rather than the proposed empirical fit based on the $\SYMabar$-$\SYMatot$ projection.

% relation reproduction
That RAR model is able to reproduce those acceleration correlations together with the their scatter. For comparison, we considered also two more DM models (NFW, DC14) which reproduce the empirical correlation qualitatively as well. In sum, we conclude that all considered DM models reproduce the correlation based on the average of many spiral galaxies of different Hubble type.

% deviations in individual galaxies
Nevertheless, focusing on individual galaxies we may extract more detailed information. Thus, we noticed fundamental deviations compared to the proposed radial acceleration correlation by McGaugh. Based on the representative NGC0055 it shows that many galaxies (of the SPARC database) don't follow strictly the given empirical formula. Despite an offset, NGC055 has a similar behavior to \cref{egn:mcgaugh-fit} in the dark matter dominated region but deviates strongly in the transition into the baryonic matter dominated region with an abrupt decrease of the DM acceleration.

% deviations in MW
The deviations become more clear in the analysis of the much better resolved Milky Way. Here, we found a linear proportionality in the very low acceleration regime ($\lesssim 10^{-12} \mathrm{m/s^2})$ in contrast to the empirical radial acceleration correlation. This deviation is due to the necessary strong cutoff.

% cutoff importance
Indeed, compared to the analysis of SPARC galaxies we know that spirals have very different halos regarding the cutoff. Those different cutoffs correspond to different slops of the radial acceleration correlation in the dark matter dominated regime. According to the RAR model we were able to identify several SPARC galaxies with a strong cutoff similar to Milky Way. But those galaxies (among many other) have a lack of information in the very outer halo (e.g. due to faint stars) what keeps the behavior in the very low acceleration regime ($\lesssim 10^{-12} \mathrm{m/s^2}$) in secret.

% importance of inner halo structure
Moving to the high acceleration regime, representing the baryonic matter dominated region, the prototype NGC0055 demonstrates clearly that the inner halo structure is important when the relevance of a cuspy or cored halo is considered. Thus, the new perspective of the radial acceleration correlation ($\SYMabar-\SYMaDM$) is preferred compared to the original representation ($\SYMabar-\SYMaobs$) which obscures the inner halo relation.

% relevance of DM core
Finally, in the very high acceleration regime we predict an increase of the centripetal acceleration due to the quantum core in the galactic center, a fundamental feature of the RAR model. This is in clear contrast to the empirical correlation and the NFW model.

% RAR model is great
In sum, the unveiled deviations in the low and high acceleration regimes are a satisfying explanation for the large scatter in the radial acceleration correlation. The conclusion of the analysis is that the acceleration correlation of individual galaxies are more complicated than the empirical radial acceleration correlation suggest and that the RAR model is able to explain this phenomena based on fundamental physical principles.

\subsubsection*{Goodness of fit}
% goodness of model analysis
The more profound understanding of the radial acceleration correlation is backed by a goodness of model analysis for 124 filtered galaxies of the SPARC data covering different Hubble types. We found that RAR and DC14 are similarly good while NFW is clearly disfavored here. This picture becomes even more clear when only magellanic galaxies are considered. On the contrary, non-megallanic galaxies are equally bad fitted with all models. This result implies a link between dark matter and the morphological type. Clearly, the type of the inner halo (cored or cuspy) may play an important role.

% special attention to oscillations
Spacial attention has to be given to galaxies with an oscillation in the rotation curve. All considered models (RAR as well as NFW and DC14) fail in fitting those rotation curves what appears mostly for non-magellanic types. This is not surprising, since all models have only one maxima bump by design. The oscillations of the RAR model with weak cutoff (or no cutoff in the limit) are too long ranged for a convenient explanation.

% relevance of underlying physics
Comparing the RAR model with DC14 yields another interesting hint regarding baryonic-to-dark relations. We may consider DC14, a baryonic feedback motivated model, as an extension of the NFW model. Thus, comparing DC14 and NFW implies that baryonic feedback mechanism is important in galaxy formation. But our RAR model fits the SPARC sample as good as DC14 although it doesn't consider any baryonic matter contribution nor any baryonic feedback mechanism. This leads to the speculation that baryonic feedback might be not so important for disk galaxies. The more probable scenario is that the RAR model enhanced with baryonic feedback mechanism would improve galaxy fitting significantly.

% RAR challange
On the other hand, the RAR model is now confronted with the question why dark and baryonic matter distributions arrange in the way they obviously do. That relation seems to be encoded in the acceleration relation, its large scatter and, in particular, the \textit{scatter shower} in the low acceleration regime of the dark matter component. The change of perspective from $\SYMabar$-$\SYMaobs$ (original) to $\SYMabar$-$\SYMaDM$ unveils the acceleration correlation on a wider spatial range and is therefore a good step forward. Since the RAR model considers only dark matter without any baryonic contribution nor feedback we are going to answer that fundamental question in a future work.

\subsubsection*{Universal relations}
% core-halo relation (Ferrarese)
Regarding dark-to-dark relations, we enriched the $M_c$-$M_{\rm tot}$ relation \citep{RAR-II} with prediction for disk galaxies of the SPARC data base. The majority of the galaxies has a core mass between $10^{4} M_\odot$ and few $10^{6} M_\odot$ while the total masses span a range from $10^{9} M_\odot$ to few $10^{12} M_\odot$. An important outcome of the results, compared to the $M_{\rm BH}$-$M_{\rm tot}$ relation \citep{2002ApJ...578...90F}, is a break in the relation at about $10^{11} M_\odot$, the bottom edge of the observable \textit{Ferrarese window}. Of great interest would be therefore the extension of the relation down to dwarf galaxies what is under current investigation in our group.

% diversity and uncertainty
We want to emphasize that the thinking of a narrow $M_c$-$M_{\rm tot}$ relation as suggested by \citet{2002ApJ...578...90F} might be misleading according the the predictability analysis of the RAR model, see also \citet{RAR-II}. In particular, the model predicts a much higher diversity in the $M_c$-$M_{\rm tot}$ population what is supported by the SPARC results. On the other hand, the diversity is due to some uncertainty in the parameter space (e.g. the cutoff parameter $W_0$). To narrow those uncertainties better observations in the inner halo are needed.

% constant surface density (Donato)
The \textit{constancy} of the surface density \citep{2009MNRAS.397.1169D} for SPARC galaxies is also given within the RAR context. From the Carnegie-Irvine Galaxy Survey we have extracted the absolute magnitude for eight overlapping galaxies. The surface density predictions of that sub-sample are very well in agreement with observations. It is important to emphasize that the predicted surface density range, covering the full sample, remains within the $3\sigma$ uncertainty.

\subsubsection*{Future perspective}
% how to deal with oscillations (information filter)
We remind to interpret the results from isothermal galaxies, corresponding to $M_{\rm tot} = M_b \ll M_s$, with caution. Compared to the the remaining galaxies they have either a lack of information or short ranged oscillations in the outer halo what prohibits to constrain the cutoff parameter with certainty. The lack of information may be simply solved with future surveys while the oscillations, what may be regarded as an abundance of information, requires further investigations. A truncation (information filter) shortly after the first maxima in the rotation curve would certainly allow to compare with other galaxies, which provide only one maxima bump at best. We expect that such analysis will unveil more profound connections between the dark matter and the morphological type. 

% how to deal with oscillations (multi-component)
Although, the very different oscillations may be due to ongoing relaxation processes it might be worth to consider also multi-component halo models. It is probable that the RAR model enhanced with baryonic contributions will help in this analysis and thus to understand galactic structures.

% dynamical classification
Further, \citet{2017MNRAS.469.2539K} introduced recently a new dynamical classification of galaxies based on circular velocity curves. Such a classification based on the inferred dark matter rotation curves would be an interesting alternative to the classical morphological classification, which seems to be linked to dark matter profiles.