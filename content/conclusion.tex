%%%%%%%%%%%%%%%%%%%%%%%%%%%%%%%%%%%%%%%%%%%%%%%%%%%%
%%%%%%%%%%%%%%%%%%%%%%%%%%%%%%%%%%%%%%%%%%%%%%%%%%%%
\section{Summary and conclusion}
\label{sec:conclusion}
%%%%%%%%%%%%%%%%%%%%%%%%%%%%%%%%%%%%%%%%%%%%%%%%%%%%
%%%%%%%%%%%%%%%%%%%%%%%%%%%%%%%%%%%%%%%%%%%%%%%%%%%%

\add{For the case of disk galaxies, as provided by the SPARC data-set (see section \ref{sec:data}), we have studied the galactic rotation curves and different galaxy scaling relations --- such as the Radial Acceleration Relation, MDAR and DM surface density relation (SDR) --- from an alternative perspective in which the halos are formed through a MEP}.

Within this paradigm we considered the DM halo as a self-gravitating system of neutral fermions at finite temperature while the baryonic mass components were provided from the SPARC data-set. 

For comparison, we have taken into account empirical DM fitting models motivated from DM-only simulations like NFW (within CDM) and Burkert (within WDM); the DC14 (or generalized NFW) model which contains different physics such as the influence of baryonic feedback in the morphology of CDM halos; and the Einasto model as recently studied in \citet{2019MNRAS.483.4086B} accounting for baryonic effects (through hydrodynamical zoom-in simulations) in either cosmology. Finally, their best-fits to the acceleration relations and SPARC RCs were compared with the fermionic model (see sections \ref{sec:result:ac} and \ref{sec:result:gof} respectively).

For all competing DM models, we fitted the DM contribution to the RC as inferred from the given total rotation curve and the baryonic component (see section \ref{LM-fitting}).

An alternative fitting approach is minimizing the least square errors of the total rotation curve (e.g. $V_i = V_{i, \rm tot}$) where the predictions are a composition of a theoretical DM halo model and the baryonic component inferred from observation. On theoretical ground, such an approach is not identical to the approach explained in section \ref{sec:fitting} since the propagation of uncertainty produces a somewhat different weighting. However, we have repeated the same analysis on both approaches and obtained consistent results despite few minor numerical variations, and without changing any qualitative conclusion obtained in this work.

The main results of this work can be summarized as follows, according to three different issues.

\subsection{Acceleration relations}
The Radial Acceleration Relation as well as MDAR analyzed here are based on an averaging of many spiral galaxies and hold for different Hubble types. Our analysis shows that all competing DM models are able to reproduce those relations, although without a clear favourite because all are similarly good (see \cref{fig:acceleration:grid}). For instance, for all DM models, we obtain nearly identical values for $\SYMafrak$ as required in \cref{eq:mcgaugh-fit}. This result is in line with a recent analysis done in \citet{2020JCAP...06..027K}. 

\subsection{Individual rotation curve fittings}
A deeper understanding of the Radial Acceleration Relation and MDAR is backed by a goodness of model analysis for $120$ filtered and individual galaxies of the SPARC data set covering different Hubble types. The DM contribution to the RCs reflects some diversity in galaxies which, in general, are better fitted by cored DM halo models instead of cuspy (e.g. NFW, see \cref{fig:goodness:all}). This is not only in agreement with a similar analysis done by \citet{2020ApJS..247...31L}, but totally in line with the results of \citet{2020JCAP...06..027K}. That is, it has been shown here that the DM halo models suitable to explain the acceleration relations do not necessary explain well the SPARC RCs.

Comparing the fitting goodness of the superior DC14 with the inferior (and statistically disfavoured) NFW implies that baryonic feedback mechanism is important in galaxy formation. On the other hand, we found that the fermionic model, compared to DC14 or Einasto, is equally good in fitting galaxies which require a significant escape of particles (i.e. $W_p < 10$, see section \ref{sec:result:gof} and \cref{fig:goodness:with-cutoff}). \add{Those galaxies are characterized by a flat inner halo, justified by very different physical principles in the most favored DM profiles: DC14 and Einasto models rely on complex baryonic feedback processes, while the MEP scenario involves a quasi-thermodynamic equilibrium of the DM particles.} This may imply that for those galaxies baryonic feedback is less relevant, thus hinting on the importance of a quasi-thermodynamic equilibrium that may be reached in those DM halos.

\subsection{Fermionic halos}
We found that for SPARC galaxies the \textit{constancy} of SDR, originally based on the Burkert model \citep{2009MNRAS.397.1169D}, is also achievable within the fermionic DM model. From the Carnegie-Irvine Galaxy Survey \citep{2011ApJS..197...21H} we have further extracted the absolute magnitude for nine overlapping SPARC galaxies. The surface density predictions of that sub-sample are well in agreement with observations.

\add{Additionally we demonstrated that particular solutions from the rich morphology of fermionic DM on halo scales can be associated with different empirical DM models, depending on the strength of particle evaporation (described by $W_p$). Fermionic DM halos are polytropic-like (with $n = 5/2$) for strong evaporation ($W_p \ll 1)$,  transition into profiles similar to Burkert for values of $W_p$ close to the point of marginal thermodynamical stability (given at $W_p^{(c)} = 7.45$) and finally develop isothermal tails for negligible evaporation ($W_p \gg 10$).}

Interestingly, the mean $\kappa$ value obtained here for the Einasto model is roughly $0.4$, implying a pronounced inner-halo density drop (i.e leading to an almost flat inner-slope, see \cref{fig:profile-illustration-mep}), such that the halos look more like the fermionic profiles.

Indeed, following the work done in \cite{2021MNRAS.502.4227A} for such fermionic profiles, it can be found that typical galaxies belonging to this sub-group (with $W_p \ll 1$) are thermodynamically and dynamically stable, with an outer-halo morphology of polytropic nature (see \cref{fig:halo-profiles}). This may be evidencing a fundamental and deep link between thermodynamics of self-gravitating fermions and galaxy formation and morphology.

