%%%%%%%%%%%%%%%%%%%%%%%%%%%%%%%%%%%%%%%%%%%%%%%%%%%%
%%%%%%%%%%%%%%%%%%%%%%%%%%%%%%%%%%%%%%%%%%%%%%%%%%%%
\section{Summary and conclusion}
\label{sec:conclusion}
%%%%%%%%%%%%%%%%%%%%%%%%%%%%%%%%%%%%%%%%%%%%%%%%%%%%
%%%%%%%%%%%%%%%%%%%%%%%%%%%%%%%%%%%%%%%%%%%%%%%%%%%%

For the case of disk galaxies, we have studied both the RCs and different galaxy scaling relations (such as the Radial Acceleration Relation, MDAR, DM surface density Relation and the Ferrarese Relation connecting the total halo mass and its supermassive central object) within the SPARC data-set, from an alternative perspective in which the halos are formed through a MEP. Within this paradigm we considered the DM halo as a self-gravitating system of neutral fermions at finite temperature while the baryonic mass components were provided from the SPARC data-set. 

For comparison, we have also considered the DC14 (or generalized NFW) model which contains different physics such as the influence of baryonic feedback in the morphology of CDM halos. In addition, we have taken into account empirical DM fitting models motivated either from DM-only simulations like NFW (within CDM), and Burkert (within WDM); or the Einasto model as recently studied in \citet{2019MNRAS.483.4086B} accounting for baryonic effects (through hydrodynamical zoom-in simulations) in either cosmology. Finally their best-fits to the acceleration relations and SPARC RCs were compared with the fermionic model in sections \ref{sec:result:ac} and \ref{sec:result:gof} respectively.

For all competing DM models, we fitted the DM contribution to the RC as inferred from the given total rotation curve and the baryonic components. An alternative fitting approach is minimizing the least square errors of the total rotation curve (e.g. $V_i = V_{i, \rm tot}$) where the predictions are a composition of a theoretical DM halo model and the baryonic component inferred from observation. On theoretical ground, such an approach is not identical to the approach explained in section \ref{sec:fitting} since the propagation of uncertainty produces a somewhat different weighting. However, we have repeated the same analysis on both approaches and obtained consistent results despite few minor numerical variations, and without changing any qualitative conclusion obtained in this work.

The main results obtained in this paper can be summarized as follows, according to three different issues.

\subsection{Acceleration relations}
The Radial Acceleration Relation as well as the MDAR relation analyzed here are based on an averaging of many spiral galaxies and holds for different Hubble types. Our analysis shows that all competing DM models are able to reproduce those relations, although without a clear favourite because all are similarly good (see \cref{fig:acceleration:grid}). For instance, for all DM models, we obtain nearly identical values for $\SYMafrak$ as required in \cref{eq:mcgaugh-fit}. This result is in line with a recent analysis done in \citet{2020JCAP...06..027K}. 

\subsection{Individual rotation curve fittings}
A deeper understanding of the Radial Acceleration Relation and MDAR is backed by a goodness of model analysis for $120$ filtered and individual galaxies of the SPARC data set covering different Hubble types. The DM contribution to the RCs reflect some diversity in galaxies which, in general, are better fitted by cored DM halo models instead of cuspy (e.g. NFW, see Fig. \ref{fig:goodness:all}). This is not only in agreement with a similar analysis done by \citet{2020ApJS..247...31L}, but totally in line with the results of \citet{2020JCAP...06..027K}. That is, it has been shown here that the DM halo models suitable to explain the acceleration relations do not necessary explain well the SPARC RCs.

Comparing the fitting goodness of DC14 and NFW implies that baryonic feedback mechanism is important in galaxy formation (\add{i.e. NFW is statistically disfavoured, see \cref{fig:goodness:all,fig:goodness:with-cutoff}}). On the other hand, comparing the fermionic model with DC14 or Einasto yields an interesting hint regarding baryonic-to-dark relations: we found that the fermionic model is equally good in fitting galaxies which require a significant escape of particles (i.e. $W_p < 10$) (see section \ref{sec:result:gof} and bottom panel in \cref{fig:goodness:with-cutoff}). This may imply that for those galaxies baryonic feedback is less relevant, thus hinting on the importance of a quasi-thermodynamic equilibrium that may be reached in those DM halos. Interestingly, the mean $\alpha$ value obtained here for Einasto models is roughly $0.4$, implying a pronounced inner-halo density drop (i.e leading to an almost flat inner-slope, see \cref{fig:profile-illustration-mep}), such that the halos look more like the fermionic profiles,  Indeed, following the work done in \cite{2021MNRAS.502.4227A} for such fermionic profiles, we have found that typical galaxies belonging to this sub-group (with $W_p \ll 1$) are found to be thermodynamically and dynamically stable, with an outer-halo morphology of polytropic nature (see fig. \ref{fig:halo-profiles}). This may be evidencing a fundamental and deep link between thermodynamics of self-gravitating fermions and galaxy formation and morphology. 
% indicating instead a possible relevance of an underlying MEP mechanism for halo formation.

\subsection{Other Universal relations}
We have analyzed in Appendix \ref{sec:parameter-correlations} the Ferrarese relation between the total halo mass and the mass of the supermassive central object. In particular, we have studied the $M_c$-$M_{\rm tot}$ relation --- as first studied in \citet{2019PDU....24..278A} within the fermionic \textit{core-halo} RAR profiles --- for typical dwarf to larger galaxy types and extended here for disk galaxies of the SPARC data set. Interestingly, for the sub-sample of galaxies where the fermionic model provides comparably good fits as Einasto or DC14 models (i.e. with $W_p < 10$), the fermionic core-halo solutions predict that the SPARC galaxies have a quantum core mass between $\SI{E4}{\Msun}$ and few $\SI{E6}{\Msun}$ with few reaching masses up to $\SI{E7}{\Msun}$. The total mass spans a range from $\SI{E9}{\Msun}$ to few $\SI{E12}{\Msun}$ as shown in \cref{fig:SPARC:Ferrarese}.

A notable prediction of the above fermionic sub-sample ($W_p < 10$), as compared to the Ferrarese relation \citep{2002ApJ...578...90F}, is that for typical SPARC galaxies having total masses $\sim \SIrange{E9}{E10}{\Msun}$ there seems to be no clear correlation between the core mass $M_c$ and the total DM mass $M_{\rm tot}$ since solutions show a wide spread in the mass plane (see white dots in \cref{fig:SPARC:Ferrarese}). Even if this result seems to be qualitatively in line with the one found observationally in \cite{2011Natur.469..377K} (i.e. for small enough galaxies with typical circular halo velocities of about \SIrange{50}{150}{\kilo\metre/\second}), it has to be taken with caution since there is much uncertainty (up to about two orders of magnitude) in the core mass $M_c$ per given halo mass (see e.g. middle top panel in \cref{fig:chi-analysis}). Only galaxies with dispersion velocity data down to the central regions allowing to infer the central object mass, can help to better constraint the core-halo solutions.

Polytropic-like solutions of the fermionic model ($W_p \ll 1$) provide a distinct connection between the halo and the core. Even with a lack of data in the central region, such a core-halo connection allows (up to some degree of uncertainty) to constrain the inferred core mass $M_c$. (see e.g. upper panel of \cref{fig:chi-analysis}). \add{This is a notable prediction of the fermionic model (not feasible for $\Lambda$CDM halos), which relies on the fact that fermionic halos can `born' (within a cosmological framework) in a core-halo quasi-stationary state with the fermion core not-yet collapsed and able to mimic the central BHs \citep{2019PDU....24..278A,2021MNRAS.502.4227A}.}
%The differentiation of the accretion process by a BH or a fermionic DM core of the baryonic matter which tend to accumulate in the central regions of galaxies is the subject of forthcoming publications.

Finally, the \textit{constancy} of the surface density \citep{2009MNRAS.397.1169D} for SPARC galaxies is also achievable within the fermionic DM model. From the Carnegie-Irvine Galaxy Survey \citep{2011ApJS..197...21H} we have extracted the absolute magnitude for nine overlapping SPARC galaxies. The surface density predictions of that sub-sample are well in agreement with observations.
