\subsection[Best-fit analysis]{$\chi^2$ analysis}
We are moving now to a more detailed $\chi^2$ analysis of three selected galaxies, each representing some characteristics of given observational data in relation to the RAR model fit. Thus, we mainly divide the SPARC galaxies in three group through the inferred dark matter component as explained in next.

The first group, represented by NGC0055, clearly shows only a single maxima in its dark matter rotation curve. As shown in \cref{fig:NGC0055:deep-chi2} this typical profile is well fitted by the RAR model with sufficient low cutoff values, corresponding to not negligible surface effects. However, due to the lack of information in the inner halo structures, especially the galactic center, there is some uncertainty in the strength of the cutoff parameter. Although those solutions provide a minimal DM halo mass, the uncertainty is physically better reflected in the core mass $M_c$, what covers a range of about two orders of magnitude. Higher core masses, given for lower $W_0$ values, imply cuspier halos what are clearly discarded here. Well disfavored are also higher DM halo masses $M_s$ which corresponding to isothermal-like halos, given for higher $W_0$ values. In contrast, those solutions provide a minimal core mass $M_c$ with a huge uncertainty in the total mass.

The second group, represented by DDO161, shows a rising part in the rotation curve towards a maximum without a clear turning point, compared to the first group. Fitting those galaxies for different $W_0$ values does not necessary favor solution with or without surface effects. The variation in the $\chi^2$ value remains rather small, see \cref{fig:DDO161:deep-chi2}. Nevertheless, clearly discarded are cuspy halos just as in the first group. There is a narrow $\chi^2$ minimum for relatively low $W_0$ values suggesting a best-fit. However, this result should be taken with caution because the obtained minimum depends on the inner data points, keeping in mind that for most galaxies in the SPARC data base the inner data points have a relatively high uncertainty. This is also the case for DDO161.

Finally, the third group, represented by NGC6015, shows some oscillations in the rotation curve, mainly in the outer halo. Following \cref{fig:NGC6015:deep-chi2}, those galaxies are clearly better fitted by extended isothermal-like halos compared to the contracted halos, given for lower $W_0$ values. The extended solutions provide a wide halo maximum followed by a flat rotation curve. This is suitable to fit the oscillations well on average, although the best-fit remains rather poor. The contracted solutions, in contrast, provide only a narrow maximum in the halo, followed by a Keplerian decreasing tail. The latter is clearly disfavored here. It is worth to note that the RAR model (and others such as NFW and DC14) are not appropriate to fit the oscillations, characterized through multiple maxima in the rotation curve.

In summary, we consider galaxies belonging to the first group as \textit{appropriate} candidates because they allow to determine the $W_0$ value, although with some uncertainty. In contrast, galaxies belonging to the other two groups we consider as \textit{inappropriate} candidates due to either a lack (e.g. DDO161) or an abundance (e.g. NGC6015) of information in the halo. This fact does not allow to determine the cutoff value, especially leaves the upper bound open. Though, in all cases it is possible to set a lower limit what clearly discards cuspy halos within the RAR model. Note that the lower limit in $W_0$ also sets a specific upper limit for the core mass.

\loadfigure{figure/chi2_NGC0055}
\loadfigure{figure/chi2_DDO161}
\loadfigure{figure/chi2_NGC6015}