\subsection{Goodness of model}
\label{sec:result:gof}
The SPARC galaxies show several phenomena in their rotation curves. Some galaxies indicate a clear cutoff in the outer halo and/or a variation of the inner halo steepness, giving hints of a cored or cuspy halo. Many galaxies show a rising tail implying that the rotation curves are incomplete. This is most probably due to faint stars in the outer most halo region. On the other hand, some galaxies are characterized by a clear oscillation in their flat rotation curve. Of interest is therefore a quantitative description about the goodness of a model fitting the entire galaxy sample.

\loadfigure{figure/goodness}

The goodness of a fit for a single galaxy is well described by the $\chi^2$ value. When competing models with different number of parameters are compared it is appropriate to consider the reduced $\chi^2$ defined as $\chi_r^2 = \chi^2/d$ with the degree of freedom $d = N-p$, $N$ the number of observables (for a single galaxy) and $p$ the number of parameters (of the considered model). The question now arises how to compare the competing models for a population of galaxies. Clearly, the sum of (reduced) $\chi^2$ values for a model would be an inappropriate indicator. Consider the case where a model fits the majority of a population very well (low $\chi^2$ values) but fails extremely (very high $\chi^2$) for just a few galaxies. Thus, the goodness of a model cannot depend on the whole population. Instead, we have to look differentiated at the population. Therefore, we ask \textit{how many} fitted galaxies have a (reduced) $\chi^2$ \textit{lower} than a given one. It turns out that the normalized population curve may be well described by the function \begin{equation}
	\label{eqn:population}
	\frac{N(\chi_r^2)}{N_{\rm max}} = \frac{1}{1 + \qbracket{\frac{\chi_r^2}{\hat\chi_r^2}}^{-n}}
\end{equation} The parameter $\hat \chi_r^2$ cuts the population in half where the first half has a lower reduced $\chi^2$ and the other half has a greater reduced $\chi^2$ compared to the median $\hat \chi_r^2$. Exactly this criteria is used to described the goodness of a model for fitting a population of galaxies. Thus, we use $\hat \chi_r^2$ to compare the competing models (RAR, DC14, NFW). The secondary parameter $n$ is a descriptor for the population distribution itself. It tells, how bad the first half of the population is fitted and how good the other half. The higher $n$ the steeper is the slope of the curve at $\hat \chi_r^2$. A high $n$ value would imply that the fits of most galaxies have a reduced $\chi^2$ slightly greater than the median $\hat \chi_r^2$. In other words, a high $n$ value imply that the considered model fits all galaxies similarly good with a reduced $\chi^2$ around the median $\hat \chi_r^2$. There would be only few very good fits and same for very bad fits compared to $\hat \chi_r^2$. Thus, the descriptor $n$ is a supplemental parameter, which gives further information about a model, but it is not useful to compare different models like $\hat \chi_r^2$.

The introduced method is similar to the empirical distribution function when the normalized population is interpreted as a probability depending on the variable $\chi_r^2$.

The goodness of the competing models (RAR, DC14, NFW) in fitting the SPARC galaxy sample is summarized in \cref{fig:gof-stairs}. RAR and DC14 are similarly good when the morphological type is ignored. NFW, in contrast, is clearly disfavored here. This picture becomes much more obvious when only megallanic types (Sd,Sdm,Sm,Im) are considered. On the other hand, focusing on non-megallanic types (S0,Sa,Sab,Sb,Sc,Scd,BCD) there is no clear favorite. This implies a connection between dark matter and the morphological type. Moreover, comparing DC14 and NFW implies that baryonic feedback is an important mechanism for galaxy formation. On the other hand, our results show that the RAR model is as good as DC14 although it doesn't consider any baryonic matter contribution nor any baryonic feedback mechanism.

Galaxies of magellanic type which are fitted very well, especially, by the RAR model are NGC0055 (Sm), UGC05986 (Sm), UGC05750 (Sdm), UGC05005 (Im), F565-V2, (Im), UGC06399 (Sm), UGC10310 (Sm), UGC07559 (Im), UGC07690 (Im), UGC05918 (Im) and UGC05414 (Im). This bolsters the results of the goodness of model analysis in \cref{fig:gof-stairs}.


\loadfigure{figure/NGC0055rc}
\loadfigure{figure/NGC0055ac}

A Remarkably good fit with the RAR model (and DC14) is given for NGC0055 (Sm), see \cref{fig:RC:NGC0055,fig:AC:NGC0055}. We now focus on this particular prototype to investigate the large scatter in the radial acceleration correlation. The prototype demonstrates that many galaxies don't follow strictly McGaugh's empirical formula for the acceleration correlation. While in the DM dominated regime ($\SYMabar < \SYMafrak$) the acceleration relation seems to be close to McGaugh's fit (despite an offset) we find an abrupt decrease just before the baryon dominated regime. These abrupt decreases we interpret as the source of the increased scatter, especially in form of a \textit{scatter shower} in the low acceleration regime of dark matter. RAR and DC14 can reproduce that abrupt decrease very well, but NFW has serious problems in the baryon dominated regime what is due to its cuspy design. An interesting characteristic of the rotation curve is its slope in the inner halo (baryon dominated region) what shows the nature of the halo: cored or cuspy. The RAR model implies a cored halo and is therefore good in fitting those halos. Contrary, the NFW model gives by design a cuspy halo. For the RAR model (with a well defined halo) as well as for the NFW model those inner halo slopes are fixed while DC14 is more flexible. In sum, the prototype NGC0055 demonstrates clearly that the inner halo shape is important when the relevance of a cuspy or cored halo is considered. It should be now obvious that the new perspective of the radial acceleration correlation ($\SYMabar-\SYMaDM$) is preferred compared to the original ($\SYMabar-\SYMaobs$) which obscures the inner halo relation.

Another interesting phenomena are oscillations in the rotation curve (RC). Most galaxies, which are poorly fitted by any of the considered models, show \textit{short range} oscillations with more than one maxima in their RC. None of the models can explain that phenomena found mostly in non-magellanic galaxy types: e.g. NGC2403 (Scd), UGC02953 (Sab), NGC6015 (Scd), UGC09133 (Sab), UGC06787 (Sab), UGC11914 (Sab), NGC1003 (Scd), NGC0247 (Sd), UGC08699 (Sab) and UGC03205 (Sab). Indeed, all models show only one maxima in their RC within the range of interest. Phenomenally, in the RAR model it is possible to vary the width of the maxima bump in the RC through the cutoff parameter $W_0$ in the strong cutoff regime. But without or with weak cutoff the RAR model shows long range oscillations, equivalent to the IS model. However, these oscillations are too long and therefore not a convenient explanation. On the other hand, in the case of strong cutoff we obtain a narrow maxima bump necessary for many RCs, especially for galaxies of magellanic type which do not show those oscillations in general, e.g. NGC0055 (Sm), UGC05986 (Sm), UGC07323 (Sdm), KK98-251 (Im), DDO168 (Im), D631-7 (Im), DDO161 (Im), UGCA442 (Sm), DDO154 (Im), F583-1 (Sm), but also for non-magellanic galaxy types, e.g. NGC5585 (Sd), NGC7793 (Sd), UGC06614 (Sa), ESO079-G014 (Sbc), F571-8 (Sc), NGC0891 (Sb), UGC06614 (Sa), UGC09037 (Scd), NGC4217 (Sb), UGC04278 (Sd). NFW has a wide maxima bump and fits therefore oscillating RC well only on average. Same for DC14 since it may vary the maxima bump width. Exactly these oscillations --- or a lack of them --- may be an interesting hint for the connections between dark matter and the morphological type as suggested by the goodness of model analysis, see \cref{fig:gof-stairs}.
