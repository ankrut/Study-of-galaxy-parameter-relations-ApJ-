\loadfigure{figure/AccelerationGrid}

%%%%%%%%%%%%%%%%%%%%%%%%%%%%%%%%%%%%%%%%%%%%%%%%%%%%
%%%%%%%%%%%%%%%%%%%%%%%%%%%%%%%%%%%%%%%%%%%%%%%%%%%%
\section{Results}
\label{sec:results}
%%%%%%%%%%%%%%%%%%%%%%%%%%%%%%%%%%%%%%%%%%%%%%%%%%%%
%%%%%%%%%%%%%%%%%%%%%%%%%%%%%%%%%%%%%%%%%%%%%%%%%%%%

After an insightful analysis of the parameter distribution of each DM model which best fits the SPARC RCs (given in appendix \ref{sec:appendix:parameter-distribution}), we compare between them following two complementary approaches.

First, we consider the entire galaxy sample and extract the radial acceleration information for the total and baryonic components at each galactocentric radii, and put them all together as in \cref{fig:acceleration:grid} (SPARC-window). In this way, we reduce any characteristics of individual galaxies into an overall (\textit{global}) picture, i.e. the Radial Acceleration Relation \citep{2016PhRvL.117t1101M} and Mass Discrepancy Acceleration Relation \citep{2004ApJ...609..652M,2014Galax...2..601M}, respectively.

In the second approach, we consider each \textit{individual} galaxy and perform a goodness of DM-model analysis, showing how well the inferred DM RCs are fitted by these models.

%%%%%%%%%%%%%%%%%%%%%%%%%%%%%%%%%%%%%%%%%%%%%%%%%%%%
\subsection{Acceleration relations}
\label{sec:result:ac}
%%%%%%%%%%%%%%%%%%%%%%%%%%%%%%%%%%%%%%%%%%%%%%%%%%%%

\loadfigure{figure/AccelerationGrid}

% radial acceleration relation
The Radial Acceleration Relation connects the centripetal accelerations of the baryonic and total matter components. This relation is not limited to disk galaxies but also holds for other galaxy types (e.g. ellipticals, lenticulars, dwarfs spheroidals and even low-surface-brightness galaxies), what makes it a true universal law among morphology classification \citep{2016PhRvL.117t1101M,2017ApJ...836..152L,2019ApJ...873..106D}. Despite a relatively large scatter, this relation on halo scales seems to imply a fundamental acceleration scale $\mathfrak{a}_0$, which is present in the non-linear fitting function as obtained in \cite{2016PhRvL.117t1101M}, and given by
% : it shows a non-linear correlation between the radial acceleration caused by the total matter and the one generated by its baryonic component only.
%
\begin{equation}
	\label{eq:mcgaugh-fit}
	\SYMatot = \frac{\SYMabar}{1 - \e^{-\sqrt{\SYMabar/{\SYMafrak}}}},
\end{equation}
%
where $\SYMafrak$ is the only adjustable parameter. In the low acceleration regime ($\SYMabar \ll \SYMafrak$), where DM dominates, it clearly shows a deviation from a linear correlation (see top panels of \cref{fig:acceleration:grid}). While in the high acceleration regime ($\SYMabar \gg \SYMafrak$), dominated by baryonic matter, the linear relation is recovered.

% different explanations
%An acceleration scale of this kind naturally arises in Modified Newtonian Dynamics (MOND) \citep{2015CaJPh..93..169K,2016arXiv160906642M,2016PhRvL.117t1101M,2018A&A...615A...3L}, which has been used to interpret such a fundamental scale as evidence against the $\Lambda$CDM paradigm and in favor to the MOND theory. However, more recent studies dedicated to analyze this universal relation within the (Bayesian) posterior distributions on the acceleration scales of individual galaxies (across a large sample), have provided evidence against the existence of such a fundamental constant and in favour of $\mathfrak{a}_0$ to be an emergent magnitude (see e.g. \citealp{2020MNRAS.494.2875M} and references therein). On the other hand, it has been extensively shown that the Radial Acceleration Relation is consistent with the $\Lambda$CDM paradigm, as found in hydrodynamical N-body simulations  \citep{2016MNRAS.456L.127D,2017MNRAS.471.1841N,2017PhRvL.118p1103L,2019MNRAS.485.1886D}. This conclusion is in line with a more phenomenological (independent) study based on Universal Rotation Curves including for baryonic mass models \citep{2018FoPh...48.1517S}.

The rotation curve for each component (e.g. bulge, disk, gas) traces its centripetal acceleration $\SYMacc = v^2/r$, giving access to independent acceleration measurements. The Radial Acceleration Relation compares the radial acceleration due to the total mass ($\SYMatot$) with that due to the baryonic mass ($\SYMabar$). \citet{2016PhRvL.117t1101M} describes that correlation empirically by \cref{eq:mcgaugh-fit}.

% source of data
\add{For the SPARC galaxies $\SYMabar = V_{\rm bar}^2/r$ and $\SYMatot = V_{\rm tot}^2/r$ are inferred from the SPARC data. See section \ref{sec:data} for details how the circular velocities and radii are obtained. For the competing DM models $\SYMabar = V_{\rm bar}^2/r$ is equally inferred from the SPARC data while $\SYMatot = [V_{\rm bar}^2 + v_{\rm DM}(r)^2]/r$ is inferred from the total mass distribution, composed of the observed baryonic component (taken from SPARC data) and the best-fitted circular velocities of the dark matter component $v_{\rm DM}(r)$ for each DM model. See sections \ref{sec:fitting} and \ref{boundaryC} for details how best-fits are obtained.}

% RAR/MDAR reproduction qualitatively
For the SPARC data as well as for each DM model, we applied then a least-square fitting to obtain $\SYMafrak$. The result obtained here from the SPARC data only (i.e. without assuming any specific DM model) is fully consistent (within errors) with $\SYMafrak \approx \SI{1.2E-10}{\metre\per\second^2}$ as obtained originally in \citet{2016PhRvL.117t1101M}, validating our procedure. Each DM model is then characterized by a specific best-fit $\SYMafrak$. The corresponding curves are plotted as solid lines in \cref{fig:acceleration:grid}. In all cases, we obtain values close to the one obtained in \citet{2016PhRvL.117t1101M}. This allows to conclude that all competing DM models are able to reproduce the Radial Acceleration Relation. Moreover, they reproduce it equally good without a clear statistically preferred model.

% mass discrepancy acceleration relation
A closely related relation is the MDAR relation between the baryonic and total mass components, defined by $D = M_{\rm tot}/M_{\rm bar}$ \add{with $M_{\rm bar}$ being the total baryonic mass and $M_{\rm tot}$ the total galaxy mass accounting for baryonic and dark matter}. For $a = \diff{\Phi}{r}$ and $\Phi(r)$ being the gravitational potential of a spherically symmetry mass distribution the mass discrepancy can be equivalently written as $D = \SYMatot/\SYMabar$. The results are illustrated in the bottom panels of \cref{fig:acceleration:grid}.

% beyond RAR/MDAR
According to some authors, the above two acceleration relations do not imply the need of any new physics and may be explained within the $\Lambda$CDM framework \citep{2016MNRAS.456L.127D,2017MNRAS.466.1648K,2016arXiv161208857S,2018FoPh...48.1517S}. For smaller disk and LSB galaxies (extending the original SPARC sample), \citet{2019ApJ...873..106D} found that the McGaugh et al. relation is a limiting case of a more complex relation with the need of adding one extra galaxy parameter. In addition, based on modern cosmological simulations, \citet{2017ApJ...835L..17K} predict even a redshift dependency of the acceleration parameter $\SYMafrak$, emphasizing that the correlation is universal only regarding the morphological classification.

% next strategy
Additionally to the acceleration relations mentioned above (which account for the entire accelerations distributions among all galaxies and at different radii), the next strategy is to focus on the DM components of each galaxy and gather best-fits of the inferred rotation curves allowing for another (related) quantitative comparison of the DM models.

%%%%%%%%%%%%%%%%%%%%%%%%%%%%%%%%%%%%%%%%%%%%%%%%%%%%
\subsection{Goodness of model}
\label{sec:result:gof}
%%%%%%%%%%%%%%%%%%%%%%%%%%%%%%%%%%%%%%%%%%%%%%%%%%%%

The SPARC galaxies show different characteristics in their rotation curve such as a nearly flat curve through the entire galaxy data; a rising trend in the inner halo followed by a single maximum; or multiple extrema in the form of oscillations. See \cref{fig:benchmark:total-rotation-curves} for three typical examples within the SPARC data-set. Some galaxies show just a rising trend implying that the rotation curves are incomplete, likely due to the faintness and/or lack of data for outermost halo stars. Of interest is therefore a quantitative description about the goodness of a DM halo model fitting the entire galaxy sample (120 galaxies).

The goodness of a fit for a single galaxy is well described by the $\chi^2$ value, see \cref{eqn:method:chi-square}. When competing models with different number of parameters are compared it is appropriate to consider the reduced $\chi^2$ defined as $\chi_r^2 = \chi^2/d$ with the degree of freedom $d = N-p$, $N$ being the number of observables (for a single galaxy) and $p$ the number of parameters (of the considered model).

The question now arises how to compare the competing models for a population of galaxies. In order to find the goodness of a model which is robust against outliers, we ask \textit{how many} fitted galaxies have a (reduced) $\chi^2$ \textit{lower} than a given one. It turns out that the population curve resembling a cumulative distribution function (CDF) follows nearly a log-normal distribution. We use the mean value, labelled as $\hat \chi^2_r$, as the criteria to described the goodness of a model for fitting a population of galaxies within the SPARC data-set. A parameter analysis of the best-fit solutions for each DM halo model is detailed in appendix \ref{sec:appendix:parameter-distribution}.

\loadfigure{figure/GoodnessAll}
\loadfigure{figure/GoodnessWithCutoff}

The goodness analysis \add{for the entire SPARC sample (120 galaxies)} is shown in \cref{fig:goodness:all} and can be summarized as: The NFW model ($p=2$) is statistically disfavoured with respect to the other DM halo models; the Einasto model ($p=3$) and the DC14 model ($p=3$), on the other hand, are statistically favoured with similarly good results; the Burkert model ($p=2$) and the fermionic model ($p=3$) produce comparable results but are somewhat statistically less favoured. %when the entire SPARC sample (120 galaxies) is considered.

\add{We recall that the DM models are suited for DM halo rotation curves with only one maximum. Therefore, we restrict further the SPARC sample to galaxies (44) showing one clear maximum in their rotation curve (see also section \ref{sec:fermionic-halos:diversity} for details about the diversity of SPARC galaxies). This conditions is equivalently reflected in the fermionic scenario} where data supports for a significant escape of DM particles, that is for energy-cutoff values at plateau of $W_p < 10$ \add{(see appendix \ref{sec:appendix:parameter-distribution} for further details)}. For that sub-sample the picture changes considerably, see \cref{fig:goodness:with-cutoff}, and can be summarized as: The NFW model is statistically even more disfavoured while all other models (including fermionic DM) become more comparable --- with a little tendency for Einasto and against Burkert.

We remind that the DC14 model is based on the analysis of hydro-dynamically simulated galaxies including complex baryonic feedback processes \citep{2014MNRAS.441.2986D}. The good performance of DC14 may indicate the importance of baryonic feedback in galaxy formation. Indeed, our results regarding DC14 are in line with the literature as well since, as can be seen from \cref{fig:parameter-distribution:dc14} (bottom panel), the bulk of our ($\alpha,\beta,\gamma$) parameters for the SPARC data-set lies within the windows (0,2.6); (2.3,4); (0,2) in rough agreement with \citet{2014MNRAS.441.2986D}. Interestingly, we obtain a mean for the stellar to DM mass ratio $X = \log_{10}(M_*/M_\mathrm{halo})$ of $-2.4$, which is within the $X$ values ($-2.5,-2.3$) where DM cusps are most effectively flattened due to baryonic effects as reported in \citet{2014MNRAS.441.2986D}. Besides, our results are in good agreement with those reported in \citet{2013ApJ...770...57B} from abundance matching, since they obtain X values between (-2.7, -1.7) for halo masses in the range between $10^{10} M_\odot$ and $10^{12} M_\odot$, with $X=-2.4$ for $M_h=10^{11} M_\odot$, the latter coinciding with the mean $X$ value in this work.

\add{In the case of Einasto profiles, the comparison with the literature is more difficult since the effects of baryonic feedback are not explicitly reported through its free parameters. For example in \cite{2019MNRAS.483.4086B} they obtain, within (zoom-in) hydro-dynamical simulations and for $\sim \SI{E10}{\Msun}$ halos, a reduction on inner-halo densities (i.e. at the convergence radius $r=\SI{0.26}{\kilo\parsec}$) of up to $\SI{45}{\percent}$ in WDM cosmologies with respect to the analogous WDM-only simulations. A reduction which is about $15 \%$ more pronounced than comparing WDM to CDM only simulations \citep{2019MNRAS.483.4086B}. Even if all the resulting DM halos are there fitted with the Einasto profile, these density reductions are only expressed through its concentration parameter (which is a function of the scale radius and the virial radius). They found indeed that Einasto profiles in WDM cosmologies have smaller concentration parameters than the CDM counterparts. This concentration parameter correlates inversely proportional to the Einasto shape parameter (denoted here as $\kappa$) as originally shown through Figs. 13 and 14 in \citet{2014MNRAS.441.3359D}. Thus, considering that $\kappa\approx 0.2$ correspond to CDM-only cosmologies \citep{2014MNRAS.441.3359D,2017MNRAS.471.3547F}, then larger $\kappa$ values are expected for Einasto profiles having flatter inner-halo slopes \citep{2014MNRAS.441.3359D}. Indeed, our results are qualitatively in line with those reported in \cite{2019MNRAS.483.4086B} since we find a mean value of $\kappa \approx 0.4$ (see \cref{fig:parameter-distribution:einasto}), thus implying smaller concentration parameters than for CDM profiles. However more work from hydro-dynamical simulations using Einasto fitting profiles is needed in order to make a proper quantitative comparison with our statistical results.}
% Now, considering that the Einasto parameter ($\kappa$) corresponding to CDM-only simulations is found to be $\approx 0.2$ \citep{2014MNRAS.441.2986D}, 
% While the reduction in density on such inner-halo scales is of about $30\%$ when considering WDM-only simulations with respect to CDM-only ones \citep{2019MNRAS.483.4086B}.
% then values of $\kappa >0.2$ are expected in WDM cosmologies when baryonic effects are included. This is indeed in line with the values reported here in \cref{fig:parameter-distribution:einasto} with a mean value $\kappa \approx 0.4$, 
    
%However, the interpretation regarding the relevance (or not) of baryonic effects for Einasto profiles is more subtle. Results from hydro-dynamical (zoom-in) simulations obtained for $\sim \SI{E10}{\Msun}$ halos, show a reduction on inner-halo densities (i.e. at the convergence radius $r=\SI{0.26}{\kilo\parsec}$) of up to $\SI{45}{\percent}$ in WDM cosmologies with respect to the analogous DM-only simulations \citep{2019MNRAS.483.4086B}. A reduction which, when calibrated through Einasto profiles, is already a $15-20 \%$ more pronounced than the one typically obtained for the same WDM halos respect to the CDM ones in DM-only cosmological simulations \citep{2019MNRAS.483.4086B}. Now, for SPARC halos with total mass of few $\SI{E10}{\Msun}$ and with roughly the same Einasto scale-radius our results show that the Einasto $\kappa$ parameter required to produce the above $\sim\SI{45}{\percent}$ density reduction at $\SI{0.26}{\kilo\parsec}$ is $\kappa\approx 0.3$ (which is close to the mean value $\kappa \approx 0.4$ as shown in \cref{fig:parameter-distribution:einasto}, and only one dex above the standard $0.2$ of CDM). However it is important to mention that for the mean $\kappa$ value ($0.4$), the corresponding inner-halo density drop is so pronounced (i.e the inner-slope is almost flat, see \cref{fig:profile-illustration-mep}), that the halos look more like fermionic profiles, indicating instead a possible relevance of an underlying MEP mechanism for halo formation.

We would like also to point out that many galaxies in the SPARC data set are missing significant information in the outer halo (e.g. due to faint stars) or show a complex behavior (oscillatory-pattern) in their rotation curves (see e.g. right box in \cref{fig:benchmark:total-rotation-curves}). In any case, it does not allow to univocally determine the cutoff parameter (i.e. $W_p$) for the fermionic DM model since any sufficiently large $W_p$ would not change the $\chi^2$ value (see e.g. bottom left panel of \cref{fig:chi-analysis}). Nevertheless, there are some other individual galaxies where the escaping particles effects are clearly preferred. Interestingly, all of those galaxies are of magellanic type: NGC0055 (Sm), UGC05986 (Sm), UGC05750 (Sdm), UGC05005 (Im), F565-V2, (Im), UGC06399 (Sm), UGC10310 (Sm), UGC07559 (Im), UGC07690 (Im), UGC05918 (Im) and UGC05414 (Im).

Moreover, many galaxies, which are poorly fitted by any of the considered models, show \textit{short range} oscillations in their rotation curves with more than one maximum. None of the models can provide a clear explanation of that phenomena, found usually in non-magellanic galaxy types: e.g. NGC2403 (Scd), UGC02953 (Sab), NGC6015 (Scd), UGC09133 (Sab), UGC06787 (Sab), UGC11914 (Sab), NGC1003 (Scd), NGC0247 (Sd), UGC08699 (Sab) and UGC03205 (Sab).

On phenomenological grounds, in the fermionic DM model it is possible to vary the width of the maximum bump in the RC through the cutoff parameter in the strong or moderate cutoff regime ($W_p \lesssim 10$). Whether with weak ($W_p \gtrsim 10$) or even without cutoff-effects, the RC solutions of the model show long range oscillations, similar to the isothermal model. In any case, these RC oscillations have a too long wavelength and therefore do not offer a convenient explanation. On the other hand, in the case of strong cutoff, we obtain a narrow maximum bump necessary for many RCs, especially for galaxies of magellanic type (see above for examples), which usually do not show those oscillations, but also for some non-magellanic galaxy types, e.g. NGC5585 (Sd), NGC7793 (Sd), UGC06614 (Sa), ESO079-G014 (Sbc), F571-8 (Sc), NGC0891 (Sb), UGC06614 (Sa), UGC09037 (Scd), NGC4217 (Sb), UGC04278 (Sd).

NFW and Burkert models cannot explain variations of the inner and outer rotation curve because the parameters ($\beta$ and $\gamma$) responsible for such a behaviour (see \cref{eqn:hernquist}) are fixed. Additionally a transition from the inner to the outer halo is generally characterized by $\alpha$ (\add{or $\kappa$}). In contrast to NFW and Burkert, the DC14 and Einasto models have a free parameter which affects the inner/outer rotation curve steepness and the sharpness of the halo transition, simultaneously. Such a flexibility is reflected in generally better $\chi^2$ values. Nevertheless, the goodness for oscillating RCs remains rather poor.
