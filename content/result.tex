%%%%%%%%%%%%%%%%%%%%%%%%%%%%%%%%%%%%%%%%%%%%%%%%%%%%
%%%%%%%%%%%%%%%%%%%%%%%%%%%%%%%%%%%%%%%%%%%%%%%%%%%%
\section{Results}
\label{sec:results}
%%%%%%%%%%%%%%%%%%%%%%%%%%%%%%%%%%%%%%%%%%%%%%%%%%%%
%%%%%%%%%%%%%%%%%%%%%%%%%%%%%%%%%%%%%%%%%%%%%%%%%%%%

After an insightful analysis of the parameter distribution of each DM model which best fits the SPARC RCs (given in appendix \ref{sec:appendix:parameter-distribution}), we compare between them following two complementary approaches.

First, we consider the entire galaxy sample and extract the radial acceleration information for the total and baryonic components at each galactocentric radii, and put them all together as in \cref{fig:acceleration:grid} (SPARC-window). In this way, we reduce any characteristics of individual galaxies into an overall (\textit{global}) picture, i.e. the Radial Acceleration Relation \citep{2016PhRvL.117t1101M} and Mass Discrepancy Acceleration Relation \citep{2004ApJ...609..652M,2014Galax...2..601M}, respectively.

In the second approach, we consider each \textit{individual} galaxy and perform a goodness of DM-model analysis, showing how well the inferred DM RCs are fitted by these models. As typical examples, we show in \cref{fig:benchmark:total-rotation-curves} detailed rotation curve fit analyses for three selected galaxies indicating the limitations of the observational data and/or the case where data supports for one clear maximum in the rotation curve of each DM model.

\loadfigure{figure/AccelerationGrid}

%%%%%%%%%%%%%%%%%%%%%%%%%%%%%%%%%%%%%%%%%%%%%%%%%%%%
\subsection{Acceleration relations}
\label{sec:result:ac}
%%%%%%%%%%%%%%%%%%%%%%%%%%%%%%%%%%%%%%%%%%%%%%%%%%%%

% radial acceleration relation
The RC for each component (e.g. bulge, disk, gas) traces its centripetal acceleration $\SYMacc = v^2/r$, giving access to independent acceleration measurements. The Radial Acceleration Relation compares the radial acceleration due to the total mass ($\SYMatot$) with that due to the baryonic mass ($\SYMabar$). \citet{2016PhRvL.117t1101M} describes that correlation empirically by \cref{eq:mcgaugh-fit}.

% source of data
\add{For the SPARC galaxies $\SYMabar$ and $\SYMatot$ are inferred from the SPARC data. For the competing DM models $\SYMabar$ is equally inferred from the SPARC data while $\SYMatot$ is inferred from the total mass distribution: composed of the measured baryonic component (i.e SPARC data) and the given DM component (i.e. best-fit).}

% RAR/MDAR reproduction qualitatively
For the SPARC data as well as for each DM model, we applied then a least-square fitting to obtain $\SYMafrak$. The result obtained here from the SPARC data only (i.e. without assuming any specific DM model) is fully consistent (within errors) with $\SYMafrak \approx \SI{1.2E-10}{\metre\per\second^2}$ as obtained originally in \citet{2016PhRvL.117t1101M}, validating our procedure. Each DM model is then characterized by a specific best-fit $\SYMafrak$. The corresponding curves are plotted as solid lines in \cref{fig:acceleration:grid}. In all cases, we obtain values close to the one obtained in \citet{2016PhRvL.117t1101M}. This allows to conclude that all competing DM models are able to reproduce the Radial Acceleration Relation. Moreover, they reproduce it equally good without a clear statistically preferred model.

% mass discrepancy acceleration relation
A closely related relation is the MDAR relation between the baryonic and total mass components, defined by $D = M_{\rm tot}/M_{\rm bar}$.  For $a = \diff{\Phi}{r}$ and $\Phi(r)$ being the gravitational potential of a spherically symmetry mass distribution the mass discrepancy can be equivalently written as $D = \SYMatot/\SYMabar$. The results are illustrated in the bottom panels of \cref{fig:acceleration:grid}.

% beyond RAR/MDAR
According to some authors, the above two acceleration relations do not imply the need of any new physics and may be explained within the $\Lambda$CDM framework \citep{2016MNRAS.456L.127D,2017MNRAS.466.1648K,2016arXiv161208857S,2018FoPh...48.1517S}. For smaller disk and LSB galaxies (extending the original SPARC sample), \citet{2019ApJ...873..106D} found that the McGaugh et al. relation is a limiting case of a more complex relation with the need of adding one extra galaxy parameter. In addition, based on modern cosmological simulations, \citet{2017ApJ...835L..17K} predict even a redshift dependency of the acceleration parameter $\SYMafrak$, emphasizing that the correlation is universal only regarding the morphological classification.

% halo focus critics
%Moreover, both relations have been criticized that their representation focus only on the low acceleration regime where DM dominates. Therefore, \citet{2019ApJ...877...18C} suggested to compare the baryonic component $\SYMabar$, with the ratio $\SYMaDM/\SYMabar$ which gives also information about DM in the high acceleration regime, dominated by baryonic matter. They realized that the original representation of such acceleration relations shows a link on halo scales but simultaneously obscures the relation between dark and baryonic matter on inner halo scales. Therefore, any DM model (cored or cuspy) would reproduce the linear relation in the baryons dominated region. However, due to rather large uncertainties in that region, any inferred relation between dark and baryonic components would be highly speculative.

% next strategy
Additionally to the acceleration relations mentioned above (which account for the entire accelerations distributions among all galaxies and at different radii), the next strategy is to focus on the DM components of each galaxy and gather best-fits of the inferred rotation curves allowing for another (related) quantitative comparison of the DM models.
%%%%%%%%%%%%%%%%%%%%%%%%%%%%%%%%%%%%%%%%%%%%%%%%%%%%
\subsection{Goodness of model}
\label{sec:result:gof}
%%%%%%%%%%%%%%%%%%%%%%%%%%%%%%%%%%%%%%%%%%%%%%%%%%%%

The SPARC galaxies show different characteristics in their rotation curve such as a nearly flat curve through the entire galaxy data; a rising trend in the inner halo followed by a single maximum; or multiple extrema in the form of oscillations. See \cref{fig:benchmark:total-rotation-curves} for three typical examples within the SPARC data-set. Some galaxies show just a rising trend implying that the rotation curves are incomplete, likely due to the faintness and/or lack of data for outermost halo stars. Of interest is therefore a quantitative description about the goodness of a DM halo model fitting the entire galaxy sample (120 galaxies).

The goodness of a fit for a single galaxy is well described by the $\chi^2$ value, see \cref{eqn:method:chi-square}. When competing models with different number of parameters are compared it is appropriate to consider the reduced $\chi^2$ defined as $\chi_r^2 = \chi^2/d$ with the degree of freedom $d = N-p$, $N$ being the number of observables (for a single galaxy) and $p$ the number of parameters (of the considered model).

The question now arises how to compare the competing models for a population of galaxies. In order to find the goodness of a model which is robust against outliers, we ask \textit{how many} fitted galaxies have a (reduced) $\chi^2$ \textit{lower} than a given one. It turns out that the population curve resembling a cumulative distribution function (CDF) follows nearly a log-normal distribution. We use the mean value, labelled as $\hat \chi^2_r$, as the criteria to described the goodness of a model for fitting a population of galaxies within the SPARC data-set. A parameter analysis of the best-fit solutions for each DM halo model is detailed in appendix \ref{sec:appendix:parameter-distribution}. The goodness analysis is shown in \cref{fig:goodness:all} and can be summarized as:

\loadfigure{figure/GoodnessAll}
\loadfigure{figure/GoodnessWithCutoff}

\begin{asparaenum}[(i)]
    \item The NFW model ($p=2$) is statistically disfavoured with respect to the other DM halo models.

    \item The Einasto model ($p=3$) and the DC14 model ($p=3$), on the other hand, are statistically favoured with similarly good results.
    
    We remind that the DC14 model is based on the analysis of hydro-dynamically simulated galaxies including complex baryonic feedback processes \citep{2014MNRAS.441.2986D}. The good performance of DC14 may indicate the importance of baryonic feedback in galaxy formation. Indeed, our results regarding DC14 are in line as well with the literature since, as can be seen from \cref{fig:parameter-distribution:dc14} (bottom panel), the bulk of our ($\alpha,\beta,\gamma$) parameters for the SPARC data-set lies within the windows (0,2.6); (2.3,4); (0,2) in rough agreement with \citet{2014MNRAS.441.2986D}. Interestingly, we obtain a mean for the stellar to DM mass ratio $X = \log_{10}(M_*/M_\mathrm{halo})$ of $-2.4$, which is within the $X$ values ($-2.5,-2.3$) where DM cusps are most effectively flattened due to baryonic effects as reported in \citet{2014MNRAS.441.2986D}. \add{Besides, our results are in good agreement with those reported in \citet{2013ApJ...770...57B} from abundance matching, since they obtain X values between (-2.7, -1.7) for halo masses in the range between $10^{10} M_\odot$ and $10^{12} M_\odot$, with $X=-2.4$ for $M_h=10^{11} M_\odot$, the latter coinciding with the mean $X$ value in this work.}  
    
    However, the interpretation regarding the relevance (or not) of baryonic effects for Einasto profiles is more subtle. Results from hydro-dynamical (zoom-in) simulations obtained for $\sim \SI{E10}{\Msun}$ halos, show a reduction on inner-halo densities (i.e. at the convergence radius $r=\SI{0.26}{\kilo\parsec}$) of up to $\SI{45}{\percent}$ in WDM cosmologies with respect to the analogous DM-only simulations \citep{2019MNRAS.483.4086B}. A reduction which, when calibrated through Einasto profiles, is already a $15-20 \%$ more pronounced than the one typically obtained for the same WDM halos respect to the CDM ones in DM-only cosmological simulations \citep{2019MNRAS.483.4086B}. Now, for SPARC halos with total mass of few $\SI{E10}{\Msun}$ and with roughly the same Einasto scale-radius our results show that the Einasto $\alpha$ parameter required to produce the above $\sim\SI{45}{\percent}$ density reduction at $\SI{0.26}{\kilo\parsec}$ is $\alpha\approx 0.3$ (which is close to the mean value $\alpha \approx 0.4$ as shown in \cref{fig:parameter-distribution:einasto}, and only one dex above the standard $0.2$ of CDM). However it is important to mention that for the mean $\alpha$ value ($0.4$), the corresponding inner-halo density drop is so pronounced (i.e the inner-slope is almost flat, see \cref{fig:profile-illustration-mep}), that the halos look more like fermionic profiles, indicating instead a possible relevance of an underlying MEP mechanism for halo formation.

    \item The Burkert model ($p=2$) and the fermionic model ($p=3$), both produce comparable results but are somewhat statistically less favoured when the entire SPARC sample (120 galaxies) is considered. However, in the fermionic scenario the picture changes considerably for the sub-sample (44 galaxies) where data supports for a significant escape of DM particles, that is for energy-cutoff values at plateau of $W_p < 10$. For that sub-sample, the NFW model is statistically even more disfavoured while all other models are comparable, though with a little tendency for Einasto and against Burkert, see \cref{fig:goodness:with-cutoff}.
\end{asparaenum}

We would like also to point out that many galaxies in the SPARC data set are missing significant information in the outer halo (e.g. due to faint stars) or show a complex behavior (oscillatory-pattern) in their rotation curves (see e.g. right box in \cref{fig:benchmark:total-rotation-curves}). In any case, it does not allow to univocally determine the cutoff parameter (i.e. $W_p$) for the fermionic DM model since any sufficiently large $W_p$ would not change the $\chi^2$ value (see e.g. bottom left panel of \cref{fig:chi-analysis}). Nevertheless, there are some other individual galaxies where the escaping particles effects are clearly preferred. Interestingly, all of those galaxies are of magellanic type: NGC0055 (Sm), UGC05986 (Sm), UGC05750 (Sdm), UGC05005 (Im), F565-V2, (Im), UGC06399 (Sm), UGC10310 (Sm), UGC07559 (Im), UGC07690 (Im), UGC05918 (Im) and UGC05414 (Im).

Moreover, many galaxies, which are poorly fitted by any of the considered models, show \textit{short range} oscillations in their rotation curves with more than one maximum. None of the models can provide a clear explanation of that phenomena, found usually in non-magellanic galaxy types: e.g. NGC2403 (Scd), UGC02953 (Sab), NGC6015 (Scd), UGC09133 (Sab), UGC06787 (Sab), UGC11914 (Sab), NGC1003 (Scd), NGC0247 (Sd), UGC08699 (Sab) and UGC03205 (Sab).

On phenomenological grounds, in the fermionic DM model it is possible to vary the width of the maximum bump in the RC through the cutoff parameter in the strong or moderate cutoff regime ($W_p \lesssim 10$). Whether with weak ($W_p \gtrsim 10$) or even without cutoff-effects, the RC solutions of the model show long range oscillations, similar to the isothermal model. In any case, these RC oscillations have a too long wavelength and therefore do not offer a convenient explanation. On the other hand, in the case of strong cutoff, we obtain a narrow maximum bump necessary for many RCs, especially for galaxies of magellanic type (see above for examples), which usually do not show those oscillations, but also for some non-magellanic galaxy types, e.g. NGC5585 (Sd), NGC7793 (Sd), UGC06614 (Sa), ESO079-G014 (Sbc), F571-8 (Sc), NGC0891 (Sb), UGC06614 (Sa), UGC09037 (Scd), NGC4217 (Sb), UGC04278 (Sd).

NFW and Burkert models cannot explain variations of the inner and outer rotation curve because the parameters ($\beta$ and $\gamma$) responsible for such a behaviour (see \cref{eqn:hernquist}) are fixed. Additionally a transition from the inner to the outer halo is generally characterized by $\alpha$. In contrast to NFW and Burkert, the DC14 and Einasto models have a free parameter which affects the inner/outer rotation curve steepness and the sharpness of the halo transition, simultaneously. Such a flexibility is reflected in generally better $\chi^2$ values. Nevertheless, the goodness for oscillating RCs remains rather poor.

%%%%%%%%%%%%%%%%%%%%%%%%%%%%%%%%%%%%%%%%%%%%%%%%%%%%
%%%%%%%%%%%%%%%%%%%%%%%%%%%%%%%%%%%%%%%%%%%%%%%%%%%%
\section{Results}
\label{sec:results}
%%%%%%%%%%%%%%%%%%%%%%%%%%%%%%%%%%%%%%%%%%%%%%%%%%%%
%%%%%%%%%%%%%%%%%%%%%%%%%%%%%%%%%%%%%%%%%%%%%%%%%%%%

After an insightful analysis of the parameter distribution of each DM model which best fits the SPARC RCs (given in appendix \ref{sec:appendix:parameter-distribution}), we compare between them following two complementary approaches.

First, we consider the entire galaxy sample and extract the radial acceleration information for the total and baryonic components at each galactocentric radii, and put them all together as in \cref{fig:acceleration:grid} (SPARC-window). In this way, we reduce any characteristics of individual galaxies into an overall (\textit{global}) picture, i.e. the Radial Acceleration Relation \citep{2016PhRvL.117t1101M} and Mass Discrepancy Acceleration Relation \citep{2004ApJ...609..652M,2014Galax...2..601M}, respectively.

In the second approach, we consider each \textit{individual} galaxy and perform a goodness of DM-model analysis, showing how well the inferred DM RCs are fitted by these models. As typical examples, we show in \cref{fig:benchmark:total-rotation-curves} detailed rotation curve fit analyses for three selected galaxies indicating the limitations of the observational data and/or the case where data supports for one clear maximum in the rotation curve of each DM model.

\loadfigure{figure/AccelerationGrid}

%%%%%%%%%%%%%%%%%%%%%%%%%%%%%%%%%%%%%%%%%%%%%%%%%%%%
\subsection{Acceleration relations}
\label{sec:result:ac}
%%%%%%%%%%%%%%%%%%%%%%%%%%%%%%%%%%%%%%%%%%%%%%%%%%%%

% radial acceleration relation
The RC for each component (e.g. bulge, disk, gas) traces its centripetal acceleration $\SYMacc = v^2/r$, giving access to independent acceleration measurements. The Radial Acceleration Relation compares the radial acceleration due to the total mass ($\SYMatot$) with that due to the baryonic mass ($\SYMabar$). \citet{2016PhRvL.117t1101M} describes that correlation empirically by \cref{eq:mcgaugh-fit}.

% source of data
\add{For the SPARC galaxies $\SYMabar$ and $\SYMatot$ are inferred from the SPARC data. For the competing DM models $\SYMabar$ is equally inferred from the SPARC data while $\SYMatot$ is inferred from the total mass distribution: composed of the measured baryonic component (i.e SPARC data) and the given DM component (i.e. best-fit).}

% RAR/MDAR reproduction qualitatively
For the SPARC data as well as for each DM model, we applied then a least-square fitting to obtain $\SYMafrak$. The result obtained here from the SPARC data only (i.e. without assuming any specific DM model) is fully consistent (within errors) with $\SYMafrak \approx \SI{1.2E-10}{\metre\per\second^2}$ as obtained originally in \citet{2016PhRvL.117t1101M}, validating our procedure. Each DM model is then characterized by a specific best-fit $\SYMafrak$. The corresponding curves are plotted as solid lines in \cref{fig:acceleration:grid}. In all cases, we obtain values close to the one obtained in \citet{2016PhRvL.117t1101M}. This allows to conclude that all competing DM models are able to reproduce the Radial Acceleration Relation. Moreover, they reproduce it equally good without a clear statistically preferred model.

% mass discrepancy acceleration relation
A closely related relation is the MDAR relation between the baryonic and total mass components, defined by $D = M_{\rm tot}/M_{\rm bar}$.  For $a = \diff{\Phi}{r}$ and $\Phi(r)$ being the gravitational potential of a spherically symmetry mass distribution the mass discrepancy can be equivalently written as $D = \SYMatot/\SYMabar$. The results are illustrated in the bottom panels of \cref{fig:acceleration:grid}.

% beyond RAR/MDAR
According to some authors, the above two acceleration relations do not imply the need of any new physics and may be explained within the $\Lambda$CDM framework \citep{2016MNRAS.456L.127D,2017MNRAS.466.1648K,2016arXiv161208857S,2018FoPh...48.1517S}. For smaller disk and LSB galaxies (extending the original SPARC sample), \citet{2019ApJ...873..106D} found that the McGaugh et al. relation is a limiting case of a more complex relation with the need of adding one extra galaxy parameter. In addition, based on modern cosmological simulations, \citet{2017ApJ...835L..17K} predict even a redshift dependency of the acceleration parameter $\SYMafrak$, emphasizing that the correlation is universal only regarding the morphological classification.

% halo focus critics
%Moreover, both relations have been criticized that their representation focus only on the low acceleration regime where DM dominates. Therefore, \citet{2019ApJ...877...18C} suggested to compare the baryonic component $\SYMabar$, with the ratio $\SYMaDM/\SYMabar$ which gives also information about DM in the high acceleration regime, dominated by baryonic matter. They realized that the original representation of such acceleration relations shows a link on halo scales but simultaneously obscures the relation between dark and baryonic matter on inner halo scales. Therefore, any DM model (cored or cuspy) would reproduce the linear relation in the baryons dominated region. However, due to rather large uncertainties in that region, any inferred relation between dark and baryonic components would be highly speculative.

% next strategy
Additionally to the acceleration relations mentioned above (which account for the entire accelerations distributions among all galaxies and at different radii), the next strategy is to focus on the DM components of each galaxy and gather best-fits of the inferred rotation curves allowing for another (related) quantitative comparison of the DM models.
%%%%%%%%%%%%%%%%%%%%%%%%%%%%%%%%%%%%%%%%%%%%%%%%%%%%
\subsection{Goodness of model}
\label{sec:result:gof}
%%%%%%%%%%%%%%%%%%%%%%%%%%%%%%%%%%%%%%%%%%%%%%%%%%%%

The SPARC galaxies show different characteristics in their rotation curve such as a nearly flat curve through the entire galaxy data; a rising trend in the inner halo followed by a single maximum; or multiple extrema in the form of oscillations. See \cref{fig:benchmark:total-rotation-curves} for three typical examples within the SPARC data-set. Some galaxies show just a rising trend implying that the rotation curves are incomplete, likely due to the faintness and/or lack of data for outermost halo stars. Of interest is therefore a quantitative description about the goodness of a DM halo model fitting the entire galaxy sample (120 galaxies).

The goodness of a fit for a single galaxy is well described by the $\chi^2$ value, see \cref{eqn:method:chi-square}. When competing models with different number of parameters are compared it is appropriate to consider the reduced $\chi^2$ defined as $\chi_r^2 = \chi^2/d$ with the degree of freedom $d = N-p$, $N$ being the number of observables (for a single galaxy) and $p$ the number of parameters (of the considered model).

The question now arises how to compare the competing models for a population of galaxies. In order to find the goodness of a model which is robust against outliers, we ask \textit{how many} fitted galaxies have a (reduced) $\chi^2$ \textit{lower} than a given one. It turns out that the population curve resembling a cumulative distribution function (CDF) follows nearly a log-normal distribution. We use the mean value, labelled as $\hat \chi^2_r$, as the criteria to described the goodness of a model for fitting a population of galaxies within the SPARC data-set. A parameter analysis of the best-fit solutions for each DM halo model is detailed in appendix \ref{sec:appendix:parameter-distribution}. The goodness analysis is shown in \cref{fig:goodness:all} and can be summarized as:

\loadfigure{figure/GoodnessAll}
\loadfigure{figure/GoodnessWithCutoff}

\begin{asparaenum}[(i)]
    \item The NFW model ($p=2$) is statistically disfavoured with respect to the other DM halo models.

    \item The Einasto model ($p=3$) and the DC14 model ($p=3$), on the other hand, are statistically favoured with similarly good results.
    
    We remind that the DC14 model is based on the analysis of hydro-dynamically simulated galaxies including complex baryonic feedback processes \citep{2014MNRAS.441.2986D}. The good performance of DC14 may indicate the importance of baryonic feedback in galaxy formation. Indeed, our results regarding DC14 are in line as well with the literature since, as can be seen from \cref{fig:parameter-distribution:dc14} (bottom panel), the bulk of our ($\alpha,\beta,\gamma$) parameters for the SPARC data-set lies within the windows (0,2.6); (2.3,4); (0,2) in rough agreement with \citet{2014MNRAS.441.2986D}. Interestingly, we obtain a mean for the stellar to DM mass ratio $X = \log_{10}(M_*/M_\mathrm{halo})$ of $-2.4$, which is within the $X$ values ($-2.5,-2.3$) where DM cusps are most effectively flattened due to baryonic effects as reported in \citet{2014MNRAS.441.2986D}. \add{Besides, our results are in good agreement with those reported in \citet{2013ApJ...770...57B} from abundance matching, since they obtain X values between (-2.7, -1.7) for halo masses in the range between $10^{10} M_\odot$ and $10^{12} M_\odot$, with $X=-2.4$ for $M_h=10^{11} M_\odot$, the latter coinciding with the mean $X$ value in this work.}  
    
    However, the interpretation regarding the relevance (or not) of baryonic effects for Einasto profiles is more subtle. Results from hydro-dynamical (zoom-in) simulations obtained for $\sim \SI{E10}{\Msun}$ halos, show a reduction on inner-halo densities (i.e. at the convergence radius $r=\SI{0.26}{\kilo\parsec}$) of up to $\SI{45}{\percent}$ in WDM cosmologies with respect to the analogous DM-only simulations \citep{2019MNRAS.483.4086B}. A reduction which, when calibrated through Einasto profiles, is already a $15-20 \%$ more pronounced than the one typically obtained for the same WDM halos respect to the CDM ones in DM-only cosmological simulations \citep{2019MNRAS.483.4086B}. Now, for SPARC halos with total mass of few $\SI{E10}{\Msun}$ and with roughly the same Einasto scale-radius our results show that the Einasto $\alpha$ parameter required to produce the above $\sim\SI{45}{\percent}$ density reduction at $\SI{0.26}{\kilo\parsec}$ is $\alpha\approx 0.3$ (which is close to the mean value $\alpha \approx 0.4$ as shown in \cref{fig:parameter-distribution:einasto}, and only one dex above the standard $0.2$ of CDM). However it is important to mention that for the mean $\alpha$ value ($0.4$), the corresponding inner-halo density drop is so pronounced (i.e the inner-slope is almost flat, see \cref{fig:profile-illustration-mep}), that the halos look more like fermionic profiles, indicating instead a possible relevance of an underlying MEP mechanism for halo formation.

    \item The Burkert model ($p=2$) and the fermionic model ($p=3$), both produce comparable results but are somewhat statistically less favoured when the entire SPARC sample (120 galaxies) is considered. However, in the fermionic scenario the picture changes considerably for the sub-sample (44 galaxies) where data supports for a significant escape of DM particles, that is for energy-cutoff values at plateau of $W_p < 10$. For that sub-sample, the NFW model is statistically even more disfavoured while all other models are comparable, though with a little tendency for Einasto and against Burkert, see \cref{fig:goodness:with-cutoff}.
\end{asparaenum}

We would like also to point out that many galaxies in the SPARC data set are missing significant information in the outer halo (e.g. due to faint stars) or show a complex behavior (oscillatory-pattern) in their rotation curves (see e.g. right box in \cref{fig:benchmark:total-rotation-curves}). In any case, it does not allow to univocally determine the cutoff parameter (i.e. $W_p$) for the fermionic DM model since any sufficiently large $W_p$ would not change the $\chi^2$ value (see e.g. bottom left panel of \cref{fig:chi-analysis}). Nevertheless, there are some other individual galaxies where the escaping particles effects are clearly preferred. Interestingly, all of those galaxies are of magellanic type: NGC0055 (Sm), UGC05986 (Sm), UGC05750 (Sdm), UGC05005 (Im), F565-V2, (Im), UGC06399 (Sm), UGC10310 (Sm), UGC07559 (Im), UGC07690 (Im), UGC05918 (Im) and UGC05414 (Im).

Moreover, many galaxies, which are poorly fitted by any of the considered models, show \textit{short range} oscillations in their rotation curves with more than one maximum. None of the models can provide a clear explanation of that phenomena, found usually in non-magellanic galaxy types: e.g. NGC2403 (Scd), UGC02953 (Sab), NGC6015 (Scd), UGC09133 (Sab), UGC06787 (Sab), UGC11914 (Sab), NGC1003 (Scd), NGC0247 (Sd), UGC08699 (Sab) and UGC03205 (Sab).

On phenomenological grounds, in the fermionic DM model it is possible to vary the width of the maximum bump in the RC through the cutoff parameter in the strong or moderate cutoff regime ($W_p \lesssim 10$). Whether with weak ($W_p \gtrsim 10$) or even without cutoff-effects, the RC solutions of the model show long range oscillations, similar to the isothermal model. In any case, these RC oscillations have a too long wavelength and therefore do not offer a convenient explanation. On the other hand, in the case of strong cutoff, we obtain a narrow maximum bump necessary for many RCs, especially for galaxies of magellanic type (see above for examples), which usually do not show those oscillations, but also for some non-magellanic galaxy types, e.g. NGC5585 (Sd), NGC7793 (Sd), UGC06614 (Sa), ESO079-G014 (Sbc), F571-8 (Sc), NGC0891 (Sb), UGC06614 (Sa), UGC09037 (Scd), NGC4217 (Sb), UGC04278 (Sd).

NFW and Burkert models cannot explain variations of the inner and outer rotation curve because the parameters ($\beta$ and $\gamma$) responsible for such a behaviour (see \cref{eqn:hernquist}) are fixed. Additionally a transition from the inner to the outer halo is generally characterized by $\alpha$. In contrast to NFW and Burkert, the DC14 and Einasto models have a free parameter which affects the inner/outer rotation curve steepness and the sharpness of the halo transition, simultaneously. Such a flexibility is reflected in generally better $\chi^2$ values. Nevertheless, the goodness for oscillating RCs remains rather poor.

%%%%%%%%%%%%%%%%%%%%%%%%%%%%%%%%%%%%%%%%%%%%%%%%%%%%
%%%%%%%%%%%%%%%%%%%%%%%%%%%%%%%%%%%%%%%%%%%%%%%%%%%%
\section{Results}
\label{sec:results}
%%%%%%%%%%%%%%%%%%%%%%%%%%%%%%%%%%%%%%%%%%%%%%%%%%%%
%%%%%%%%%%%%%%%%%%%%%%%%%%%%%%%%%%%%%%%%%%%%%%%%%%%%

After an insightful analysis of the parameter distribution of each DM model which best fits the SPARC RCs (given in appendix \ref{sec:appendix:parameter-distribution}), we compare between them following two complementary approaches.

First, we consider the entire galaxy sample and extract the radial acceleration information for the total and baryonic components at each galactocentric radii, and put them all together as in \cref{fig:acceleration:grid} (SPARC-window). In this way, we reduce any characteristics of individual galaxies into an overall (\textit{global}) picture, i.e. the Radial Acceleration Relation \citep{2016PhRvL.117t1101M} and Mass Discrepancy Acceleration Relation \citep{2004ApJ...609..652M,2014Galax...2..601M}, respectively.

In the second approach, we consider each \textit{individual} galaxy and perform a goodness of DM-model analysis, showing how well the inferred DM RCs are fitted by these models. As typical examples, we show in \cref{fig:benchmark:total-rotation-curves} detailed rotation curve fit analyses for three selected galaxies indicating the limitations of the observational data and/or the case where data supports for one clear maximum in the rotation curve of each DM model.

\loadfigure{figure/AccelerationGrid}

%%%%%%%%%%%%%%%%%%%%%%%%%%%%%%%%%%%%%%%%%%%%%%%%%%%%
\subsection{Acceleration relations}
\label{sec:result:ac}
%%%%%%%%%%%%%%%%%%%%%%%%%%%%%%%%%%%%%%%%%%%%%%%%%%%%

% radial acceleration relation
The RC for each component (e.g. bulge, disk, gas) traces its centripetal acceleration $\SYMacc = v^2/r$, giving access to independent acceleration measurements. The Radial Acceleration Relation compares the radial acceleration due to the total mass ($\SYMatot$) with that due to the baryonic mass ($\SYMabar$). \citet{2016PhRvL.117t1101M} describes that correlation empirically by \cref{eq:mcgaugh-fit}.

% source of data
\add{For the SPARC galaxies $\SYMabar$ and $\SYMatot$ are inferred from the SPARC data. For the competing DM models $\SYMabar$ is equally inferred from the SPARC data while $\SYMatot$ is inferred from the total mass distribution: composed of the measured baryonic component (i.e SPARC data) and the given DM component (i.e. best-fit).}

% RAR/MDAR reproduction qualitatively
For the SPARC data as well as for each DM model, we applied then a least-square fitting to obtain $\SYMafrak$. The result obtained here from the SPARC data only (i.e. without assuming any specific DM model) is fully consistent (within errors) with $\SYMafrak \approx \SI{1.2E-10}{\metre\per\second^2}$ as obtained originally in \citet{2016PhRvL.117t1101M}, validating our procedure. Each DM model is then characterized by a specific best-fit $\SYMafrak$. The corresponding curves are plotted as solid lines in \cref{fig:acceleration:grid}. In all cases, we obtain values close to the one obtained in \citet{2016PhRvL.117t1101M}. This allows to conclude that all competing DM models are able to reproduce the Radial Acceleration Relation. Moreover, they reproduce it equally good without a clear statistically preferred model.

% mass discrepancy acceleration relation
A closely related relation is the MDAR relation between the baryonic and total mass components, defined by $D = M_{\rm tot}/M_{\rm bar}$.  For $a = \diff{\Phi}{r}$ and $\Phi(r)$ being the gravitational potential of a spherically symmetry mass distribution the mass discrepancy can be equivalently written as $D = \SYMatot/\SYMabar$. The results are illustrated in the bottom panels of \cref{fig:acceleration:grid}.

% beyond RAR/MDAR
According to some authors, the above two acceleration relations do not imply the need of any new physics and may be explained within the $\Lambda$CDM framework \citep{2016MNRAS.456L.127D,2017MNRAS.466.1648K,2016arXiv161208857S,2018FoPh...48.1517S}. For smaller disk and LSB galaxies (extending the original SPARC sample), \citet{2019ApJ...873..106D} found that the McGaugh et al. relation is a limiting case of a more complex relation with the need of adding one extra galaxy parameter. In addition, based on modern cosmological simulations, \citet{2017ApJ...835L..17K} predict even a redshift dependency of the acceleration parameter $\SYMafrak$, emphasizing that the correlation is universal only regarding the morphological classification.

% halo focus critics
%Moreover, both relations have been criticized that their representation focus only on the low acceleration regime where DM dominates. Therefore, \citet{2019ApJ...877...18C} suggested to compare the baryonic component $\SYMabar$, with the ratio $\SYMaDM/\SYMabar$ which gives also information about DM in the high acceleration regime, dominated by baryonic matter. They realized that the original representation of such acceleration relations shows a link on halo scales but simultaneously obscures the relation between dark and baryonic matter on inner halo scales. Therefore, any DM model (cored or cuspy) would reproduce the linear relation in the baryons dominated region. However, due to rather large uncertainties in that region, any inferred relation between dark and baryonic components would be highly speculative.

% next strategy
Additionally to the acceleration relations mentioned above (which account for the entire accelerations distributions among all galaxies and at different radii), the next strategy is to focus on the DM components of each galaxy and gather best-fits of the inferred rotation curves allowing for another (related) quantitative comparison of the DM models.
%%%%%%%%%%%%%%%%%%%%%%%%%%%%%%%%%%%%%%%%%%%%%%%%%%%%
\subsection{Goodness of model}
\label{sec:result:gof}
%%%%%%%%%%%%%%%%%%%%%%%%%%%%%%%%%%%%%%%%%%%%%%%%%%%%

The SPARC galaxies show different characteristics in their rotation curve such as a nearly flat curve through the entire galaxy data; a rising trend in the inner halo followed by a single maximum; or multiple extrema in the form of oscillations. See \cref{fig:benchmark:total-rotation-curves} for three typical examples within the SPARC data-set. Some galaxies show just a rising trend implying that the rotation curves are incomplete, likely due to the faintness and/or lack of data for outermost halo stars. Of interest is therefore a quantitative description about the goodness of a DM halo model fitting the entire galaxy sample (120 galaxies).

The goodness of a fit for a single galaxy is well described by the $\chi^2$ value, see \cref{eqn:method:chi-square}. When competing models with different number of parameters are compared it is appropriate to consider the reduced $\chi^2$ defined as $\chi_r^2 = \chi^2/d$ with the degree of freedom $d = N-p$, $N$ being the number of observables (for a single galaxy) and $p$ the number of parameters (of the considered model).

The question now arises how to compare the competing models for a population of galaxies. In order to find the goodness of a model which is robust against outliers, we ask \textit{how many} fitted galaxies have a (reduced) $\chi^2$ \textit{lower} than a given one. It turns out that the population curve resembling a cumulative distribution function (CDF) follows nearly a log-normal distribution. We use the mean value, labelled as $\hat \chi^2_r$, as the criteria to described the goodness of a model for fitting a population of galaxies within the SPARC data-set. A parameter analysis of the best-fit solutions for each DM halo model is detailed in appendix \ref{sec:appendix:parameter-distribution}. The goodness analysis is shown in \cref{fig:goodness:all} and can be summarized as:

\loadfigure{figure/GoodnessAll}
\loadfigure{figure/GoodnessWithCutoff}

\begin{asparaenum}[(i)]
    \item The NFW model ($p=2$) is statistically disfavoured with respect to the other DM halo models.

    \item The Einasto model ($p=3$) and the DC14 model ($p=3$), on the other hand, are statistically favoured with similarly good results.
    
    We remind that the DC14 model is based on the analysis of hydro-dynamically simulated galaxies including complex baryonic feedback processes \citep{2014MNRAS.441.2986D}. The good performance of DC14 may indicate the importance of baryonic feedback in galaxy formation. Indeed, our results regarding DC14 are in line as well with the literature since, as can be seen from \cref{fig:parameter-distribution:dc14} (bottom panel), the bulk of our ($\alpha,\beta,\gamma$) parameters for the SPARC data-set lies within the windows (0,2.6); (2.3,4); (0,2) in rough agreement with \citet{2014MNRAS.441.2986D}. Interestingly, we obtain a mean for the stellar to DM mass ratio $X = \log_{10}(M_*/M_\mathrm{halo})$ of $-2.4$, which is within the $X$ values ($-2.5,-2.3$) where DM cusps are most effectively flattened due to baryonic effects as reported in \citet{2014MNRAS.441.2986D}. \add{Besides, our results are in good agreement with those reported in \citet{2013ApJ...770...57B} from abundance matching, since they obtain X values between (-2.7, -1.7) for halo masses in the range between $10^{10} M_\odot$ and $10^{12} M_\odot$, with $X=-2.4$ for $M_h=10^{11} M_\odot$, the latter coinciding with the mean $X$ value in this work.}  
    
    However, the interpretation regarding the relevance (or not) of baryonic effects for Einasto profiles is more subtle. Results from hydro-dynamical (zoom-in) simulations obtained for $\sim \SI{E10}{\Msun}$ halos, show a reduction on inner-halo densities (i.e. at the convergence radius $r=\SI{0.26}{\kilo\parsec}$) of up to $\SI{45}{\percent}$ in WDM cosmologies with respect to the analogous DM-only simulations \citep{2019MNRAS.483.4086B}. A reduction which, when calibrated through Einasto profiles, is already a $15-20 \%$ more pronounced than the one typically obtained for the same WDM halos respect to the CDM ones in DM-only cosmological simulations \citep{2019MNRAS.483.4086B}. Now, for SPARC halos with total mass of few $\SI{E10}{\Msun}$ and with roughly the same Einasto scale-radius our results show that the Einasto $\alpha$ parameter required to produce the above $\sim\SI{45}{\percent}$ density reduction at $\SI{0.26}{\kilo\parsec}$ is $\alpha\approx 0.3$ (which is close to the mean value $\alpha \approx 0.4$ as shown in \cref{fig:parameter-distribution:einasto}, and only one dex above the standard $0.2$ of CDM). However it is important to mention that for the mean $\alpha$ value ($0.4$), the corresponding inner-halo density drop is so pronounced (i.e the inner-slope is almost flat, see \cref{fig:profile-illustration-mep}), that the halos look more like fermionic profiles, indicating instead a possible relevance of an underlying MEP mechanism for halo formation.

    \item The Burkert model ($p=2$) and the fermionic model ($p=3$), both produce comparable results but are somewhat statistically less favoured when the entire SPARC sample (120 galaxies) is considered. However, in the fermionic scenario the picture changes considerably for the sub-sample (44 galaxies) where data supports for a significant escape of DM particles, that is for energy-cutoff values at plateau of $W_p < 10$. For that sub-sample, the NFW model is statistically even more disfavoured while all other models are comparable, though with a little tendency for Einasto and against Burkert, see \cref{fig:goodness:with-cutoff}.
\end{asparaenum}

We would like also to point out that many galaxies in the SPARC data set are missing significant information in the outer halo (e.g. due to faint stars) or show a complex behavior (oscillatory-pattern) in their rotation curves (see e.g. right box in \cref{fig:benchmark:total-rotation-curves}). In any case, it does not allow to univocally determine the cutoff parameter (i.e. $W_p$) for the fermionic DM model since any sufficiently large $W_p$ would not change the $\chi^2$ value (see e.g. bottom left panel of \cref{fig:chi-analysis}). Nevertheless, there are some other individual galaxies where the escaping particles effects are clearly preferred. Interestingly, all of those galaxies are of magellanic type: NGC0055 (Sm), UGC05986 (Sm), UGC05750 (Sdm), UGC05005 (Im), F565-V2, (Im), UGC06399 (Sm), UGC10310 (Sm), UGC07559 (Im), UGC07690 (Im), UGC05918 (Im) and UGC05414 (Im).

Moreover, many galaxies, which are poorly fitted by any of the considered models, show \textit{short range} oscillations in their rotation curves with more than one maximum. None of the models can provide a clear explanation of that phenomena, found usually in non-magellanic galaxy types: e.g. NGC2403 (Scd), UGC02953 (Sab), NGC6015 (Scd), UGC09133 (Sab), UGC06787 (Sab), UGC11914 (Sab), NGC1003 (Scd), NGC0247 (Sd), UGC08699 (Sab) and UGC03205 (Sab).

On phenomenological grounds, in the fermionic DM model it is possible to vary the width of the maximum bump in the RC through the cutoff parameter in the strong or moderate cutoff regime ($W_p \lesssim 10$). Whether with weak ($W_p \gtrsim 10$) or even without cutoff-effects, the RC solutions of the model show long range oscillations, similar to the isothermal model. In any case, these RC oscillations have a too long wavelength and therefore do not offer a convenient explanation. On the other hand, in the case of strong cutoff, we obtain a narrow maximum bump necessary for many RCs, especially for galaxies of magellanic type (see above for examples), which usually do not show those oscillations, but also for some non-magellanic galaxy types, e.g. NGC5585 (Sd), NGC7793 (Sd), UGC06614 (Sa), ESO079-G014 (Sbc), F571-8 (Sc), NGC0891 (Sb), UGC06614 (Sa), UGC09037 (Scd), NGC4217 (Sb), UGC04278 (Sd).

NFW and Burkert models cannot explain variations of the inner and outer rotation curve because the parameters ($\beta$ and $\gamma$) responsible for such a behaviour (see \cref{eqn:hernquist}) are fixed. Additionally a transition from the inner to the outer halo is generally characterized by $\alpha$. In contrast to NFW and Burkert, the DC14 and Einasto models have a free parameter which affects the inner/outer rotation curve steepness and the sharpness of the halo transition, simultaneously. Such a flexibility is reflected in generally better $\chi^2$ values. Nevertheless, the goodness for oscillating RCs remains rather poor.

%%%%%%%%%%%%%%%%%%%%%%%%%%%%%%%%%%%%%%%%%%%%%%%%%%%%
%%%%%%%%%%%%%%%%%%%%%%%%%%%%%%%%%%%%%%%%%%%%%%%%%%%%
\section{Results}
\label{sec:results}
%%%%%%%%%%%%%%%%%%%%%%%%%%%%%%%%%%%%%%%%%%%%%%%%%%%%
%%%%%%%%%%%%%%%%%%%%%%%%%%%%%%%%%%%%%%%%%%%%%%%%%%%%

After an insightful analysis of the parameter distribution of each DM model which best fits the SPARC RCs (given in appendix \ref{sec:appendix:parameter-distribution}), we compare between them following two complementary approaches.

First, we consider the entire galaxy sample and extract the radial acceleration information for the total and baryonic components at each galactocentric radii, and put them all together as in \cref{fig:acceleration:grid} (SPARC-window). In this way, we reduce any characteristics of individual galaxies into an overall (\textit{global}) picture, i.e. the Radial Acceleration Relation \citep{2016PhRvL.117t1101M} and Mass Discrepancy Acceleration Relation \citep{2004ApJ...609..652M,2014Galax...2..601M}, respectively.

In the second approach, we consider each \textit{individual} galaxy and perform a goodness of DM-model analysis, showing how well the inferred DM RCs are fitted by these models. As typical examples, we show in \cref{fig:benchmark:total-rotation-curves} detailed rotation curve fit analyses for three selected galaxies indicating the limitations of the observational data and/or the case where data supports for one clear maximum in the rotation curve of each DM model.

\input{content/result.accelerationRelation.tex}
\input{content/result.goodness}
\input{content/result.polytropicHalos}
\input{content/result.limitations.tex}
%%%%%%%%%%%%%%%%%%%%%%%%%%%%%%%%%%%%%%%%%%%%%%%%%%%%
\subsection[Best-fit analysis]{Diversity of SPARC rotation curves}
% Limitation of the observational data
\label{sec:result:limitations}
%%%%%%%%%%%%%%%%%%%%%%%%%%%%%%%%%%%%%%%%%%%%%%%%%%%%

We show in this section a detailed $\chi^2$ analysis of the RC fits for three selected galaxies, each representing some characteristics of given observational data. We divide the SPARC galaxies in three groups by the inferred DM component as explained next. This analysis is based on the fermionic model where such a grouping seems to be appropriate to select galaxies with valuable predictions about the inner halo.

The first group, represented by UGC05986, shows only a single maximum in its DM RC, i.e. a rising trend in the inner halo followed by a clear turning point, see left plots in \cref{fig:benchmark:total-rotation-curves}. This typical profile is \add{better} fitted by the fermionic model with a significant escape of particles ($W_p \ll 1$) , see \add{the valley in} top panels of \cref{fig:chi-analysis}. However, due to the lack of information in the inner halo structures there is some uncertainty in the strength of particle escape. The uncertainty is physically better reflected in the core mass $M_c$ which covers about two orders of magnitude (see middle panel of first row in \cref{fig:chi-analysis}).

\loadfigure{figure/BenchmarkTotalRotationCurves}
%\loadfigure{figure/BenchmarkDarkMatterProfiles}
\loadfigure{figure/Chi2Analysis}

This result goes totally in line with an analogous phenomenological analysis \citep{2019PDU....24..278A}, developed for typical dwarf, spiral and elliptical galaxies within the RAR model. According to that analysis (done for $mc^2\approx \SI{50}{\kilo\eV}$), the maximal core mass of larger galaxies is limited by the critical configuration where the quantum core becomes unstable and collapses to a BH of mass $M_c^{cr} \approx \SI{2E8}{\Msun}$. %For the smaller galaxies like dwarfs the maximal core mass is limited by the so called \textit{halo-deficit} solution before the core becomes unstable. Such solutions are characterized by large escape of particles that results in the development of a cuspy halo. The analysis of this paper shows that such cuspy halos within the MEP model are clearly discarded for galaxies of the SPARC data set.

Among the cases, which are disfavored, are the ones with very large total DM masses $M_s$ corresponding to isothermal-like halos and implying negligible escape of particles ($W_p \gtrsim 10$). These solutions provide a minimal core mass $M_c$ with a huge uncertainty in the total mass. %Therefore, the first group represented by UGC05986 favors cored solutions with \textit{mild} surface effects due to particle escape.

The second group, represented by DDO161, shows a rising part in the RC towards a maximum without a clear turning point compared to the first group, see central plots in \cref{fig:benchmark:total-rotation-curves}. Fitting those galaxies for different $W_p$ values does not favor solutions with or without escaping particles effects. The variation in the $\chi^2$ value remains rather small, see middle panels of \cref{fig:chi-analysis}.

%Nevertheless, clearly discarded are cuspy halos (i.e. \textit{halo-deficit} solutions) just as in the first group. There is a narrow $\chi^2$ minimum for relatively low $W_0$ values suggesting a best-fit. However, this result should be taken with caution because the obtained minimum depends on the inner data points, keeping in mind that for most galaxies in the SPARC data set the inner data points have a relatively high uncertainty. This is also the case for DDO161. Galaxies falling in the second group show a similar behavior as in the first group represented by UGC05986 with the difference that isothermal-like halos are not necessary discarded due to a lack of information in the outer halo.

Finally, the third group, represented by NGC6015, shows some oscillations in the RC, mainly in the outer halo, see right plots in \cref{fig:benchmark:total-rotation-curves}. There are various and speculative reasons for the oscillation, e.g., ongoing merging process, deviation from equilibrium, etc. In any case, those galaxies are clearly better fitted by extended isothermal-like halos ($W_p \gtrsim 10$) --- although being far from good --- see bottom panels of \cref{fig:chi-analysis}. Such solutions provide a wide halo maximum followed by a flat RC. In contrast, the non-isothermal solutions with a cutoff provide only a narrow maximum in the halo, followed by a Keplerian decreasing tail.

It is worth to recall that different DM models such as the fermionic model, NFW or DC14 are not appropriate to fit the oscillations, characterized through multiple maxima in the RC. All solutions with a wide halo are suitable to fit the oscillations well on average, although the best-fits remain rather poor, leaving almost no insight into the physical properties of DM on halo scales for those galaxies.

%%%%%%%%%%%%%%%%%%%%%%%%%%%%%%%%%%%%%%%%%%%%%%%%%%%%
\subsection[Best-fit analysis]{Diversity of SPARC rotation curves}
% Limitation of the observational data
\label{sec:result:limitations}
%%%%%%%%%%%%%%%%%%%%%%%%%%%%%%%%%%%%%%%%%%%%%%%%%%%%

We show in this section a detailed $\chi^2$ analysis of the RC fits for three selected galaxies, each representing some characteristics of given observational data. We divide the SPARC galaxies in three groups by the inferred DM component as explained next. This analysis is based on the fermionic model where such a grouping seems to be appropriate to select galaxies with valuable predictions about the inner halo.

The first group, represented by UGC05986, shows only a single maximum in its DM RC, i.e. a rising trend in the inner halo followed by a clear turning point, see left plots in \cref{fig:benchmark:total-rotation-curves}. This typical profile is \add{better} fitted by the fermionic model with a significant escape of particles ($W_p \ll 1$) , see \add{the valley in} top panels of \cref{fig:chi-analysis}. However, due to the lack of information in the inner halo structures there is some uncertainty in the strength of particle escape. The uncertainty is physically better reflected in the core mass $M_c$ which covers about two orders of magnitude (see middle panel of first row in \cref{fig:chi-analysis}).

\loadfigure{figure/BenchmarkTotalRotationCurves}
%\loadfigure{figure/BenchmarkDarkMatterProfiles}
\loadfigure{figure/Chi2Analysis}

This result goes totally in line with an analogous phenomenological analysis \citep{2019PDU....24..278A}, developed for typical dwarf, spiral and elliptical galaxies within the RAR model. According to that analysis (done for $mc^2\approx \SI{50}{\kilo\eV}$), the maximal core mass of larger galaxies is limited by the critical configuration where the quantum core becomes unstable and collapses to a BH of mass $M_c^{cr} \approx \SI{2E8}{\Msun}$. %For the smaller galaxies like dwarfs the maximal core mass is limited by the so called \textit{halo-deficit} solution before the core becomes unstable. Such solutions are characterized by large escape of particles that results in the development of a cuspy halo. The analysis of this paper shows that such cuspy halos within the MEP model are clearly discarded for galaxies of the SPARC data set.

Among the cases, which are disfavored, are the ones with very large total DM masses $M_s$ corresponding to isothermal-like halos and implying negligible escape of particles ($W_p \gtrsim 10$). These solutions provide a minimal core mass $M_c$ with a huge uncertainty in the total mass. %Therefore, the first group represented by UGC05986 favors cored solutions with \textit{mild} surface effects due to particle escape.

The second group, represented by DDO161, shows a rising part in the RC towards a maximum without a clear turning point compared to the first group, see central plots in \cref{fig:benchmark:total-rotation-curves}. Fitting those galaxies for different $W_p$ values does not favor solutions with or without escaping particles effects. The variation in the $\chi^2$ value remains rather small, see middle panels of \cref{fig:chi-analysis}.

%Nevertheless, clearly discarded are cuspy halos (i.e. \textit{halo-deficit} solutions) just as in the first group. There is a narrow $\chi^2$ minimum for relatively low $W_0$ values suggesting a best-fit. However, this result should be taken with caution because the obtained minimum depends on the inner data points, keeping in mind that for most galaxies in the SPARC data set the inner data points have a relatively high uncertainty. This is also the case for DDO161. Galaxies falling in the second group show a similar behavior as in the first group represented by UGC05986 with the difference that isothermal-like halos are not necessary discarded due to a lack of information in the outer halo.

Finally, the third group, represented by NGC6015, shows some oscillations in the RC, mainly in the outer halo, see right plots in \cref{fig:benchmark:total-rotation-curves}. There are various and speculative reasons for the oscillation, e.g., ongoing merging process, deviation from equilibrium, etc. In any case, those galaxies are clearly better fitted by extended isothermal-like halos ($W_p \gtrsim 10$) --- although being far from good --- see bottom panels of \cref{fig:chi-analysis}. Such solutions provide a wide halo maximum followed by a flat RC. In contrast, the non-isothermal solutions with a cutoff provide only a narrow maximum in the halo, followed by a Keplerian decreasing tail.

It is worth to recall that different DM models such as the fermionic model, NFW or DC14 are not appropriate to fit the oscillations, characterized through multiple maxima in the RC. All solutions with a wide halo are suitable to fit the oscillations well on average, although the best-fits remain rather poor, leaving almost no insight into the physical properties of DM on halo scales for those galaxies.

%%%%%%%%%%%%%%%%%%%%%%%%%%%%%%%%%%%%%%%%%%%%%%%%%%%%
\subsection[Best-fit analysis]{Diversity of SPARC rotation curves}
% Limitation of the observational data
\label{sec:result:limitations}
%%%%%%%%%%%%%%%%%%%%%%%%%%%%%%%%%%%%%%%%%%%%%%%%%%%%

We show in this section a detailed $\chi^2$ analysis of the RC fits for three selected galaxies, each representing some characteristics of given observational data. We divide the SPARC galaxies in three groups by the inferred DM component as explained next. This analysis is based on the fermionic model where such a grouping seems to be appropriate to select galaxies with valuable predictions about the inner halo.

The first group, represented by UGC05986, shows only a single maximum in its DM RC, i.e. a rising trend in the inner halo followed by a clear turning point, see left plots in \cref{fig:benchmark:total-rotation-curves}. This typical profile is \add{better} fitted by the fermionic model with a significant escape of particles ($W_p \ll 1$) , see \add{the valley in} top panels of \cref{fig:chi-analysis}. However, due to the lack of information in the inner halo structures there is some uncertainty in the strength of particle escape. The uncertainty is physically better reflected in the core mass $M_c$ which covers about two orders of magnitude (see middle panel of first row in \cref{fig:chi-analysis}).

\loadfigure{figure/BenchmarkTotalRotationCurves}
%\loadfigure{figure/BenchmarkDarkMatterProfiles}
\loadfigure{figure/Chi2Analysis}

This result goes totally in line with an analogous phenomenological analysis \citep{2019PDU....24..278A}, developed for typical dwarf, spiral and elliptical galaxies within the RAR model. According to that analysis (done for $mc^2\approx \SI{50}{\kilo\eV}$), the maximal core mass of larger galaxies is limited by the critical configuration where the quantum core becomes unstable and collapses to a BH of mass $M_c^{cr} \approx \SI{2E8}{\Msun}$. %For the smaller galaxies like dwarfs the maximal core mass is limited by the so called \textit{halo-deficit} solution before the core becomes unstable. Such solutions are characterized by large escape of particles that results in the development of a cuspy halo. The analysis of this paper shows that such cuspy halos within the MEP model are clearly discarded for galaxies of the SPARC data set.

Among the cases, which are disfavored, are the ones with very large total DM masses $M_s$ corresponding to isothermal-like halos and implying negligible escape of particles ($W_p \gtrsim 10$). These solutions provide a minimal core mass $M_c$ with a huge uncertainty in the total mass. %Therefore, the first group represented by UGC05986 favors cored solutions with \textit{mild} surface effects due to particle escape.

The second group, represented by DDO161, shows a rising part in the RC towards a maximum without a clear turning point compared to the first group, see central plots in \cref{fig:benchmark:total-rotation-curves}. Fitting those galaxies for different $W_p$ values does not favor solutions with or without escaping particles effects. The variation in the $\chi^2$ value remains rather small, see middle panels of \cref{fig:chi-analysis}.

%Nevertheless, clearly discarded are cuspy halos (i.e. \textit{halo-deficit} solutions) just as in the first group. There is a narrow $\chi^2$ minimum for relatively low $W_0$ values suggesting a best-fit. However, this result should be taken with caution because the obtained minimum depends on the inner data points, keeping in mind that for most galaxies in the SPARC data set the inner data points have a relatively high uncertainty. This is also the case for DDO161. Galaxies falling in the second group show a similar behavior as in the first group represented by UGC05986 with the difference that isothermal-like halos are not necessary discarded due to a lack of information in the outer halo.

Finally, the third group, represented by NGC6015, shows some oscillations in the RC, mainly in the outer halo, see right plots in \cref{fig:benchmark:total-rotation-curves}. There are various and speculative reasons for the oscillation, e.g., ongoing merging process, deviation from equilibrium, etc. In any case, those galaxies are clearly better fitted by extended isothermal-like halos ($W_p \gtrsim 10$) --- although being far from good --- see bottom panels of \cref{fig:chi-analysis}. Such solutions provide a wide halo maximum followed by a flat RC. In contrast, the non-isothermal solutions with a cutoff provide only a narrow maximum in the halo, followed by a Keplerian decreasing tail.

It is worth to recall that different DM models such as the fermionic model, NFW or DC14 are not appropriate to fit the oscillations, characterized through multiple maxima in the RC. All solutions with a wide halo are suitable to fit the oscillations well on average, although the best-fits remain rather poor, leaving almost no insight into the physical properties of DM on halo scales for those galaxies.
