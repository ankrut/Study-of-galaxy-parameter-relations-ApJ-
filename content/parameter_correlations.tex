%%%%%%%%%%%%%%%%%%%%%%%%%%%%%%%%%%%%%%%%%%%%%%%%%%%%
%%%%%%%%%%%%%%%%%%%%%%%%%%%%%%%%%%%%%%%%%%%%%%%%%%%%
\section{Parameter correlations of fermionic core-halo solutions}
\label{sec:parameter-correlations}
%%%%%%%%%%%%%%%%%%%%%%%%%%%%%%%%%%%%%%%%%%%%%%%%%%%%
%%%%%%%%%%%%%%%%%%%%%%%%%%%%%%%%%%%%%%%%%%%%%%%%%%%%

We analyze different pairs of structural galaxy parameters, obtained from core-halo best-fit solutions of the fermionic model for a particle mass of $\SI{50}{\kilo\eV}$. Of interest are values at the core such as the core mass $M_c = M(r_c)$ and at the halo such as the halo radius $r_h$ and halo mass $M_h = M(r_h)$. The core radius $r_c$ is defined at the first maximum in the rotation curve and the halo radius $r_h$ is defined at the second maximum.

% boundary mass definition
Further, we are also interested in the total DM mass $M_{\rm tot}$ what requires a careful definition here. Usually, mass distributions infinite in mass and space are truncated for example at a critical radius or density to obtain reasonable values. In that fashion, we define the \textit{boundary mass} $M_b = M(r_b)$, being $r_b$ the boundary radius where the density falls to the critical density of the Local Group (about $\SI{E-5}{\Msun\per\parsec^3}$).

% surface mass definition
On contrary, fermionic DM exposed to particle escaping (i.e. evaporation) provides naturally finite mass distributions. Thus, we define the \textit{surface mass} $M_s = M(r_s)$, being $r_s$ the natural surface radius where the density falls to zero. For a large escape of particles ($W_p \ll 10$) we have total masses $M_b \approx M_s$ while for a negligible escape of particles ($W_p \gg 10$) the mass distribution grows in extent and the boundary radius must be imposed such that $M_b \ll M_s$.

\loadfigure{figure/ParameterCorrelations}

% obtained galaxy parameter range
Regarding the halo, we obtain the radii mainly in the interval $[10^3,10^5]\si{\parsec}$ and masses in the interval $[10^8,10^{12}] \si{\Msun}$. As shown in \cref{fig:parameter-correlction:core-halo} the halo radius and mass follow a clear relation described by \begin{equation}
	\label{eqn:rel:Mh-rh}
	\ln \qbracket{\frac{M_h}{M_\odot}} \approx 2\ln \qbracket{\frac{r_h}{\mathrm{pc}}} + 5.8 \pm 1.7
\end{equation} Interestingly, this relation is independent of particle escape, i.e it holds for solutions with an isothermal-like halo developing a flat tail as well as for polytropic non-isothermal halos implying a large escape of particles.

%%%%%%%%%%%%%%%%%%%%%%%%%%%%%%%%%%%%%%%%%%%%%%%%%%%%
%%%%%%%%%%%%%%%%%%%%%%%%%%%%%%%%%%%%%%%%%%%%%%%%%%%%
\subsection{DM surface density relation}
\label{sec:dark-matter-surface-density}
%%%%%%%%%%%%%%%%%%%%%%%%%%%%%%%%%%%%%%%%%%%%%%%%%%%%
%%%%%%%%%%%%%%%%%%%%%%%%%%%%%%%%%%%%%%%%%%%%%%%%%%%%

We focus now on the constant surface density \citep{2009MNRAS.397.1169D} 
%
\begin{equation}
\label{eqn:Donato}
	\Sigma_{0D} = \rho_{\rm 0D} r_0 \approx 140_{-50}^{+80} \si{\Msun\per\parsec^2}.
\end{equation} 
%
This value is valid for about 14 orders of magnitude in absolute magnitude ($M_B$) where $\rho_0$ is the \textit{central} DM halo density and $r_0$ the one-halo-scale-length, both of the Burkert model. At $r_0$ the density falls to one-forth of the central density, i.e. $\rho(r_0) = \rho_{\rm 0D}/4$.

% parameter correspondence
Note that the \textit{center} in the Burkert model corresponds to the plateau in the fermionic DM model, i.e. $\rho_{\rm 0D} \approx \rho_{\rm p}$ where the plateau density $\rho_{\rm p}$ is defined at the first minimum in the RC. Following the definition of the Burkert radius $r_0$, we identify the one-halo-scale-length $r_B$ of the fermionic model such that $\rho(r_B) = \rho_p/4$. We thus calculate the product $\rho_p r_B$ for each galaxy.

% absolute magnitude (from CGS)
The absolute magnitude $M_B$ was taken from the Carnegie-Irvine Galaxy Survey \citep{2011ApJS..197...21H}, providing nine overlapping galaxies with the SPARC sample. These candidates are well in agreement with the DM surface density observations, see \cref{fig:SPARC:Donato}. The shown candidates include isothermal-like (blue outlined points) and non-isothermal (green circles) solutions. For comparison, the results are amended by the MW solution following the fermionic RAR model \citep{2018PDU....21...82A}.

\loadfigure{figure/Donato}

% histogram
Although absolute magnitude information is incomplete, all of the predicted DM surface densities are within the range of the $3\sigma$ area as well. This is visualised by a histogram for the full sample (dark grey bars, 120 galaxies) with comparison to the sub-sample (green bars, 44 galaxies) including non-isothermal solutions (i.e. $W_p < 10$). Considering the sub-sample only, we obtain a mean surface density of about $\SI{148.5}{\Msun/\parsec^2}$, fully inside the error bars in \cref{eqn:Donato}.

% equivalent relation
The surface density relation given by \cref{eqn:Donato} is qualitatively consistent with the scaling relation $M_h \sim r_h^2$ as given by \cref{eqn:rel:Mh-rh}. Due to the different halo profiles of the fermionic model, ranging from polytropes with $n=5/2$ to isothermal (see section \ref{sec:morph}), there is a non-linear relation between the halo radius $r_h$ and the one-halo-scale-length $r_B$. Nevertheless, considering that the halo is nearly homogeneous up to approximately the halo radius $r_h$ (i.e. $M_h \sim \rho_p r_h^3$), and that $r_h \sim r_B$ we obtain $\rho_p r_B \approx \const$. (see also \citealp{2019PDU....24..278A}).

%%%%%%%%%%%%%%%%%%%%%%%%%%%%%%%%%%%%%%%%%%%%%%%%%%%%
%%%%%%%%%%%%%%%%%%%%%%%%%%%%%%%%%%%%%%%%%%%%%%%%%%%%
\subsection{Central core vs. total halo mass relation}
\label{sec:parameter-corelation:ferrarese}
%%%%%%%%%%%%%%%%%%%%%%%%%%%%%%%%%%%%%%%%%%%%%%%%%%%%
%%%%%%%%%%%%%%%%%%%%%%%%%%%%%%%%%%%%%%%%%%%%%%%%%%%%

\loadfigure{figure/Ferrarese}

Finally, we turn to the $M_{\rm BH}$-$M_{\rm tot}$ relation \citep{2002ApJ...578...90F,2011Natur.469..377K,2015ApJ...800..124B} where $M_{\rm tot}$ is the total DM halo mass and $M_{\rm BH}$ is the mass of the compact dark object at the center of galaxies. Traditionally, the central dark objects are assumed as SMBHs but here interpreted as DM quantum cores in the case of inactive galaxies. In the following we consider $M_{\rm BH} = M_c$, being $M_c = M(r_c)$ the quantum core mass. For non-isothermal solutions ($W_p \lesssim 10$) we take advantage of the natural benefits of fermionic DM and consider the total mass $M_{\rm tot} = M_s$ without imposing any arbitrary boundary condition. For isothermal solutions ($W_p \gtrsim 10$) we must impose the boundary radius and consider $M_{\rm tot} = M_b$.

With those definitions of core and total DM mass, \citet{2019PDU....24..278A} showed that fermionic DM according to the RAR model is able to explain the $M_{\rm BH} - M_{\rm tot}$ relation by analyzing typical galaxies ranging from dwarfs up to ellipticals and BCGs. Here, we extend the results with predictions inferred from disk galaxies of the SPARC data set and compare them with the solutions for the Milky Way galaxy \citep{2018PDU....21...82A}. The results are illustrated in \cref{fig:SPARC:Ferrarese}.

% group division criteria
Of great interest are non-isothermal (polytropic-like) solutions predicting a significant escape of particles ($W_p \lesssim 10$), implying total masses naturally below few $\SI{E12}{\Msun}$. For the sake of completeness, we show also the isothermal solutions ($W_p \gtrsim 10$), showing a correlation of the form $M_c \propto M_h^{0.6}$. A power law correlation $M_c \propto M_h^{1/2}$ very close to the one found here on phenomenological grounds, can be derived from pure thermodynamical arguments within a Fermi-Dirac phase space distribution as done in \cite{2019PhRvD.100l3506C}. Even if one should take these isothermal branch of solutions with caution due to insufficient data in the outer halo (i.e. it does not allow to safely constrain $W_p$) it is quite remarkable the almost perfect match between the theoretical and numerical correlations.

% mass ranges
The majority of the best-fit solutions has a total DM mass between $\SI{E9}{\Msun}$ and $\SI{E12}{\Msun}$. Only a couple of  candidates have few $\SI{E12}{\Msun}$. The core mass spans a range between $\SI{E4}{\Msun}$ and $\SI{E7}{\Msun}$. Note that there is some uncertainty in the core mass $M_c$ up to about two orders of magnitude due to insufficient data in the inner halo. See section \ref{sec:result:limitations} for details.

The fermionic best-fit solutions for disk galaxies of the SPARC data-set together with the Milky Way (MW) solution \citep{2018PDU....21...82A} fit very well in an overall picture. This is reflected by the green region in \cref{fig:SPARC:Ferrarese} which covers all predictions for a given fermionic halo mass in the range $M_h \approx \SIrange{E7}{E12}{\Msun}$ and fulfilling $\rho_p r_B \approx \SI{140}{\Msun\per\parsec^2}$ as inferred from the Donato relation. In contrast to the SPARC galaxies the total DM mass of the MW and especially its central compact core are well constraint, indicating that SPARC galaxies with a comparable core mass are plausible and of astrophysical interest despite relatively large uncertainties.

Comparing our results for SPARC galaxies with the analysis of larger galaxies, it indicates a transition from (similar) $M_{\rm BH}$-$M_{\rm tot}$ relations as found by \citet{2002ApJ...578...90F,2015ApJ...800..124B} for $M_{\rm tot} \gtrsim \SI{E11}{\Msun}$, into a region with a larger diversity of core masses $M_c$ for $M_{\rm tot} \lesssim \SI{E11}{\Msun}$ as shown in \cref{fig:SPARC:Ferrarese}. This suggest that the dark central objects in smaller galaxies do not correlate well with their hosting halos.

