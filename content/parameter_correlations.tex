%%%%%%%%%%%%%%%%%%%%%%%%%%%%%%%%%%%%%%%%%%%%%%%%%%%%
%%%%%%%%%%%%%%%%%%%%%%%%%%%%%%%%%%%%%%%%%%%%%%%%%%%%
\section{Parameter correlations of fermionic halos}
\label{sec:parameter-correlations}
%%%%%%%%%%%%%%%%%%%%%%%%%%%%%%%%%%%%%%%%%%%%%%%%%%%%
%%%%%%%%%%%%%%%%%%%%%%%%%%%%%%%%%%%%%%%%%%%%%%%%%%%%

We analyzed different pairs of structural galaxy parameters, obtained from core-halo best-fit solutions of the fermionic model for a particle mass of $\SI{50}{\kilo\eV}$.

Of interest here are values at the halo such as the halo radius $r_h$ and halo mass $M_h = M(r_h)$. The halo radius $r_h$ is defined at the second maximum in the rotation curve $v(r)$.

% obtained galaxy parameter range
For mass and radius we obtain values located mainly in the intervals $r_h \in [10^3,10^5]\si{\parsec}$ and $M_h \in [10^8,10^{12}] \si{\Msun}$. As shown in \cref{fig:parameter-correlction:core-halo} the halo radius and mass follow a clear relation described by \begin{equation}
	\label{eqn:rel:Mh-rh}
	\ln \qbracket{\frac{M_h}{M_\odot}} \approx 2\ln \qbracket{\frac{r_h}{\mathrm{pc}}} + 5.8 \pm 1.7
\end{equation} Interestingly, this relation is independent of particle escape, i.e it holds for solutions with an isothermal-like halo developing a flat tail as well as for polytropic non-isothermal halos implying a large escape of particles.

\loadfigure{figure/ParameterCorrelations}

