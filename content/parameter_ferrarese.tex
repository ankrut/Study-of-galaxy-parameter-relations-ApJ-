%%%%%%%%%%%%%%%%%%%%%%%%%%%%%%%%%%%%%%%%%%%%%%%%%%%%
%%%%%%%%%%%%%%%%%%%%%%%%%%%%%%%%%%%%%%%%%%%%%%%%%%%%
\subsection{Central core vs. total halo mass relation}
\label{sec:parameter-corelation:ferrarese}
%%%%%%%%%%%%%%%%%%%%%%%%%%%%%%%%%%%%%%%%%%%%%%%%%%%%
%%%%%%%%%%%%%%%%%%%%%%%%%%%%%%%%%%%%%%%%%%%%%%%%%%%%

\loadfigure{figure/Ferrarese}

Finally, we turn to the $M_{\rm BH}$-$M_{\rm tot}$ relation \citep{2002ApJ...578...90F,2011Natur.469..377K,2015ApJ...800..124B} where $M_{\rm tot}$ is the total DM halo mass and $M_{\rm BH}$ is the mass of the compact dark object at the center of galaxies. Traditionally, the central dark objects are assumed as SMBHs but here interpreted as DM quantum cores in the case of inactive galaxies. In the following we consider $M_{\rm BH} = M_c$, being $M_c = M(r_c)$ the quantum core mass. For non-isothermal solutions ($W_p \lesssim 10$) we take advantage of the natural benefits of fermionic DM and consider the total mass $M_{\rm tot} = M_s$ without imposing any arbitrary boundary condition. For isothermal solutions ($W_p \gtrsim 10$) we must impose the boundary radius and consider $M_{\rm tot} = M_b$.

With those definitions of core and total DM mass, \citet{2019PDU....24..278A} showed that fermionic DM according to the RAR model is able to explain the $M_{\rm BH} - M_{\rm tot}$ relation by analyzing typical galaxies ranging from dwarfs up to ellipticals and BCGs. Here, we extend the results with predictions inferred from disk galaxies of the SPARC data set and compare them with the solutions for the Milky Way galaxy \citep{2018PDU....21...82A}. The results are illustrated in \cref{fig:SPARC:Ferrarese}.

% group division criteria
Of great interest are non-isothermal (polytropic-like) solutions predicting a significant escape of particles ($W_p \lesssim 10$), implying total masses naturally below few $\SI{E12}{\Msun}$. For the sake of completeness, we show also the isothermal solutions ($W_p \gtrsim 10$), showing a correlation of the form $M_c \propto M_h^{0.6}$. A power law correlation $M_c \propto M_h^{1/2}$ very close to the one found here on phenomenological grounds, can be derived from pure thermodynamical arguments within a Fermi-Dirac phase space distribution as done in \cite{2019PhRvD.100l3506C}. Even if one should take these isothermal branch of solutions with caution due to insufficient data in the outer halo (i.e. it does not allow to safely constrain $W_p$) it is quite remarkable the almost perfect match between the theoretical and numerical correlations.

% mass ranges
The majority of the best-fit solutions has a total DM mass between $\SI{E9}{\Msun}$ and $\SI{E12}{\Msun}$. Only a couple of  candidates have few $\SI{E12}{\Msun}$. The core mass spans a range between $\SI{E4}{\Msun}$ and $\SI{E7}{\Msun}$. Note that there is some uncertainty in the core mass $M_c$ up to about two orders of magnitude due to insufficient data in the inner halo. See section \ref{sec:result:limitations} for details.

The fermionic best-fit solutions for disk galaxies of the SPARC data-set together with the Milky Way (MW) solution \citep{2018PDU....21...82A} fit very well in an overall picture. This is reflected by the green region in \cref{fig:SPARC:Ferrarese} which covers all predictions for a given fermionic halo mass in the range $M_h \approx \SIrange{E7}{E12}{\Msun}$ and fulfilling $\rho_p r_B \approx \SI{140}{\Msun\per\parsec^2}$ as inferred from the Donato relation. In contrast to the SPARC galaxies the total DM mass of the MW and especially its central compact core are well constraint, indicating that SPARC galaxies with a comparable core mass are plausible and of astrophysical interest despite relatively large uncertainties.

Comparing our results for SPARC galaxies with the analysis of larger galaxies, it indicates a transition from (similar) $M_{\rm BH}$-$M_{\rm tot}$ relations as found by \citet{2002ApJ...578...90F,2015ApJ...800..124B} for $M_{\rm tot} \gtrsim \SI{E11}{\Msun}$, into a region with a larger diversity of core masses $M_c$ for $M_{\rm tot} \lesssim \SI{E11}{\Msun}$ as shown in \cref{fig:SPARC:Ferrarese}. This suggest that the dark central objects in smaller galaxies do not correlate well with their hosting halos.
