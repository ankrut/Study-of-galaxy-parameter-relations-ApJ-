%%%%%%%%%%%%%%%%%%%%%%%%%%%%%%%%%%%%%%%%%%%%%%%%%%%%
\subsection{Baryonic feedback}
\label{sec:baryonic-effect}
%%%%%%%%%%%%%%%%%%%%%%%%%%%%%%%%%%%%%%%%%%%%%%%%%%%%

\add{The DM in galaxies evolves together with baryons, thus it is expected some degree of baryonic feedback on scales where baryons dominate. This also seems to be the case within the SPARC galaxies here considered, where almost half of the points in upper-left panel of Fig. \ref{fig:acceleration:grid} 
roughly fulfill $\SYMatot < 2 \SYMabar$. One of the main baryonic effects onto the total gravitational potential is thought to happen due to a sustained process of stellar bursts, which drives baryonic material from inner-halo regions while ending in a reduction of DM densities at those halo scales (see e.g. \citealp{2012MNRAS.422.1231G,2016MNRAS.459.2573R}). However, the quantification of such baryonic effects have always been calibrated and applied to cuspy CDM halos. Only recently it was shown that baryonic feedback is DM-model dependent.} 

\add{That is, in WDM cosmologies such effects are typically diminished with respect to CDM \citep{2019MNRAS.483.4086B}. The main reasons behind this attenuation are that in WDM cosmologies DM halos form later, are less centrally dense on inner-halo scales, and therefore contain galaxies that are less massive with less baryon content than the CDM counterparts. Since the halos of the Burkert, Einasto and fermionic DM model correspond to WDM cosmologies, it is expected that baryonic effects are milder in those cases. However, a thorough quantification of this feedback has only been worked out for Einasto profiles \citep{2019MNRAS.483.4086B} and still remains to be solved in the case of fermionic halos before taking any conclusion, though out of the scope of this work.}

\add{In any case, and regardless of a potential feedback of the baryons onto the fermionic DM profile, it is important to emphasize a key result of this work: the flatness in the inner-halo slope of MEP profiles is due to a (quasi) thermodynamic equilibrium reached by the DM particles, thus involving a different physical principle than the one explained above for the baryons.}
%one of the main outcomes of this paper regarding halos formed through an MEP scenario, different physical principles justify the flat inner halo slope in the most favored DM profiles: while generalized NFW or Einasto models rely on complex baryonic feedback processes, the MEP scenario involves a quasi-thermodynamic equilibrium of the DM particles. [REPHRASE this last concept.]

\add{Finally, two DM model considered here --- DC14 \citep{2014MNRAS.441.2986D} and Einasto \citep{2019MNRAS.483.4086B} --- (effectively) account for baryonic feedback. That is, any baryonic feedback expected to arise for SPARC galaxies should be reflected in the best-fit parameters of the DM profiles with its consequent universal shape reflecting such effects. In section \ref{sec:result:gof} we compare our statistical results regarding baryonic effects in the DM model free-parameters with the ones reported in the literature.}

%DM halos within the MEP-formation scenario share this properties because they belong to WDM cosmologies --- fermion particle masses are in the keV range -- and are certainly less dense than in the CDM case since they develop a plateau on the inner-halo \citep{2021MNRAS.502.4227A}. Thus, baryonic feedback effects in the WDM models here studied are expected to be milder than in CDM ones.

%Even when some baryonic effects are present as to cause resulting DM profiles with somewhat lower inner-halo densities (as for DC14 and Einasto), these effects can be thought to be indirectly (or effectively) accounted in our statistical analysis since we perform parametric fits. \add{}.

%Interestingly \cite{2019MNRAS.483.4086B} reports, in DM-only simulations, a reduction in WDM inner-halo density relative to CDM ($\rho^{Ein}_{\rm WDM}/\rho^{Ein}_{\rm CDM}(r_{in})$) of about $0.7$, which is consistent with an Einasto concentration parameter in the WDM case of about $10$ \citep{2019MNRAS.483.4086B. Even when no explicit reference to the $\kappa$ parameter is there given for such a concentration, it must be somewhat larger than $\kappa=0.2$ in order to cause the above density-reduction (recall $\kappa=0.2$ is a best fit parameter to standard CDM halos: see e.g. fig. 2 in \citealp{2014MNRAS.441.3359D}). Our results are then consistent with these values for the Einasto case (see \cref{fig:parameter-distribution:einasto})