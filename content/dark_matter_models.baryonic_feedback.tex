%%%%%%%%%%%%%%%%%%%%%%%%%%%%%%%%%%%%%%%%%%%%%%%%%%%%
\subsection{Baryonic feedback}
\label{sec:baryonic-effect}
%%%%%%%%%%%%%%%%%%%%%%%%%%%%%%%%%%%%%%%%%%%%%%%%%%%%

While some of the models here considered (effectively) account for baryonic feedback effects such as the DC14 model \citep{2014MNRAS.441.2986D} and the Einasto profile \citep{2019MNRAS.483.4086B}, the others do not. However it is important to recall that the baryonic feedback is DM-model dependent, and, for example, in \add{warm DM} (WDM) cosmologies such effects are typically diminished with respect to CDM \citep{2019MNRAS.483.4086B}. The main reasons behind this attenuation are that in WDM cosmologies DM halos form later, are less centrally dense on inner-halo scales, and therefore contain galaxies that are less massive than the CDM counterparts. DM halos within the MEP-formation scenario share this properties because they belong to WDM cosmologies (the fermion masses are in the keV range), and are certainly less dense than in the CDM case since they develop a plateau on the inner-halo \citep{2021MNRAS.502.4227A}. Thus baryonic feedback effects in the WDM models here studied are expected to be milder than in CDM ones.
    
Even when some baryonic effects are present as to cause resulting DM profiles with somewhat lower inner-halo densities (as for DC14 and Einasto), these effects can be thought to be indirectly (or effectively) accounted in our statistical analysis since we perform parametric fits. That is, for DM profiles whose universal shape is expected to account already for some baryonic feedback (such as Einasto or DC14 mentioned above), if such effects are considerable for SPARC galaxies, they should be reflected in the best-fit parameters. In section \ref{sec:result:gof} (ii) we compare our statistical results regarding baryonic effects in the DM model free-parameters, with the ones reported in the literature.

%Interestingly \cite{2019MNRAS.483.4086B} reports, in DM-only simulations, a reduction in WDM inner-halo density relative to CDM ($\rho^{Ein}_{\rm WDM}/\rho^{Ein}_{\rm CDM}(r_{in})$) of about $0.7$, which is consistent with an Einasto concentration parameter in the WDM case of about $10$ \citep{2019MNRAS.483.4086B. Even when no explicit reference to the $\alpha$ parameter is there given for such a concentration, it must be somewhat larger than $\alpha=0.2$ in order to cause the above density-reduction (recall $\alpha=0.2$ is a best fit parameter to standard CDM halos: see e.g. fig. 2 in \citealp{2014MNRAS.441.3359D}). Our results are then consistent with these values for the Einasto case (see \cref{fig:parameter-distribution:einasto})